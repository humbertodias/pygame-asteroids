%
% API Documentation for API Documentation
% Module time
%
% Generated by epydoc 3.0.1
% [Thu Dec 10 09:57:58 2015]
%

%%%%%%%%%%%%%%%%%%%%%%%%%%%%%%%%%%%%%%%%%%%%%%%%%%%%%%%%%%%%%%%%%%%%%%%%%%%
%%                          Module Description                           %%
%%%%%%%%%%%%%%%%%%%%%%%%%%%%%%%%%%%%%%%%%%%%%%%%%%%%%%%%%%%%%%%%%%%%%%%%%%%

    \index{time \textit{(module)}|(}
\section{Module time}

    \label{time}
\begin{alltt}
This module provides various functions to manipulate time values.

There are two standard representations of time.  One is the number
of seconds since the Epoch, in UTC (a.k.a. GMT).  It may be an integer
or a floating point number (to represent fractions of seconds).
The Epoch is system-defined; on Unix, it is generally January 1st, 1970.
The actual value can be retrieved by calling gmtime(0).

The other representation is a tuple of 9 integers giving local time.
The tuple items are:
  year (four digits, e.g. 1998)
  month (1-12)
  day (1-31)
  hours (0-23)
  minutes (0-59)
  seconds (0-59)
  weekday (0-6, Monday is 0)
  Julian day (day in the year, 1-366)
  DST (Daylight Savings Time) flag (-1, 0 or 1)
If the DST flag is 0, the time is given in the regular time zone;
if it is 1, the time is given in the DST time zone;
if it is -1, mktime() should guess based on the date and time.

Variables:

timezone -- difference in seconds between UTC and local standard time
altzone -- difference in  seconds between UTC and local DST time
daylight -- whether local time should reflect DST
tzname -- tuple of (standard time zone name, DST time zone name)

Functions:

time() -- return current time in seconds since the Epoch as a float
clock() -- return CPU time since process start as a float
sleep() -- delay for a number of seconds given as a float
gmtime() -- convert seconds since Epoch to UTC tuple
localtime() -- convert seconds since Epoch to local time tuple
asctime() -- convert time tuple to string
ctime() -- convert time in seconds to string
mktime() -- convert local time tuple to seconds since Epoch
strftime() -- convert time tuple to string according to format specification
strptime() -- parse string to time tuple according to format specification
tzset() -- change the local timezone
\end{alltt}


%%%%%%%%%%%%%%%%%%%%%%%%%%%%%%%%%%%%%%%%%%%%%%%%%%%%%%%%%%%%%%%%%%%%%%%%%%%
%%                               Functions                               %%
%%%%%%%%%%%%%%%%%%%%%%%%%%%%%%%%%%%%%%%%%%%%%%%%%%%%%%%%%%%%%%%%%%%%%%%%%%%

  \subsection{Functions}

    \label{time:asctime}
    \index{time.asctime \textit{(function)}}

    \vspace{0.5ex}

\hspace{.8\funcindent}\begin{boxedminipage}{\funcwidth}

    \raggedright \textbf{asctime}(\textit{tuple}={\tt ...})

    \vspace{-1.5ex}

    \rule{\textwidth}{0.5\fboxrule}
\setlength{\parskip}{2ex}
    Convert a time tuple to a string, e.g. 'Sat Jun 06 16:26:11 1998'. When
    the time tuple is not present, current time as returned by localtime() 
    is used.

\setlength{\parskip}{1ex}
      \textbf{Return Value}
    \vspace{-1ex}

      \begin{quote}
      string

      \end{quote}

    \end{boxedminipage}

    \label{time:clock}
    \index{time.clock \textit{(function)}}

    \vspace{0.5ex}

\hspace{.8\funcindent}\begin{boxedminipage}{\funcwidth}

    \raggedright \textbf{clock}()

    \vspace{-1.5ex}

    \rule{\textwidth}{0.5\fboxrule}
\setlength{\parskip}{2ex}
    Return the CPU time or real time since the start of the process or 
    since the first call to clock().  This has as much precision as the 
    system records.

\setlength{\parskip}{1ex}
      \textbf{Return Value}
    \vspace{-1ex}

      \begin{quote}
      floating point number

      \end{quote}

    \end{boxedminipage}

    \label{time:ctime}
    \index{time.ctime \textit{(function)}}

    \vspace{0.5ex}

\hspace{.8\funcindent}\begin{boxedminipage}{\funcwidth}

    \raggedright \textbf{ctime}(\textit{seconds})

    \vspace{-1.5ex}

    \rule{\textwidth}{0.5\fboxrule}
\setlength{\parskip}{2ex}
    Convert a time in seconds since the Epoch to a string in local time. 
    This is equivalent to asctime(localtime(seconds)). When the time tuple 
    is not present, current time as returned by localtime() is used.

\setlength{\parskip}{1ex}
      \textbf{Return Value}
    \vspace{-1ex}

      \begin{quote}
      string

      \end{quote}

    \end{boxedminipage}

    \label{time:gmtime}
    \index{time.gmtime \textit{(function)}}

    \vspace{0.5ex}

\hspace{.8\funcindent}\begin{boxedminipage}{\funcwidth}

    \raggedright \textbf{gmtime}(\textit{seconds}={\tt ...})

    \vspace{-1.5ex}

    \rule{\textwidth}{0.5\fboxrule}
\setlength{\parskip}{2ex}
\begin{alltt}
                       tm\_sec, tm\_wday, tm\_yday, tm\_isdst)

Convert seconds since the Epoch to a time tuple expressing UTC (a.k.a.
GMT).  When 'seconds' is not passed in, convert the current time instead.
\end{alltt}

\setlength{\parskip}{1ex}
      \textbf{Return Value}
    \vspace{-1ex}

      \begin{quote}
      (tm\_year, tm\_mon, tm\_mday, tm\_hour, tm\_min,

      \end{quote}

    \end{boxedminipage}

    \label{time:localtime}
    \index{time.localtime \textit{(function)}}

    \vspace{0.5ex}

\hspace{.8\funcindent}\begin{boxedminipage}{\funcwidth}

    \raggedright \textbf{localtime}(\textit{seconds}={\tt ...})

    \vspace{-1.5ex}

    \rule{\textwidth}{0.5\fboxrule}
\setlength{\parskip}{2ex}
\begin{alltt}
                          tm\_sec,tm\_wday,tm\_yday,tm\_isdst)

Convert seconds since the Epoch to a time tuple expressing local time.
When 'seconds' is not passed in, convert the current time instead.
\end{alltt}

\setlength{\parskip}{1ex}
      \textbf{Return Value}
    \vspace{-1ex}

      \begin{quote}
      (tm\_year,tm\_mon,tm\_mday,tm\_hour,tm\_min,

      \end{quote}

    \end{boxedminipage}

    \label{time:mktime}
    \index{time.mktime \textit{(function)}}

    \vspace{0.5ex}

\hspace{.8\funcindent}\begin{boxedminipage}{\funcwidth}

    \raggedright \textbf{mktime}(\textit{tuple})

    \vspace{-1.5ex}

    \rule{\textwidth}{0.5\fboxrule}
\setlength{\parskip}{2ex}
    Convert a time tuple in local time to seconds since the Epoch.

\setlength{\parskip}{1ex}
      \textbf{Return Value}
    \vspace{-1ex}

      \begin{quote}
      floating point number

      \end{quote}

    \end{boxedminipage}

    \label{time:sleep}
    \index{time.sleep \textit{(function)}}

    \vspace{0.5ex}

\hspace{.8\funcindent}\begin{boxedminipage}{\funcwidth}

    \raggedright \textbf{sleep}(\textit{seconds})

    \vspace{-1.5ex}

    \rule{\textwidth}{0.5\fboxrule}
\setlength{\parskip}{2ex}
    Delay execution for a given number of seconds.  The argument may be a 
    floating point number for subsecond precision.

\setlength{\parskip}{1ex}
    \end{boxedminipage}

    \label{time:strftime}
    \index{time.strftime \textit{(function)}}

    \vspace{0.5ex}

\hspace{.8\funcindent}\begin{boxedminipage}{\funcwidth}

    \raggedright \textbf{strftime}(\textit{format}, \textit{tuple}={\tt ...})

    \vspace{-1.5ex}

    \rule{\textwidth}{0.5\fboxrule}
\setlength{\parskip}{2ex}
    Convert a time tuple to a string according to a format specification. 
    See the library reference manual for formatting codes. When the time 
    tuple is not present, current time as returned by localtime() is used.

\setlength{\parskip}{1ex}
      \textbf{Return Value}
    \vspace{-1ex}

      \begin{quote}
      string

      \end{quote}

    \end{boxedminipage}

    \label{time:strptime}
    \index{time.strptime \textit{(function)}}

    \vspace{0.5ex}

\hspace{.8\funcindent}\begin{boxedminipage}{\funcwidth}

    \raggedright \textbf{strptime}(\textit{string}, \textit{format})

    \vspace{-1.5ex}

    \rule{\textwidth}{0.5\fboxrule}
\setlength{\parskip}{2ex}
    Parse a string to a time tuple according to a format specification. See
    the library reference manual for formatting codes (same as strftime()).

\setlength{\parskip}{1ex}
      \textbf{Return Value}
    \vspace{-1ex}

      \begin{quote}
      struct\_time

      \end{quote}

    \end{boxedminipage}

    \label{time:time}
    \index{time.time \textit{(function)}}

    \vspace{0.5ex}

\hspace{.8\funcindent}\begin{boxedminipage}{\funcwidth}

    \raggedright \textbf{time}()

    \vspace{-1.5ex}

    \rule{\textwidth}{0.5\fboxrule}
\setlength{\parskip}{2ex}
    Return the current time in seconds since the Epoch. Fractions of a 
    second may be present if the system clock provides them.

\setlength{\parskip}{1ex}
      \textbf{Return Value}
    \vspace{-1ex}

      \begin{quote}
      floating point number

      \end{quote}

    \end{boxedminipage}

    \label{time:tzset}
    \index{time.tzset \textit{(function)}}

    \vspace{0.5ex}

\hspace{.8\funcindent}\begin{boxedminipage}{\funcwidth}

    \raggedright \textbf{tzset}()

    \vspace{-1.5ex}

    \rule{\textwidth}{0.5\fboxrule}
\setlength{\parskip}{2ex}
    Initialize, or reinitialize, the local timezone to the value stored in 
    os.environ['TZ']. The TZ environment variable should be specified in 
    standard Unix timezone format as documented in the tzset man page (eg. 
    'US/Eastern', 'Europe/Amsterdam'). Unknown timezones will silently fall
    back to UTC. If the TZ environment variable is not set, the local 
    timezone is set to the systems best guess of wallclock time. Changing 
    the TZ environment variable without calling tzset *may* change the 
    local timezone used by methods such as localtime, but this behaviour 
    should not be relied on.

\setlength{\parskip}{1ex}
    \end{boxedminipage}


%%%%%%%%%%%%%%%%%%%%%%%%%%%%%%%%%%%%%%%%%%%%%%%%%%%%%%%%%%%%%%%%%%%%%%%%%%%
%%                               Variables                               %%
%%%%%%%%%%%%%%%%%%%%%%%%%%%%%%%%%%%%%%%%%%%%%%%%%%%%%%%%%%%%%%%%%%%%%%%%%%%

  \subsection{Variables}

    \vspace{-1cm}
\hspace{\varindent}\begin{longtable}{|p{\varnamewidth}|p{\vardescrwidth}|l}
\cline{1-2}
\cline{1-2} \centering \textbf{Name} & \centering \textbf{Description}& \\
\cline{1-2}
\endhead\cline{1-2}\multicolumn{3}{r}{\small\textit{continued on next page}}\\\endfoot\cline{1-2}
\endlastfoot\raggedright \_\-\_\-p\-a\-c\-k\-a\-g\-e\-\_\-\_\- & \raggedright \textbf{Value:} 
{\tt None}&\\
\cline{1-2}
\raggedright a\-c\-c\-e\-p\-t\-2\-d\-y\-e\-a\-r\- & \raggedright \textbf{Value:} 
{\tt 1}&\\
\cline{1-2}
\raggedright a\-l\-t\-z\-o\-n\-e\- & \raggedright \textbf{Value:} 
{\tt 7200}&\\
\cline{1-2}
\raggedright d\-a\-y\-l\-i\-g\-h\-t\- & \raggedright \textbf{Value:} 
{\tt 1}&\\
\cline{1-2}
\raggedright t\-i\-m\-e\-z\-o\-n\-e\- & \raggedright \textbf{Value:} 
{\tt 10800}&\\
\cline{1-2}
\raggedright t\-z\-n\-a\-m\-e\- & \raggedright \textbf{Value:} 
{\tt \texttt{(}\texttt{'}\texttt{BRT}\texttt{'}\texttt{, }\texttt{'}\texttt{BRST}\texttt{'}\texttt{)}}&\\
\cline{1-2}
\end{longtable}


%%%%%%%%%%%%%%%%%%%%%%%%%%%%%%%%%%%%%%%%%%%%%%%%%%%%%%%%%%%%%%%%%%%%%%%%%%%
%%                           Class Description                           %%
%%%%%%%%%%%%%%%%%%%%%%%%%%%%%%%%%%%%%%%%%%%%%%%%%%%%%%%%%%%%%%%%%%%%%%%%%%%

    \index{time.struct\_time \textit{(class)}|(}
\subsection{Class struct\_time}

    \label{time:struct_time}
\begin{tabular}{cccccc}
% Line for object, linespec=[False]
\multicolumn{2}{r}{\settowidth{\BCL}{object}\multirow{2}{\BCL}{object}}
&&
  \\\cline{3-3}
  &&\multicolumn{1}{c|}{}
&&
  \\
&&\multicolumn{2}{l}{\textbf{time.struct\_time}}
\end{tabular}

The time value as returned by gmtime(), localtime(), and strptime(), and 
accepted by asctime(), mktime() and strftime().  May be considered as a 
sequence of 9 integers.

Note that several fields' values are not the same as those defined by the C
language standard for struct tm.  For example, the value of the field 
tm\_year is the actual year, not year - 1900.  See individual fields' 
descriptions for details.


%%%%%%%%%%%%%%%%%%%%%%%%%%%%%%%%%%%%%%%%%%%%%%%%%%%%%%%%%%%%%%%%%%%%%%%%%%%
%%                                Methods                                %%
%%%%%%%%%%%%%%%%%%%%%%%%%%%%%%%%%%%%%%%%%%%%%%%%%%%%%%%%%%%%%%%%%%%%%%%%%%%

  \subsubsection{Methods}

    \label{time:struct_time:__add__}
    \index{time.struct\_time.\_\_add\_\_ \textit{(function)}}

    \vspace{0.5ex}

\hspace{.8\funcindent}\begin{boxedminipage}{\funcwidth}

    \raggedright \textbf{\_\_add\_\_}(\textit{x}, \textit{y})

    \vspace{-1.5ex}

    \rule{\textwidth}{0.5\fboxrule}
\setlength{\parskip}{2ex}
    x+y

\setlength{\parskip}{1ex}
    \end{boxedminipage}

    \label{time:struct_time:__contains__}
    \index{time.struct\_time.\_\_contains\_\_ \textit{(function)}}

    \vspace{0.5ex}

\hspace{.8\funcindent}\begin{boxedminipage}{\funcwidth}

    \raggedright \textbf{\_\_contains\_\_}(\textit{x}, \textit{y})

    \vspace{-1.5ex}

    \rule{\textwidth}{0.5\fboxrule}
\setlength{\parskip}{2ex}
    y in x

\setlength{\parskip}{1ex}
    \end{boxedminipage}

    \label{time:struct_time:__eq__}
    \index{time.struct\_time.\_\_eq\_\_ \textit{(function)}}

    \vspace{0.5ex}

\hspace{.8\funcindent}\begin{boxedminipage}{\funcwidth}

    \raggedright \textbf{\_\_eq\_\_}(\textit{x}, \textit{y})

    \vspace{-1.5ex}

    \rule{\textwidth}{0.5\fboxrule}
\setlength{\parskip}{2ex}
    x==y

\setlength{\parskip}{1ex}
    \end{boxedminipage}

    \label{time:struct_time:__ge__}
    \index{time.struct\_time.\_\_ge\_\_ \textit{(function)}}

    \vspace{0.5ex}

\hspace{.8\funcindent}\begin{boxedminipage}{\funcwidth}

    \raggedright \textbf{\_\_ge\_\_}(\textit{x}, \textit{y})

    \vspace{-1.5ex}

    \rule{\textwidth}{0.5\fboxrule}
\setlength{\parskip}{2ex}
    x{\textgreater}=y

\setlength{\parskip}{1ex}
    \end{boxedminipage}

    \label{time:struct_time:__getitem__}
    \index{time.struct\_time.\_\_getitem\_\_ \textit{(function)}}

    \vspace{0.5ex}

\hspace{.8\funcindent}\begin{boxedminipage}{\funcwidth}

    \raggedright \textbf{\_\_getitem\_\_}(\textit{x}, \textit{y})

    \vspace{-1.5ex}

    \rule{\textwidth}{0.5\fboxrule}
\setlength{\parskip}{2ex}
    x[y]

\setlength{\parskip}{1ex}
    \end{boxedminipage}

    \label{time:struct_time:__getslice__}
    \index{time.struct\_time.\_\_getslice\_\_ \textit{(function)}}

    \vspace{0.5ex}

\hspace{.8\funcindent}\begin{boxedminipage}{\funcwidth}

    \raggedright \textbf{\_\_getslice\_\_}(\textit{x}, \textit{i}, \textit{j})

    \vspace{-1.5ex}

    \rule{\textwidth}{0.5\fboxrule}
\setlength{\parskip}{2ex}
    x[i:j]

    Use of negative indices is not supported.

\setlength{\parskip}{1ex}
    \end{boxedminipage}

    \label{time:struct_time:__gt__}
    \index{time.struct\_time.\_\_gt\_\_ \textit{(function)}}

    \vspace{0.5ex}

\hspace{.8\funcindent}\begin{boxedminipage}{\funcwidth}

    \raggedright \textbf{\_\_gt\_\_}(\textit{x}, \textit{y})

    \vspace{-1.5ex}

    \rule{\textwidth}{0.5\fboxrule}
\setlength{\parskip}{2ex}
    x{\textgreater}y

\setlength{\parskip}{1ex}
    \end{boxedminipage}

    \vspace{0.5ex}

\hspace{.8\funcindent}\begin{boxedminipage}{\funcwidth}

    \raggedright \textbf{\_\_hash\_\_}(\textit{x})

    \vspace{-1.5ex}

    \rule{\textwidth}{0.5\fboxrule}
\setlength{\parskip}{2ex}
    hash(x)

\setlength{\parskip}{1ex}
      Overrides: object.\_\_hash\_\_

    \end{boxedminipage}

    \label{time:struct_time:__le__}
    \index{time.struct\_time.\_\_le\_\_ \textit{(function)}}

    \vspace{0.5ex}

\hspace{.8\funcindent}\begin{boxedminipage}{\funcwidth}

    \raggedright \textbf{\_\_le\_\_}(\textit{x}, \textit{y})

    \vspace{-1.5ex}

    \rule{\textwidth}{0.5\fboxrule}
\setlength{\parskip}{2ex}
    x{\textless}=y

\setlength{\parskip}{1ex}
    \end{boxedminipage}

    \label{time:struct_time:__len__}
    \index{time.struct\_time.\_\_len\_\_ \textit{(function)}}

    \vspace{0.5ex}

\hspace{.8\funcindent}\begin{boxedminipage}{\funcwidth}

    \raggedright \textbf{\_\_len\_\_}(\textit{x})

    \vspace{-1.5ex}

    \rule{\textwidth}{0.5\fboxrule}
\setlength{\parskip}{2ex}
    len(x)

\setlength{\parskip}{1ex}
    \end{boxedminipage}

    \label{time:struct_time:__lt__}
    \index{time.struct\_time.\_\_lt\_\_ \textit{(function)}}

    \vspace{0.5ex}

\hspace{.8\funcindent}\begin{boxedminipage}{\funcwidth}

    \raggedright \textbf{\_\_lt\_\_}(\textit{x}, \textit{y})

    \vspace{-1.5ex}

    \rule{\textwidth}{0.5\fboxrule}
\setlength{\parskip}{2ex}
    x{\textless}y

\setlength{\parskip}{1ex}
    \end{boxedminipage}

    \label{time:struct_time:__mul__}
    \index{time.struct\_time.\_\_mul\_\_ \textit{(function)}}

    \vspace{0.5ex}

\hspace{.8\funcindent}\begin{boxedminipage}{\funcwidth}

    \raggedright \textbf{\_\_mul\_\_}(\textit{x}, \textit{n})

    \vspace{-1.5ex}

    \rule{\textwidth}{0.5\fboxrule}
\setlength{\parskip}{2ex}
    x*n

\setlength{\parskip}{1ex}
    \end{boxedminipage}

    \label{time:struct_time:__ne__}
    \index{time.struct\_time.\_\_ne\_\_ \textit{(function)}}

    \vspace{0.5ex}

\hspace{.8\funcindent}\begin{boxedminipage}{\funcwidth}

    \raggedright \textbf{\_\_ne\_\_}(\textit{x}, \textit{y})

    \vspace{-1.5ex}

    \rule{\textwidth}{0.5\fboxrule}
\setlength{\parskip}{2ex}
    x!=y

\setlength{\parskip}{1ex}
    \end{boxedminipage}

    \vspace{0.5ex}

\hspace{.8\funcindent}\begin{boxedminipage}{\funcwidth}

    \raggedright \textbf{\_\_new\_\_}(\textit{T}, \textit{S}, \textit{...})

\setlength{\parskip}{2ex}
\setlength{\parskip}{1ex}
      \textbf{Return Value}
    \vspace{-1ex}

      \begin{quote}
      a new object with type S, a subtype of T

      \end{quote}

      Overrides: object.\_\_new\_\_

    \end{boxedminipage}

    \vspace{0.5ex}

\hspace{.8\funcindent}\begin{boxedminipage}{\funcwidth}

    \raggedright \textbf{\_\_reduce\_\_}(\textit{...})

\setlength{\parskip}{2ex}
    helper for pickle

\setlength{\parskip}{1ex}
      Overrides: object.\_\_reduce\_\_ 	extit{(inherited documentation)}

    \end{boxedminipage}

    \vspace{0.5ex}

\hspace{.8\funcindent}\begin{boxedminipage}{\funcwidth}

    \raggedright \textbf{\_\_repr\_\_}(\textit{x})

    \vspace{-1.5ex}

    \rule{\textwidth}{0.5\fboxrule}
\setlength{\parskip}{2ex}
    repr(x)

\setlength{\parskip}{1ex}
      Overrides: object.\_\_repr\_\_

    \end{boxedminipage}

    \label{time:struct_time:__rmul__}
    \index{time.struct\_time.\_\_rmul\_\_ \textit{(function)}}

    \vspace{0.5ex}

\hspace{.8\funcindent}\begin{boxedminipage}{\funcwidth}

    \raggedright \textbf{\_\_rmul\_\_}(\textit{x}, \textit{n})

    \vspace{-1.5ex}

    \rule{\textwidth}{0.5\fboxrule}
\setlength{\parskip}{2ex}
    n*x

\setlength{\parskip}{1ex}
    \end{boxedminipage}


\large{\textbf{\textit{Inherited from object}}}

\begin{quote}
\_\_delattr\_\_(), \_\_format\_\_(), \_\_getattribute\_\_(), \_\_init\_\_(), \_\_reduce\_ex\_\_(), \_\_setattr\_\_(), \_\_sizeof\_\_(), \_\_str\_\_(), \_\_subclasshook\_\_()
\end{quote}

%%%%%%%%%%%%%%%%%%%%%%%%%%%%%%%%%%%%%%%%%%%%%%%%%%%%%%%%%%%%%%%%%%%%%%%%%%%
%%                              Properties                               %%
%%%%%%%%%%%%%%%%%%%%%%%%%%%%%%%%%%%%%%%%%%%%%%%%%%%%%%%%%%%%%%%%%%%%%%%%%%%

  \subsubsection{Properties}

    \vspace{-1cm}
\hspace{\varindent}\begin{longtable}{|p{\varnamewidth}|p{\vardescrwidth}|l}
\cline{1-2}
\cline{1-2} \centering \textbf{Name} & \centering \textbf{Description}& \\
\cline{1-2}
\endhead\cline{1-2}\multicolumn{3}{r}{\small\textit{continued on next page}}\\\endfoot\cline{1-2}
\endlastfoot\raggedright t\-m\-\_\-h\-o\-u\-r\- & \raggedright hours, range [0, 23]&\\
\cline{1-2}
\raggedright t\-m\-\_\-i\-s\-d\-s\-t\- & \raggedright 1 if summer time is in effect, 0 if not, and -1 if unknown&\\
\cline{1-2}
\raggedright t\-m\-\_\-m\-d\-a\-y\- & \raggedright day of month, range [1, 31]&\\
\cline{1-2}
\raggedright t\-m\-\_\-m\-i\-n\- & \raggedright minutes, range [0, 59]&\\
\cline{1-2}
\raggedright t\-m\-\_\-m\-o\-n\- & \raggedright month of year, range [1, 12]&\\
\cline{1-2}
\raggedright t\-m\-\_\-s\-e\-c\- & \raggedright seconds, range [0, 61])&\\
\cline{1-2}
\raggedright t\-m\-\_\-w\-d\-a\-y\- & \raggedright day of week, range [0, 6], Monday is 0&\\
\cline{1-2}
\raggedright t\-m\-\_\-y\-d\-a\-y\- & \raggedright day of year, range [1, 366]&\\
\cline{1-2}
\raggedright t\-m\-\_\-y\-e\-a\-r\- & \raggedright year, for example, 1993&\\
\cline{1-2}
\multicolumn{2}{|l|}{\textit{Inherited from object}}\\
\multicolumn{2}{|p{\varwidth}|}{\raggedright \_\_class\_\_}\\
\cline{1-2}
\end{longtable}


%%%%%%%%%%%%%%%%%%%%%%%%%%%%%%%%%%%%%%%%%%%%%%%%%%%%%%%%%%%%%%%%%%%%%%%%%%%
%%                            Class Variables                            %%
%%%%%%%%%%%%%%%%%%%%%%%%%%%%%%%%%%%%%%%%%%%%%%%%%%%%%%%%%%%%%%%%%%%%%%%%%%%

  \subsubsection{Class Variables}

    \vspace{-1cm}
\hspace{\varindent}\begin{longtable}{|p{\varnamewidth}|p{\vardescrwidth}|l}
\cline{1-2}
\cline{1-2} \centering \textbf{Name} & \centering \textbf{Description}& \\
\cline{1-2}
\endhead\cline{1-2}\multicolumn{3}{r}{\small\textit{continued on next page}}\\\endfoot\cline{1-2}
\endlastfoot\raggedright n\-\_\-f\-i\-e\-l\-d\-s\- & \raggedright \textbf{Value:} 
{\tt 9}&\\
\cline{1-2}
\raggedright n\-\_\-s\-e\-q\-u\-e\-n\-c\-e\-\_\-f\-i\-e\-l\-d\-s\- & \raggedright \textbf{Value:} 
{\tt 9}&\\
\cline{1-2}
\raggedright n\-\_\-u\-n\-n\-a\-m\-e\-d\-\_\-f\-i\-e\-l\-d\-s\- & \raggedright \textbf{Value:} 
{\tt 0}&\\
\cline{1-2}
\end{longtable}

    \index{time.struct\_time \textit{(class)}|)}
    \index{time \textit{(module)}|)}
