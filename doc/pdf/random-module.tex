%
% API Documentation for API Documentation
% Module random
%
% Generated by epydoc 3.0.1
% [Thu Dec 10 09:57:58 2015]
%

%%%%%%%%%%%%%%%%%%%%%%%%%%%%%%%%%%%%%%%%%%%%%%%%%%%%%%%%%%%%%%%%%%%%%%%%%%%
%%                          Module Description                           %%
%%%%%%%%%%%%%%%%%%%%%%%%%%%%%%%%%%%%%%%%%%%%%%%%%%%%%%%%%%%%%%%%%%%%%%%%%%%

    \index{random \textit{(module)}|(}
\section{Module random}

    \label{random}
\begin{alltt}
Random variable generators.

    integers
    --------
           uniform within range

    sequences
    ---------
           pick random element
           pick random sample
           generate random permutation

    distributions on the real line:
    ------------------------------
           uniform
           triangular
           normal (Gaussian)
           lognormal
           negative exponential
           gamma
           beta
           pareto
           Weibull

    distributions on the circle (angles 0 to 2pi)
    ---------------------------------------------
           circular uniform
           von Mises

General notes on the underlying Mersenne Twister core generator:

* The period is 2**19937-1.
* It is one of the most extensively tested generators in existence.
* Without a direct way to compute N steps forward, the semantics of
  jumpahead(n) are weakened to simply jump to another distant state and rely
  on the large period to avoid overlapping sequences.
* The random() method is implemented in C, executes in a single Python step,
  and is, therefore, threadsafe.
\end{alltt}


%%%%%%%%%%%%%%%%%%%%%%%%%%%%%%%%%%%%%%%%%%%%%%%%%%%%%%%%%%%%%%%%%%%%%%%%%%%
%%                               Functions                               %%
%%%%%%%%%%%%%%%%%%%%%%%%%%%%%%%%%%%%%%%%%%%%%%%%%%%%%%%%%%%%%%%%%%%%%%%%%%%

  \subsection{Functions}

    \label{random:betavariate}
    \index{random.betavariate \textit{(function)}}

    \vspace{0.5ex}

\hspace{.8\funcindent}\begin{boxedminipage}{\funcwidth}

    \raggedright \textbf{betavariate}(\textit{alpha}, \textit{beta})

    \vspace{-1.5ex}

    \rule{\textwidth}{0.5\fboxrule}
\setlength{\parskip}{2ex}
    Beta distribution.

    Conditions on the parameters are alpha {\textgreater} 0 and beta 
    {\textgreater} 0. Returned values range between 0 and 1.

\setlength{\parskip}{1ex}
    \end{boxedminipage}

    \label{random:choice}
    \index{random.choice \textit{(function)}}

    \vspace{0.5ex}

\hspace{.8\funcindent}\begin{boxedminipage}{\funcwidth}

    \raggedright \textbf{choice}(\textit{seq})

    \vspace{-1.5ex}

    \rule{\textwidth}{0.5\fboxrule}
\setlength{\parskip}{2ex}
    Choose a random element from a non-empty sequence.

\setlength{\parskip}{1ex}
    \end{boxedminipage}

    \label{random:expovariate}
    \index{random.expovariate \textit{(function)}}

    \vspace{0.5ex}

\hspace{.8\funcindent}\begin{boxedminipage}{\funcwidth}

    \raggedright \textbf{expovariate}(\textit{lambd})

    \vspace{-1.5ex}

    \rule{\textwidth}{0.5\fboxrule}
\setlength{\parskip}{2ex}
    Exponential distribution.

    lambd is 1.0 divided by the desired mean.  It should be nonzero.  (The 
    parameter would be called "lambda", but that is a reserved word in 
    Python.)  Returned values range from 0 to positive infinity if lambd is
    positive, and from negative infinity to 0 if lambd is negative.

\setlength{\parskip}{1ex}
    \end{boxedminipage}

    \label{random:gammavariate}
    \index{random.gammavariate \textit{(function)}}

    \vspace{0.5ex}

\hspace{.8\funcindent}\begin{boxedminipage}{\funcwidth}

    \raggedright \textbf{gammavariate}(\textit{alpha}, \textit{beta})

    \vspace{-1.5ex}

    \rule{\textwidth}{0.5\fboxrule}
\setlength{\parskip}{2ex}
\begin{alltt}
Gamma distribution.  Not the gamma function!

Conditions on the parameters are alpha {\textgreater} 0 and beta {\textgreater} 0.

The probability distribution function is:

            x ** (alpha - 1) * math.exp(-x / beta)
  pdf(x) =  --------------------------------------
              math.gamma(alpha) * beta ** alpha
\end{alltt}

\setlength{\parskip}{1ex}
    \end{boxedminipage}

    \label{random:gauss}
    \index{random.gauss \textit{(function)}}

    \vspace{0.5ex}

\hspace{.8\funcindent}\begin{boxedminipage}{\funcwidth}

    \raggedright \textbf{gauss}(\textit{mu}, \textit{sigma})

    \vspace{-1.5ex}

    \rule{\textwidth}{0.5\fboxrule}
\setlength{\parskip}{2ex}
    Gaussian distribution.

    mu is the mean, and sigma is the standard deviation.  This is slightly 
    faster than the normalvariate() function.

    Not thread-safe without a lock around calls.

\setlength{\parskip}{1ex}
    \end{boxedminipage}

    \label{random:getrandbits}
    \index{random.getrandbits \textit{(function)}}

    \vspace{0.5ex}

\hspace{.8\funcindent}\begin{boxedminipage}{\funcwidth}

    \raggedright \textbf{getrandbits}(\textit{k})

    \vspace{-1.5ex}

    \rule{\textwidth}{0.5\fboxrule}
\setlength{\parskip}{2ex}
    Generates a long int with k random bits.

\setlength{\parskip}{1ex}
      \textbf{Return Value}
    \vspace{-1ex}

      \begin{quote}
      x

      \end{quote}

    \end{boxedminipage}

    \label{random:getstate}
    \index{random.getstate \textit{(function)}}

    \vspace{0.5ex}

\hspace{.8\funcindent}\begin{boxedminipage}{\funcwidth}

    \raggedright \textbf{getstate}()

    \vspace{-1.5ex}

    \rule{\textwidth}{0.5\fboxrule}
\setlength{\parskip}{2ex}
    Return internal state; can be passed to setstate() later.

\setlength{\parskip}{1ex}
    \end{boxedminipage}

    \label{random:jumpahead}
    \index{random.jumpahead \textit{(function)}}

    \vspace{0.5ex}

\hspace{.8\funcindent}\begin{boxedminipage}{\funcwidth}

    \raggedright \textbf{jumpahead}(\textit{n})

    \vspace{-1.5ex}

    \rule{\textwidth}{0.5\fboxrule}
\setlength{\parskip}{2ex}
    Change the internal state to one that is likely far away from the 
    current state.  This method will not be in Py3.x, so it is better to 
    simply reseed.

\setlength{\parskip}{1ex}
    \end{boxedminipage}

    \label{random:lognormvariate}
    \index{random.lognormvariate \textit{(function)}}

    \vspace{0.5ex}

\hspace{.8\funcindent}\begin{boxedminipage}{\funcwidth}

    \raggedright \textbf{lognormvariate}(\textit{mu}, \textit{sigma})

    \vspace{-1.5ex}

    \rule{\textwidth}{0.5\fboxrule}
\setlength{\parskip}{2ex}
    Log normal distribution.

    If you take the natural logarithm of this distribution, you'll get a 
    normal distribution with mean mu and standard deviation sigma. mu can 
    have any value, and sigma must be greater than zero.

\setlength{\parskip}{1ex}
    \end{boxedminipage}

    \label{random:normalvariate}
    \index{random.normalvariate \textit{(function)}}

    \vspace{0.5ex}

\hspace{.8\funcindent}\begin{boxedminipage}{\funcwidth}

    \raggedright \textbf{normalvariate}(\textit{mu}, \textit{sigma})

    \vspace{-1.5ex}

    \rule{\textwidth}{0.5\fboxrule}
\setlength{\parskip}{2ex}
    Normal distribution.

    mu is the mean, and sigma is the standard deviation.

\setlength{\parskip}{1ex}
    \end{boxedminipage}

    \label{random:paretovariate}
    \index{random.paretovariate \textit{(function)}}

    \vspace{0.5ex}

\hspace{.8\funcindent}\begin{boxedminipage}{\funcwidth}

    \raggedright \textbf{paretovariate}(\textit{alpha})

    \vspace{-1.5ex}

    \rule{\textwidth}{0.5\fboxrule}
\setlength{\parskip}{2ex}
    Pareto distribution.  alpha is the shape parameter.

\setlength{\parskip}{1ex}
    \end{boxedminipage}

    \label{random:randint}
    \index{random.randint \textit{(function)}}

    \vspace{0.5ex}

\hspace{.8\funcindent}\begin{boxedminipage}{\funcwidth}

    \raggedright \textbf{randint}(\textit{a}, \textit{b})

    \vspace{-1.5ex}

    \rule{\textwidth}{0.5\fboxrule}
\setlength{\parskip}{2ex}
    Return random integer in range [a, b], including both end points.

\setlength{\parskip}{1ex}
    \end{boxedminipage}

    \label{random:random}
    \index{random.random \textit{(function)}}

    \vspace{0.5ex}

\hspace{.8\funcindent}\begin{boxedminipage}{\funcwidth}

    \raggedright \textbf{random}()

\setlength{\parskip}{2ex}
\setlength{\parskip}{1ex}
      \textbf{Return Value}
    \vspace{-1ex}

      \begin{quote}
      x in the interval [0, 1).

      \end{quote}

    \end{boxedminipage}

    \label{random:randrange}
    \index{random.randrange \textit{(function)}}

    \vspace{0.5ex}

\hspace{.8\funcindent}\begin{boxedminipage}{\funcwidth}

    \raggedright \textbf{randrange}(\textit{start}, \textit{stop}={\tt None}, \textit{step}={\tt 1}, \textit{\_int}={\tt {\textless}type 'int'{\textgreater}}, \textit{\_maxwidth}={\tt 9007199254740992})

    \vspace{-1.5ex}

    \rule{\textwidth}{0.5\fboxrule}
\setlength{\parskip}{2ex}
    Choose a random item from range(start, stop[, step]).

    This fixes the problem with randint() which includes the endpoint; in 
    Python this is usually not what you want.

\setlength{\parskip}{1ex}
    \end{boxedminipage}

    \label{random:sample}
    \index{random.sample \textit{(function)}}

    \vspace{0.5ex}

\hspace{.8\funcindent}\begin{boxedminipage}{\funcwidth}

    \raggedright \textbf{sample}(\textit{population}, \textit{k})

    \vspace{-1.5ex}

    \rule{\textwidth}{0.5\fboxrule}
\setlength{\parskip}{2ex}
    Chooses k unique random elements from a population sequence.

    Returns a new list containing elements from the population while 
    leaving the original population unchanged.  The resulting list is in 
    selection order so that all sub-slices will also be valid random 
    samples.  This allows raffle winners (the sample) to be partitioned 
    into grand prize and second place winners (the subslices).

    Members of the population need not be hashable or unique.  If the 
    population contains repeats, then each occurrence is a possible 
    selection in the sample.

    To choose a sample in a range of integers, use xrange as an argument. 
    This is especially fast and space efficient for sampling from a large 
    population:   sample(xrange(10000000), 60)

\setlength{\parskip}{1ex}
    \end{boxedminipage}

    \label{random:seed}
    \index{random.seed \textit{(function)}}

    \vspace{0.5ex}

\hspace{.8\funcindent}\begin{boxedminipage}{\funcwidth}

    \raggedright \textbf{seed}(\textit{a}={\tt None})

    \vspace{-1.5ex}

    \rule{\textwidth}{0.5\fboxrule}
\setlength{\parskip}{2ex}
    Initialize internal state from hashable object.

    None or no argument seeds from current time or from an operating system
    specific randomness source if available.

    If a is not None or an int or long, hash(a) is used instead.

\setlength{\parskip}{1ex}
    \end{boxedminipage}

    \label{random:setstate}
    \index{random.setstate \textit{(function)}}

    \vspace{0.5ex}

\hspace{.8\funcindent}\begin{boxedminipage}{\funcwidth}

    \raggedright \textbf{setstate}(\textit{state})

    \vspace{-1.5ex}

    \rule{\textwidth}{0.5\fboxrule}
\setlength{\parskip}{2ex}
    Restore internal state from object returned by getstate().

\setlength{\parskip}{1ex}
    \end{boxedminipage}

    \label{random:shuffle}
    \index{random.shuffle \textit{(function)}}

    \vspace{0.5ex}

\hspace{.8\funcindent}\begin{boxedminipage}{\funcwidth}

    \raggedright \textbf{shuffle}(\textit{x}, \textit{random}={\tt None})

    \vspace{-1.5ex}

    \rule{\textwidth}{0.5\fboxrule}
\setlength{\parskip}{2ex}
    x, random=random.random -{\textgreater} shuffle list x in place; return
    None.

    Optional arg random is a 0-argument function returning a random float 
    in [0.0, 1.0); by default, the standard random.random.

\setlength{\parskip}{1ex}
    \end{boxedminipage}

    \label{random:triangular}
    \index{random.triangular \textit{(function)}}

    \vspace{0.5ex}

\hspace{.8\funcindent}\begin{boxedminipage}{\funcwidth}

    \raggedright \textbf{triangular}(\textit{low}={\tt 0.0}, \textit{high}={\tt 1.0}, \textit{mode}={\tt None})

    \vspace{-1.5ex}

    \rule{\textwidth}{0.5\fboxrule}
\setlength{\parskip}{2ex}
    Triangular distribution.

    Continuous distribution bounded by given lower and upper limits, and 
    having a given mode value in-between.

    http://en.wikipedia.org/wiki/Triangular\_distribution

\setlength{\parskip}{1ex}
    \end{boxedminipage}

    \label{random:uniform}
    \index{random.uniform \textit{(function)}}

    \vspace{0.5ex}

\hspace{.8\funcindent}\begin{boxedminipage}{\funcwidth}

    \raggedright \textbf{uniform}(\textit{a}, \textit{b})

    \vspace{-1.5ex}

    \rule{\textwidth}{0.5\fboxrule}
\setlength{\parskip}{2ex}
    Get a random number in the range [a, b) or [a, b] depending on 
    rounding.

\setlength{\parskip}{1ex}
    \end{boxedminipage}

    \label{random:vonmisesvariate}
    \index{random.vonmisesvariate \textit{(function)}}

    \vspace{0.5ex}

\hspace{.8\funcindent}\begin{boxedminipage}{\funcwidth}

    \raggedright \textbf{vonmisesvariate}(\textit{mu}, \textit{kappa})

    \vspace{-1.5ex}

    \rule{\textwidth}{0.5\fboxrule}
\setlength{\parskip}{2ex}
    Circular data distribution.

    mu is the mean angle, expressed in radians between 0 and 2*pi, and 
    kappa is the concentration parameter, which must be greater than or 
    equal to zero.  If kappa is equal to zero, this distribution reduces to
    a uniform random angle over the range 0 to 2*pi.

\setlength{\parskip}{1ex}
    \end{boxedminipage}

    \label{random:weibullvariate}
    \index{random.weibullvariate \textit{(function)}}

    \vspace{0.5ex}

\hspace{.8\funcindent}\begin{boxedminipage}{\funcwidth}

    \raggedright \textbf{weibullvariate}(\textit{alpha}, \textit{beta})

    \vspace{-1.5ex}

    \rule{\textwidth}{0.5\fboxrule}
\setlength{\parskip}{2ex}
    Weibull distribution.

    alpha is the scale parameter and beta is the shape parameter.

\setlength{\parskip}{1ex}
    \end{boxedminipage}


%%%%%%%%%%%%%%%%%%%%%%%%%%%%%%%%%%%%%%%%%%%%%%%%%%%%%%%%%%%%%%%%%%%%%%%%%%%
%%                           Class Description                           %%
%%%%%%%%%%%%%%%%%%%%%%%%%%%%%%%%%%%%%%%%%%%%%%%%%%%%%%%%%%%%%%%%%%%%%%%%%%%

    \index{random.Random \textit{(class)}|(}
\subsection{Class Random}

    \label{random:Random}
\begin{tabular}{cccccccc}
% Line for object, linespec=[False, False]
\multicolumn{2}{r}{\settowidth{\BCL}{object}\multirow{2}{\BCL}{object}}
&&
&&
  \\\cline{3-3}
  &&\multicolumn{1}{c|}{}
&&
&&
  \\
% Line for \_random.Random, linespec=[False]
\multicolumn{4}{r}{\settowidth{\BCL}{\_random.Random}\multirow{2}{\BCL}{\_random.Random}}
&&
  \\\cline{5-5}
  &&&&\multicolumn{1}{c|}{}
&&
  \\
&&&&\multicolumn{2}{l}{\textbf{random.Random}}
\end{tabular}

\textbf{Known Subclasses:}
random.SystemRandom,
    random.WichmannHill

Random number generator base class used by bound module functions.

Used to instantiate instances of Random to get generators that don't share 
state.  Especially useful for multi-threaded programs, creating a different
instance of Random for each thread, and using the jumpahead() method to 
ensure that the generated sequences seen by each thread don't overlap.

Class Random can also be subclassed if you want to use a different basic 
generator of your own devising: in that case, override the following 
methods: random(), seed(), getstate(), setstate() and jumpahead(). 
Optionally, implement a getrandbits() method so that randrange() can cover 
arbitrarily large ranges.


%%%%%%%%%%%%%%%%%%%%%%%%%%%%%%%%%%%%%%%%%%%%%%%%%%%%%%%%%%%%%%%%%%%%%%%%%%%
%%                                Methods                                %%
%%%%%%%%%%%%%%%%%%%%%%%%%%%%%%%%%%%%%%%%%%%%%%%%%%%%%%%%%%%%%%%%%%%%%%%%%%%

  \subsubsection{Methods}

    \label{random:Random:__getstate__}
    \index{random.Random.\_\_getstate\_\_ \textit{(function)}}

    \vspace{0.5ex}

\hspace{.8\funcindent}\begin{boxedminipage}{\funcwidth}

    \raggedright \textbf{\_\_getstate\_\_}(\textit{self})

\setlength{\parskip}{2ex}
\setlength{\parskip}{1ex}
    \end{boxedminipage}

    \vspace{0.5ex}

\hspace{.8\funcindent}\begin{boxedminipage}{\funcwidth}

    \raggedright \textbf{\_\_init\_\_}(\textit{self}, \textit{x}={\tt None})

    \vspace{-1.5ex}

    \rule{\textwidth}{0.5\fboxrule}
\setlength{\parskip}{2ex}
    Initialize an instance.

    Optional argument x controls seeding, as for Random.seed().

\setlength{\parskip}{1ex}
      Overrides: object.\_\_init\_\_

    \end{boxedminipage}

    \vspace{0.5ex}

\hspace{.8\funcindent}\begin{boxedminipage}{\funcwidth}

    \raggedright \textbf{\_\_reduce\_\_}(\textit{self})

\setlength{\parskip}{2ex}
    helper for pickle

\setlength{\parskip}{1ex}
      Overrides: object.\_\_reduce\_\_ 	extit{(inherited documentation)}

    \end{boxedminipage}

    \label{random:Random:__setstate__}
    \index{random.Random.\_\_setstate\_\_ \textit{(function)}}

    \vspace{0.5ex}

\hspace{.8\funcindent}\begin{boxedminipage}{\funcwidth}

    \raggedright \textbf{\_\_setstate\_\_}(\textit{self}, \textit{state})

\setlength{\parskip}{2ex}
\setlength{\parskip}{1ex}
    \end{boxedminipage}

    \label{random:Random:betavariate}
    \index{random.Random.betavariate \textit{(function)}}

    \vspace{0.5ex}

\hspace{.8\funcindent}\begin{boxedminipage}{\funcwidth}

    \raggedright \textbf{betavariate}(\textit{self}, \textit{alpha}, \textit{beta})

    \vspace{-1.5ex}

    \rule{\textwidth}{0.5\fboxrule}
\setlength{\parskip}{2ex}
    Beta distribution.

    Conditions on the parameters are alpha {\textgreater} 0 and beta 
    {\textgreater} 0. Returned values range between 0 and 1.

\setlength{\parskip}{1ex}
    \end{boxedminipage}

    \label{random:Random:choice}
    \index{random.Random.choice \textit{(function)}}

    \vspace{0.5ex}

\hspace{.8\funcindent}\begin{boxedminipage}{\funcwidth}

    \raggedright \textbf{choice}(\textit{self}, \textit{seq})

    \vspace{-1.5ex}

    \rule{\textwidth}{0.5\fboxrule}
\setlength{\parskip}{2ex}
    Choose a random element from a non-empty sequence.

\setlength{\parskip}{1ex}
    \end{boxedminipage}

    \label{random:Random:expovariate}
    \index{random.Random.expovariate \textit{(function)}}

    \vspace{0.5ex}

\hspace{.8\funcindent}\begin{boxedminipage}{\funcwidth}

    \raggedright \textbf{expovariate}(\textit{self}, \textit{lambd})

    \vspace{-1.5ex}

    \rule{\textwidth}{0.5\fboxrule}
\setlength{\parskip}{2ex}
    Exponential distribution.

    lambd is 1.0 divided by the desired mean.  It should be nonzero.  (The 
    parameter would be called "lambda", but that is a reserved word in 
    Python.)  Returned values range from 0 to positive infinity if lambd is
    positive, and from negative infinity to 0 if lambd is negative.

\setlength{\parskip}{1ex}
    \end{boxedminipage}

    \label{random:Random:gammavariate}
    \index{random.Random.gammavariate \textit{(function)}}

    \vspace{0.5ex}

\hspace{.8\funcindent}\begin{boxedminipage}{\funcwidth}

    \raggedright \textbf{gammavariate}(\textit{self}, \textit{alpha}, \textit{beta})

    \vspace{-1.5ex}

    \rule{\textwidth}{0.5\fboxrule}
\setlength{\parskip}{2ex}
\begin{alltt}
Gamma distribution.  Not the gamma function!

Conditions on the parameters are alpha {\textgreater} 0 and beta {\textgreater} 0.

The probability distribution function is:

            x ** (alpha - 1) * math.exp(-x / beta)
  pdf(x) =  --------------------------------------
              math.gamma(alpha) * beta ** alpha
\end{alltt}

\setlength{\parskip}{1ex}
    \end{boxedminipage}

    \label{random:Random:gauss}
    \index{random.Random.gauss \textit{(function)}}

    \vspace{0.5ex}

\hspace{.8\funcindent}\begin{boxedminipage}{\funcwidth}

    \raggedright \textbf{gauss}(\textit{self}, \textit{mu}, \textit{sigma})

    \vspace{-1.5ex}

    \rule{\textwidth}{0.5\fboxrule}
\setlength{\parskip}{2ex}
    Gaussian distribution.

    mu is the mean, and sigma is the standard deviation.  This is slightly 
    faster than the normalvariate() function.

    Not thread-safe without a lock around calls.

\setlength{\parskip}{1ex}
    \end{boxedminipage}

    \vspace{0.5ex}

\hspace{.8\funcindent}\begin{boxedminipage}{\funcwidth}

    \raggedright \textbf{getstate}(\textit{self})

    \vspace{-1.5ex}

    \rule{\textwidth}{0.5\fboxrule}
\setlength{\parskip}{2ex}
    Return internal state; can be passed to setstate() later.

\setlength{\parskip}{1ex}
      \textbf{Return Value}
    \vspace{-1ex}

      \begin{quote}
      tuple containing the current state.

      \end{quote}

      Overrides: \_random.Random.getstate

    \end{boxedminipage}

    \vspace{0.5ex}

\hspace{.8\funcindent}\begin{boxedminipage}{\funcwidth}

    \raggedright \textbf{jumpahead}(\textit{self}, \textit{n})

    \vspace{-1.5ex}

    \rule{\textwidth}{0.5\fboxrule}
\setlength{\parskip}{2ex}
    Change the internal state to one that is likely far away from the 
    current state.  This method will not be in Py3.x, so it is better to 
    simply reseed.

\setlength{\parskip}{1ex}
      \textbf{Return Value}
    \vspace{-1ex}

      \begin{quote}
      None

      \end{quote}

      Overrides: \_random.Random.jumpahead

    \end{boxedminipage}

    \label{random:Random:lognormvariate}
    \index{random.Random.lognormvariate \textit{(function)}}

    \vspace{0.5ex}

\hspace{.8\funcindent}\begin{boxedminipage}{\funcwidth}

    \raggedright \textbf{lognormvariate}(\textit{self}, \textit{mu}, \textit{sigma})

    \vspace{-1.5ex}

    \rule{\textwidth}{0.5\fboxrule}
\setlength{\parskip}{2ex}
    Log normal distribution.

    If you take the natural logarithm of this distribution, you'll get a 
    normal distribution with mean mu and standard deviation sigma. mu can 
    have any value, and sigma must be greater than zero.

\setlength{\parskip}{1ex}
    \end{boxedminipage}

    \label{random:Random:normalvariate}
    \index{random.Random.normalvariate \textit{(function)}}

    \vspace{0.5ex}

\hspace{.8\funcindent}\begin{boxedminipage}{\funcwidth}

    \raggedright \textbf{normalvariate}(\textit{self}, \textit{mu}, \textit{sigma})

    \vspace{-1.5ex}

    \rule{\textwidth}{0.5\fboxrule}
\setlength{\parskip}{2ex}
    Normal distribution.

    mu is the mean, and sigma is the standard deviation.

\setlength{\parskip}{1ex}
    \end{boxedminipage}

    \label{random:Random:paretovariate}
    \index{random.Random.paretovariate \textit{(function)}}

    \vspace{0.5ex}

\hspace{.8\funcindent}\begin{boxedminipage}{\funcwidth}

    \raggedright \textbf{paretovariate}(\textit{self}, \textit{alpha})

    \vspace{-1.5ex}

    \rule{\textwidth}{0.5\fboxrule}
\setlength{\parskip}{2ex}
    Pareto distribution.  alpha is the shape parameter.

\setlength{\parskip}{1ex}
    \end{boxedminipage}

    \label{random:Random:randint}
    \index{random.Random.randint \textit{(function)}}

    \vspace{0.5ex}

\hspace{.8\funcindent}\begin{boxedminipage}{\funcwidth}

    \raggedright \textbf{randint}(\textit{self}, \textit{a}, \textit{b})

    \vspace{-1.5ex}

    \rule{\textwidth}{0.5\fboxrule}
\setlength{\parskip}{2ex}
    Return random integer in range [a, b], including both end points.

\setlength{\parskip}{1ex}
    \end{boxedminipage}

    \label{random:Random:randrange}
    \index{random.Random.randrange \textit{(function)}}

    \vspace{0.5ex}

\hspace{.8\funcindent}\begin{boxedminipage}{\funcwidth}

    \raggedright \textbf{randrange}(\textit{self}, \textit{start}, \textit{stop}={\tt None}, \textit{step}={\tt 1}, \textit{\_int}={\tt {\textless}type 'int'{\textgreater}}, \textit{\_maxwidth}={\tt 9007199254740992})

    \vspace{-1.5ex}

    \rule{\textwidth}{0.5\fboxrule}
\setlength{\parskip}{2ex}
    Choose a random item from range(start, stop[, step]).

    This fixes the problem with randint() which includes the endpoint; in 
    Python this is usually not what you want.

\setlength{\parskip}{1ex}
    \end{boxedminipage}

    \label{random:Random:sample}
    \index{random.Random.sample \textit{(function)}}

    \vspace{0.5ex}

\hspace{.8\funcindent}\begin{boxedminipage}{\funcwidth}

    \raggedright \textbf{sample}(\textit{self}, \textit{population}, \textit{k})

    \vspace{-1.5ex}

    \rule{\textwidth}{0.5\fboxrule}
\setlength{\parskip}{2ex}
    Chooses k unique random elements from a population sequence.

    Returns a new list containing elements from the population while 
    leaving the original population unchanged.  The resulting list is in 
    selection order so that all sub-slices will also be valid random 
    samples.  This allows raffle winners (the sample) to be partitioned 
    into grand prize and second place winners (the subslices).

    Members of the population need not be hashable or unique.  If the 
    population contains repeats, then each occurrence is a possible 
    selection in the sample.

    To choose a sample in a range of integers, use xrange as an argument. 
    This is especially fast and space efficient for sampling from a large 
    population:   sample(xrange(10000000), 60)

\setlength{\parskip}{1ex}
    \end{boxedminipage}

    \vspace{0.5ex}

\hspace{.8\funcindent}\begin{boxedminipage}{\funcwidth}

    \raggedright \textbf{seed}(\textit{self}, \textit{a}={\tt None})

    \vspace{-1.5ex}

    \rule{\textwidth}{0.5\fboxrule}
\setlength{\parskip}{2ex}
    Initialize internal state from hashable object.

    None or no argument seeds from current time or from an operating system
    specific randomness source if available.

    If a is not None or an int or long, hash(a) is used instead.

\setlength{\parskip}{1ex}
      \textbf{Return Value}
    \vspace{-1ex}

      \begin{quote}
      None

      \end{quote}

      Overrides: \_random.Random.seed

    \end{boxedminipage}

    \vspace{0.5ex}

\hspace{.8\funcindent}\begin{boxedminipage}{\funcwidth}

    \raggedright \textbf{setstate}(\textit{self}, \textit{state})

    \vspace{-1.5ex}

    \rule{\textwidth}{0.5\fboxrule}
\setlength{\parskip}{2ex}
    Restore internal state from object returned by getstate().

\setlength{\parskip}{1ex}
      \textbf{Return Value}
    \vspace{-1ex}

      \begin{quote}
      None

      \end{quote}

      Overrides: \_random.Random.setstate

    \end{boxedminipage}

    \label{random:Random:shuffle}
    \index{random.Random.shuffle \textit{(function)}}

    \vspace{0.5ex}

\hspace{.8\funcindent}\begin{boxedminipage}{\funcwidth}

    \raggedright \textbf{shuffle}(\textit{self}, \textit{x}, \textit{random}={\tt None})

    \vspace{-1.5ex}

    \rule{\textwidth}{0.5\fboxrule}
\setlength{\parskip}{2ex}
    x, random=random.random -{\textgreater} shuffle list x in place; return
    None.

    Optional arg random is a 0-argument function returning a random float 
    in [0.0, 1.0); by default, the standard random.random.

\setlength{\parskip}{1ex}
    \end{boxedminipage}

    \label{random:Random:triangular}
    \index{random.Random.triangular \textit{(function)}}

    \vspace{0.5ex}

\hspace{.8\funcindent}\begin{boxedminipage}{\funcwidth}

    \raggedright \textbf{triangular}(\textit{self}, \textit{low}={\tt 0.0}, \textit{high}={\tt 1.0}, \textit{mode}={\tt None})

    \vspace{-1.5ex}

    \rule{\textwidth}{0.5\fboxrule}
\setlength{\parskip}{2ex}
    Triangular distribution.

    Continuous distribution bounded by given lower and upper limits, and 
    having a given mode value in-between.

    http://en.wikipedia.org/wiki/Triangular\_distribution

\setlength{\parskip}{1ex}
    \end{boxedminipage}

    \label{random:Random:uniform}
    \index{random.Random.uniform \textit{(function)}}

    \vspace{0.5ex}

\hspace{.8\funcindent}\begin{boxedminipage}{\funcwidth}

    \raggedright \textbf{uniform}(\textit{self}, \textit{a}, \textit{b})

    \vspace{-1.5ex}

    \rule{\textwidth}{0.5\fboxrule}
\setlength{\parskip}{2ex}
    Get a random number in the range [a, b) or [a, b] depending on 
    rounding.

\setlength{\parskip}{1ex}
    \end{boxedminipage}

    \label{random:Random:vonmisesvariate}
    \index{random.Random.vonmisesvariate \textit{(function)}}

    \vspace{0.5ex}

\hspace{.8\funcindent}\begin{boxedminipage}{\funcwidth}

    \raggedright \textbf{vonmisesvariate}(\textit{self}, \textit{mu}, \textit{kappa})

    \vspace{-1.5ex}

    \rule{\textwidth}{0.5\fboxrule}
\setlength{\parskip}{2ex}
    Circular data distribution.

    mu is the mean angle, expressed in radians between 0 and 2*pi, and 
    kappa is the concentration parameter, which must be greater than or 
    equal to zero.  If kappa is equal to zero, this distribution reduces to
    a uniform random angle over the range 0 to 2*pi.

\setlength{\parskip}{1ex}
    \end{boxedminipage}

    \label{random:Random:weibullvariate}
    \index{random.Random.weibullvariate \textit{(function)}}

    \vspace{0.5ex}

\hspace{.8\funcindent}\begin{boxedminipage}{\funcwidth}

    \raggedright \textbf{weibullvariate}(\textit{self}, \textit{alpha}, \textit{beta})

    \vspace{-1.5ex}

    \rule{\textwidth}{0.5\fboxrule}
\setlength{\parskip}{2ex}
    Weibull distribution.

    alpha is the scale parameter and beta is the shape parameter.

\setlength{\parskip}{1ex}
    \end{boxedminipage}


\large{\textbf{\textit{Inherited from \_random.Random}}}

\begin{quote}
\_\_getattribute\_\_(), \_\_new\_\_(), getrandbits(), random()
\end{quote}

\large{\textbf{\textit{Inherited from object}}}

\begin{quote}
\_\_delattr\_\_(), \_\_format\_\_(), \_\_hash\_\_(), \_\_reduce\_ex\_\_(), \_\_repr\_\_(), \_\_setattr\_\_(), \_\_sizeof\_\_(), \_\_str\_\_(), \_\_subclasshook\_\_()
\end{quote}

%%%%%%%%%%%%%%%%%%%%%%%%%%%%%%%%%%%%%%%%%%%%%%%%%%%%%%%%%%%%%%%%%%%%%%%%%%%
%%                              Properties                               %%
%%%%%%%%%%%%%%%%%%%%%%%%%%%%%%%%%%%%%%%%%%%%%%%%%%%%%%%%%%%%%%%%%%%%%%%%%%%

  \subsubsection{Properties}

    \vspace{-1cm}
\hspace{\varindent}\begin{longtable}{|p{\varnamewidth}|p{\vardescrwidth}|l}
\cline{1-2}
\cline{1-2} \centering \textbf{Name} & \centering \textbf{Description}& \\
\cline{1-2}
\endhead\cline{1-2}\multicolumn{3}{r}{\small\textit{continued on next page}}\\\endfoot\cline{1-2}
\endlastfoot\multicolumn{2}{|l|}{\textit{Inherited from object}}\\
\multicolumn{2}{|p{\varwidth}|}{\raggedright \_\_class\_\_}\\
\cline{1-2}
\end{longtable}


%%%%%%%%%%%%%%%%%%%%%%%%%%%%%%%%%%%%%%%%%%%%%%%%%%%%%%%%%%%%%%%%%%%%%%%%%%%
%%                            Class Variables                            %%
%%%%%%%%%%%%%%%%%%%%%%%%%%%%%%%%%%%%%%%%%%%%%%%%%%%%%%%%%%%%%%%%%%%%%%%%%%%

  \subsubsection{Class Variables}

    \vspace{-1cm}
\hspace{\varindent}\begin{longtable}{|p{\varnamewidth}|p{\vardescrwidth}|l}
\cline{1-2}
\cline{1-2} \centering \textbf{Name} & \centering \textbf{Description}& \\
\cline{1-2}
\endhead\cline{1-2}\multicolumn{3}{r}{\small\textit{continued on next page}}\\\endfoot\cline{1-2}
\endlastfoot\raggedright V\-E\-R\-S\-I\-O\-N\- & \raggedright \textbf{Value:} 
{\tt 3}&\\
\cline{1-2}
\end{longtable}

    \index{random.Random \textit{(class)}|)}

%%%%%%%%%%%%%%%%%%%%%%%%%%%%%%%%%%%%%%%%%%%%%%%%%%%%%%%%%%%%%%%%%%%%%%%%%%%
%%                           Class Description                           %%
%%%%%%%%%%%%%%%%%%%%%%%%%%%%%%%%%%%%%%%%%%%%%%%%%%%%%%%%%%%%%%%%%%%%%%%%%%%

    \index{random.SystemRandom \textit{(class)}|(}
\subsection{Class SystemRandom}

    \label{random:SystemRandom}
\begin{tabular}{cccccccccc}
% Line for object, linespec=[False, False, False]
\multicolumn{2}{r}{\settowidth{\BCL}{object}\multirow{2}{\BCL}{object}}
&&
&&
&&
  \\\cline{3-3}
  &&\multicolumn{1}{c|}{}
&&
&&
&&
  \\
% Line for \_random.Random, linespec=[False, False]
\multicolumn{4}{r}{\settowidth{\BCL}{\_random.Random}\multirow{2}{\BCL}{\_random.Random}}
&&
&&
  \\\cline{5-5}
  &&&&\multicolumn{1}{c|}{}
&&
&&
  \\
% Line for random.Random, linespec=[False]
\multicolumn{6}{r}{\settowidth{\BCL}{random.Random}\multirow{2}{\BCL}{random.Random}}
&&
  \\\cline{7-7}
  &&&&&&\multicolumn{1}{c|}{}
&&
  \\
&&&&&&\multicolumn{2}{l}{\textbf{random.SystemRandom}}
\end{tabular}

\begin{alltt}
Alternate random number generator using sources provided
by the operating system (such as /dev/urandom on Unix or
CryptGenRandom on Windows).

 Not available on all systems (see os.urandom() for details).
\end{alltt}


%%%%%%%%%%%%%%%%%%%%%%%%%%%%%%%%%%%%%%%%%%%%%%%%%%%%%%%%%%%%%%%%%%%%%%%%%%%
%%                                Methods                                %%
%%%%%%%%%%%%%%%%%%%%%%%%%%%%%%%%%%%%%%%%%%%%%%%%%%%%%%%%%%%%%%%%%%%%%%%%%%%

  \subsubsection{Methods}

    \vspace{0.5ex}

\hspace{.8\funcindent}\begin{boxedminipage}{\funcwidth}

    \raggedright \textbf{getrandbits}(\textit{k})

    \vspace{-1.5ex}

    \rule{\textwidth}{0.5\fboxrule}
\setlength{\parskip}{2ex}
    Generates a long int with k random bits.

\setlength{\parskip}{1ex}
      \textbf{Return Value}
    \vspace{-1ex}

      \begin{quote}
      x

      \end{quote}

      Overrides: \_random.Random.getrandbits

    \end{boxedminipage}

    \vspace{0.5ex}

\hspace{.8\funcindent}\begin{boxedminipage}{\funcwidth}

    \raggedright \textbf{getstate}(\textit{self}, *\textit{args}, **\textit{kwds})

    \vspace{-1.5ex}

    \rule{\textwidth}{0.5\fboxrule}
\setlength{\parskip}{2ex}
    Method should not be called for a system random number generator.

\setlength{\parskip}{1ex}
      \textbf{Return Value}
    \vspace{-1ex}

      \begin{quote}
      None

      \end{quote}

      Overrides: \_random.Random.getstate

    \end{boxedminipage}

    \vspace{0.5ex}

\hspace{.8\funcindent}\begin{boxedminipage}{\funcwidth}

    \raggedright \textbf{jumpahead}(\textit{self}, *\textit{args}, **\textit{kwds})

    \vspace{-1.5ex}

    \rule{\textwidth}{0.5\fboxrule}
\setlength{\parskip}{2ex}
    Stub method.  Not used for a system random number generator.

\setlength{\parskip}{1ex}
      \textbf{Return Value}
    \vspace{-1ex}

      \begin{quote}
      None

      \end{quote}

      Overrides: \_random.Random.jumpahead

    \end{boxedminipage}

    \vspace{0.5ex}

\hspace{.8\funcindent}\begin{boxedminipage}{\funcwidth}

    \raggedright \textbf{random}(\textit{self})

    \vspace{-1.5ex}

    \rule{\textwidth}{0.5\fboxrule}
\setlength{\parskip}{2ex}
    Get the next random number in the range [0.0, 1.0).

\setlength{\parskip}{1ex}
      \textbf{Return Value}
    \vspace{-1ex}

      \begin{quote}
      x in the interval [0, 1).

      \end{quote}

      Overrides: \_random.Random.random

    \end{boxedminipage}

    \vspace{0.5ex}

\hspace{.8\funcindent}\begin{boxedminipage}{\funcwidth}

    \raggedright \textbf{seed}(\textit{self}, *\textit{args}, **\textit{kwds})

    \vspace{-1.5ex}

    \rule{\textwidth}{0.5\fboxrule}
\setlength{\parskip}{2ex}
    Stub method.  Not used for a system random number generator.

\setlength{\parskip}{1ex}
      \textbf{Return Value}
    \vspace{-1ex}

      \begin{quote}
      None

      \end{quote}

      Overrides: \_random.Random.seed

    \end{boxedminipage}

    \vspace{0.5ex}

\hspace{.8\funcindent}\begin{boxedminipage}{\funcwidth}

    \raggedright \textbf{setstate}(\textit{self}, *\textit{args}, **\textit{kwds})

    \vspace{-1.5ex}

    \rule{\textwidth}{0.5\fboxrule}
\setlength{\parskip}{2ex}
    Method should not be called for a system random number generator.

\setlength{\parskip}{1ex}
      \textbf{Return Value}
    \vspace{-1ex}

      \begin{quote}
      None

      \end{quote}

      Overrides: \_random.Random.setstate

    \end{boxedminipage}


\large{\textbf{\textit{Inherited from random.Random\textit{(Section \ref{random:Random})}}}}

\begin{quote}
\_\_getstate\_\_(), \_\_init\_\_(), \_\_reduce\_\_(), \_\_setstate\_\_(), betavariate(), choice(), expovariate(), gammavariate(), gauss(), lognormvariate(), normalvariate(), paretovariate(), randint(), randrange(), sample(), shuffle(), triangular(), uniform(), vonmisesvariate(), weibullvariate()
\end{quote}

\large{\textbf{\textit{Inherited from \_random.Random}}}

\begin{quote}
\_\_getattribute\_\_(), \_\_new\_\_()
\end{quote}

\large{\textbf{\textit{Inherited from object}}}

\begin{quote}
\_\_delattr\_\_(), \_\_format\_\_(), \_\_hash\_\_(), \_\_reduce\_ex\_\_(), \_\_repr\_\_(), \_\_setattr\_\_(), \_\_sizeof\_\_(), \_\_str\_\_(), \_\_subclasshook\_\_()
\end{quote}

%%%%%%%%%%%%%%%%%%%%%%%%%%%%%%%%%%%%%%%%%%%%%%%%%%%%%%%%%%%%%%%%%%%%%%%%%%%
%%                              Properties                               %%
%%%%%%%%%%%%%%%%%%%%%%%%%%%%%%%%%%%%%%%%%%%%%%%%%%%%%%%%%%%%%%%%%%%%%%%%%%%

  \subsubsection{Properties}

    \vspace{-1cm}
\hspace{\varindent}\begin{longtable}{|p{\varnamewidth}|p{\vardescrwidth}|l}
\cline{1-2}
\cline{1-2} \centering \textbf{Name} & \centering \textbf{Description}& \\
\cline{1-2}
\endhead\cline{1-2}\multicolumn{3}{r}{\small\textit{continued on next page}}\\\endfoot\cline{1-2}
\endlastfoot\multicolumn{2}{|l|}{\textit{Inherited from object}}\\
\multicolumn{2}{|p{\varwidth}|}{\raggedright \_\_class\_\_}\\
\cline{1-2}
\end{longtable}


%%%%%%%%%%%%%%%%%%%%%%%%%%%%%%%%%%%%%%%%%%%%%%%%%%%%%%%%%%%%%%%%%%%%%%%%%%%
%%                            Class Variables                            %%
%%%%%%%%%%%%%%%%%%%%%%%%%%%%%%%%%%%%%%%%%%%%%%%%%%%%%%%%%%%%%%%%%%%%%%%%%%%

  \subsubsection{Class Variables}

    \vspace{-1cm}
\hspace{\varindent}\begin{longtable}{|p{\varnamewidth}|p{\vardescrwidth}|l}
\cline{1-2}
\cline{1-2} \centering \textbf{Name} & \centering \textbf{Description}& \\
\cline{1-2}
\endhead\cline{1-2}\multicolumn{3}{r}{\small\textit{continued on next page}}\\\endfoot\cline{1-2}
\endlastfoot\multicolumn{2}{|l|}{\textit{Inherited from random.Random \textit{(Section \ref{random:Random})}}}\\
\multicolumn{2}{|p{\varwidth}|}{\raggedright VERSION}\\
\cline{1-2}
\end{longtable}

    \index{random.SystemRandom \textit{(class)}|)}

%%%%%%%%%%%%%%%%%%%%%%%%%%%%%%%%%%%%%%%%%%%%%%%%%%%%%%%%%%%%%%%%%%%%%%%%%%%
%%                           Class Description                           %%
%%%%%%%%%%%%%%%%%%%%%%%%%%%%%%%%%%%%%%%%%%%%%%%%%%%%%%%%%%%%%%%%%%%%%%%%%%%

    \index{random.WichmannHill \textit{(class)}|(}
\subsection{Class WichmannHill}

    \label{random:WichmannHill}
\begin{tabular}{cccccccccc}
% Line for object, linespec=[False, False, False]
\multicolumn{2}{r}{\settowidth{\BCL}{object}\multirow{2}{\BCL}{object}}
&&
&&
&&
  \\\cline{3-3}
  &&\multicolumn{1}{c|}{}
&&
&&
&&
  \\
% Line for \_random.Random, linespec=[False, False]
\multicolumn{4}{r}{\settowidth{\BCL}{\_random.Random}\multirow{2}{\BCL}{\_random.Random}}
&&
&&
  \\\cline{5-5}
  &&&&\multicolumn{1}{c|}{}
&&
&&
  \\
% Line for random.Random, linespec=[False]
\multicolumn{6}{r}{\settowidth{\BCL}{random.Random}\multirow{2}{\BCL}{random.Random}}
&&
  \\\cline{7-7}
  &&&&&&\multicolumn{1}{c|}{}
&&
  \\
&&&&&&\multicolumn{2}{l}{\textbf{random.WichmannHill}}
\end{tabular}


%%%%%%%%%%%%%%%%%%%%%%%%%%%%%%%%%%%%%%%%%%%%%%%%%%%%%%%%%%%%%%%%%%%%%%%%%%%
%%                                Methods                                %%
%%%%%%%%%%%%%%%%%%%%%%%%%%%%%%%%%%%%%%%%%%%%%%%%%%%%%%%%%%%%%%%%%%%%%%%%%%%

  \subsubsection{Methods}

    \vspace{0.5ex}

\hspace{.8\funcindent}\begin{boxedminipage}{\funcwidth}

    \raggedright \textbf{getstate}(\textit{self})

    \vspace{-1.5ex}

    \rule{\textwidth}{0.5\fboxrule}
\setlength{\parskip}{2ex}
    Return internal state; can be passed to setstate() later.

\setlength{\parskip}{1ex}
      \textbf{Return Value}
    \vspace{-1ex}

      \begin{quote}
      tuple containing the current state.

      \end{quote}

      Overrides: \_random.Random.getstate

    \end{boxedminipage}

    \vspace{0.5ex}

\hspace{.8\funcindent}\begin{boxedminipage}{\funcwidth}

    \raggedright \textbf{jumpahead}(\textit{self}, \textit{n})

    \vspace{-1.5ex}

    \rule{\textwidth}{0.5\fboxrule}
\setlength{\parskip}{2ex}
\begin{alltt}
Act as if n calls to random() were made, but quickly.

n is an int, greater than or equal to 0.

Example use:  If you have 2 threads and know that each will
consume no more than a million random numbers, create two Random
objects r1 and r2, then do
    r2.setstate(r1.getstate())
    r2.jumpahead(1000000)
Then r1 and r2 will use guaranteed-disjoint segments of the full
period.
\end{alltt}

\setlength{\parskip}{1ex}
      \textbf{Return Value}
    \vspace{-1ex}

      \begin{quote}
      None

      \end{quote}

      Overrides: \_random.Random.jumpahead

    \end{boxedminipage}

    \vspace{0.5ex}

\hspace{.8\funcindent}\begin{boxedminipage}{\funcwidth}

    \raggedright \textbf{random}(\textit{self})

    \vspace{-1.5ex}

    \rule{\textwidth}{0.5\fboxrule}
\setlength{\parskip}{2ex}
    Get the next random number in the range [0.0, 1.0).

\setlength{\parskip}{1ex}
      \textbf{Return Value}
    \vspace{-1ex}

      \begin{quote}
      x in the interval [0, 1).

      \end{quote}

      Overrides: \_random.Random.random

    \end{boxedminipage}

    \vspace{0.5ex}

\hspace{.8\funcindent}\begin{boxedminipage}{\funcwidth}

    \raggedright \textbf{seed}(\textit{self}, \textit{a}={\tt None})

    \vspace{-1.5ex}

    \rule{\textwidth}{0.5\fboxrule}
\setlength{\parskip}{2ex}
    Initialize internal state from hashable object.

    None or no argument seeds from current time or from an operating system
    specific randomness source if available.

    If a is not None or an int or long, hash(a) is used instead.

    If a is an int or long, a is used directly.  Distinct values between 0 
    and 27814431486575L inclusive are guaranteed to yield distinct internal
    states (this guarantee is specific to the default Wichmann-Hill 
    generator).

\setlength{\parskip}{1ex}
      \textbf{Return Value}
    \vspace{-1ex}

      \begin{quote}
      None

      \end{quote}

      Overrides: \_random.Random.seed

    \end{boxedminipage}

    \vspace{0.5ex}

\hspace{.8\funcindent}\begin{boxedminipage}{\funcwidth}

    \raggedright \textbf{setstate}(\textit{self}, \textit{state})

    \vspace{-1.5ex}

    \rule{\textwidth}{0.5\fboxrule}
\setlength{\parskip}{2ex}
    Restore internal state from object returned by getstate().

\setlength{\parskip}{1ex}
      \textbf{Return Value}
    \vspace{-1ex}

      \begin{quote}
      None

      \end{quote}

      Overrides: \_random.Random.setstate

    \end{boxedminipage}

    \label{random:WichmannHill:whseed}
    \index{random.WichmannHill.whseed \textit{(function)}}

    \vspace{0.5ex}

\hspace{.8\funcindent}\begin{boxedminipage}{\funcwidth}

    \raggedright \textbf{whseed}(\textit{self}, \textit{a}={\tt None})

    \vspace{-1.5ex}

    \rule{\textwidth}{0.5\fboxrule}
\setlength{\parskip}{2ex}
    Seed from hashable object's hash code.

    None or no argument seeds from current time.  It is not guaranteed that
    objects with distinct hash codes lead to distinct internal states.

    This is obsolete, provided for compatibility with the seed routine used
    prior to Python 2.1.  Use the .seed() method instead.

\setlength{\parskip}{1ex}
    \end{boxedminipage}


\large{\textbf{\textit{Inherited from random.Random\textit{(Section \ref{random:Random})}}}}

\begin{quote}
\_\_getstate\_\_(), \_\_init\_\_(), \_\_reduce\_\_(), \_\_setstate\_\_(), betavariate(), choice(), expovariate(), gammavariate(), gauss(), lognormvariate(), normalvariate(), paretovariate(), randint(), randrange(), sample(), shuffle(), triangular(), uniform(), vonmisesvariate(), weibullvariate()
\end{quote}

\large{\textbf{\textit{Inherited from \_random.Random}}}

\begin{quote}
\_\_getattribute\_\_(), \_\_new\_\_(), getrandbits()
\end{quote}

\large{\textbf{\textit{Inherited from object}}}

\begin{quote}
\_\_delattr\_\_(), \_\_format\_\_(), \_\_hash\_\_(), \_\_reduce\_ex\_\_(), \_\_repr\_\_(), \_\_setattr\_\_(), \_\_sizeof\_\_(), \_\_str\_\_(), \_\_subclasshook\_\_()
\end{quote}

%%%%%%%%%%%%%%%%%%%%%%%%%%%%%%%%%%%%%%%%%%%%%%%%%%%%%%%%%%%%%%%%%%%%%%%%%%%
%%                              Properties                               %%
%%%%%%%%%%%%%%%%%%%%%%%%%%%%%%%%%%%%%%%%%%%%%%%%%%%%%%%%%%%%%%%%%%%%%%%%%%%

  \subsubsection{Properties}

    \vspace{-1cm}
\hspace{\varindent}\begin{longtable}{|p{\varnamewidth}|p{\vardescrwidth}|l}
\cline{1-2}
\cline{1-2} \centering \textbf{Name} & \centering \textbf{Description}& \\
\cline{1-2}
\endhead\cline{1-2}\multicolumn{3}{r}{\small\textit{continued on next page}}\\\endfoot\cline{1-2}
\endlastfoot\multicolumn{2}{|l|}{\textit{Inherited from object}}\\
\multicolumn{2}{|p{\varwidth}|}{\raggedright \_\_class\_\_}\\
\cline{1-2}
\end{longtable}


%%%%%%%%%%%%%%%%%%%%%%%%%%%%%%%%%%%%%%%%%%%%%%%%%%%%%%%%%%%%%%%%%%%%%%%%%%%
%%                            Class Variables                            %%
%%%%%%%%%%%%%%%%%%%%%%%%%%%%%%%%%%%%%%%%%%%%%%%%%%%%%%%%%%%%%%%%%%%%%%%%%%%

  \subsubsection{Class Variables}

    \vspace{-1cm}
\hspace{\varindent}\begin{longtable}{|p{\varnamewidth}|p{\vardescrwidth}|l}
\cline{1-2}
\cline{1-2} \centering \textbf{Name} & \centering \textbf{Description}& \\
\cline{1-2}
\endhead\cline{1-2}\multicolumn{3}{r}{\small\textit{continued on next page}}\\\endfoot\cline{1-2}
\endlastfoot\raggedright V\-E\-R\-S\-I\-O\-N\- & \raggedright \textbf{Value:} 
{\tt 1}&\\
\cline{1-2}
\end{longtable}

    \index{random.WichmannHill \textit{(class)}|)}
    \index{random \textit{(module)}|)}
