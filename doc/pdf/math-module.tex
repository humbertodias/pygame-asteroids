%
% API Documentation for API Documentation
% Module math
%
% Generated by epydoc 3.0.1
% [Thu Dec 10 09:57:57 2015]
%

%%%%%%%%%%%%%%%%%%%%%%%%%%%%%%%%%%%%%%%%%%%%%%%%%%%%%%%%%%%%%%%%%%%%%%%%%%%
%%                          Module Description                           %%
%%%%%%%%%%%%%%%%%%%%%%%%%%%%%%%%%%%%%%%%%%%%%%%%%%%%%%%%%%%%%%%%%%%%%%%%%%%

    \index{math \textit{(module)}|(}
\section{Module math}

    \label{math}
This module is always available.  It provides access to the mathematical 
functions defined by the C standard.


%%%%%%%%%%%%%%%%%%%%%%%%%%%%%%%%%%%%%%%%%%%%%%%%%%%%%%%%%%%%%%%%%%%%%%%%%%%
%%                               Functions                               %%
%%%%%%%%%%%%%%%%%%%%%%%%%%%%%%%%%%%%%%%%%%%%%%%%%%%%%%%%%%%%%%%%%%%%%%%%%%%

  \subsection{Functions}

    \label{math:acos}
    \index{math.acos \textit{(function)}}

    \vspace{0.5ex}

\hspace{.8\funcindent}\begin{boxedminipage}{\funcwidth}

    \raggedright \textbf{acos}(\textit{x})

    \vspace{-1.5ex}

    \rule{\textwidth}{0.5\fboxrule}
\setlength{\parskip}{2ex}
    Return the arc cosine (measured in radians) of x.

\setlength{\parskip}{1ex}
    \end{boxedminipage}

    \label{math:acosh}
    \index{math.acosh \textit{(function)}}

    \vspace{0.5ex}

\hspace{.8\funcindent}\begin{boxedminipage}{\funcwidth}

    \raggedright \textbf{acosh}(\textit{x})

    \vspace{-1.5ex}

    \rule{\textwidth}{0.5\fboxrule}
\setlength{\parskip}{2ex}
    Return the hyperbolic arc cosine (measured in radians) of x.

\setlength{\parskip}{1ex}
    \end{boxedminipage}

    \label{math:asin}
    \index{math.asin \textit{(function)}}

    \vspace{0.5ex}

\hspace{.8\funcindent}\begin{boxedminipage}{\funcwidth}

    \raggedright \textbf{asin}(\textit{x})

    \vspace{-1.5ex}

    \rule{\textwidth}{0.5\fboxrule}
\setlength{\parskip}{2ex}
    Return the arc sine (measured in radians) of x.

\setlength{\parskip}{1ex}
    \end{boxedminipage}

    \label{math:asinh}
    \index{math.asinh \textit{(function)}}

    \vspace{0.5ex}

\hspace{.8\funcindent}\begin{boxedminipage}{\funcwidth}

    \raggedright \textbf{asinh}(\textit{x})

    \vspace{-1.5ex}

    \rule{\textwidth}{0.5\fboxrule}
\setlength{\parskip}{2ex}
    Return the hyperbolic arc sine (measured in radians) of x.

\setlength{\parskip}{1ex}
    \end{boxedminipage}

    \label{math:atan}
    \index{math.atan \textit{(function)}}

    \vspace{0.5ex}

\hspace{.8\funcindent}\begin{boxedminipage}{\funcwidth}

    \raggedright \textbf{atan}(\textit{x})

    \vspace{-1.5ex}

    \rule{\textwidth}{0.5\fboxrule}
\setlength{\parskip}{2ex}
    Return the arc tangent (measured in radians) of x.

\setlength{\parskip}{1ex}
    \end{boxedminipage}

    \label{math:atan2}
    \index{math.atan2 \textit{(function)}}

    \vspace{0.5ex}

\hspace{.8\funcindent}\begin{boxedminipage}{\funcwidth}

    \raggedright \textbf{atan2}(\textit{y}, \textit{x})

    \vspace{-1.5ex}

    \rule{\textwidth}{0.5\fboxrule}
\setlength{\parskip}{2ex}
    Return the arc tangent (measured in radians) of y/x. Unlike atan(y/x), 
    the signs of both x and y are considered.

\setlength{\parskip}{1ex}
    \end{boxedminipage}

    \label{math:atanh}
    \index{math.atanh \textit{(function)}}

    \vspace{0.5ex}

\hspace{.8\funcindent}\begin{boxedminipage}{\funcwidth}

    \raggedright \textbf{atanh}(\textit{x})

    \vspace{-1.5ex}

    \rule{\textwidth}{0.5\fboxrule}
\setlength{\parskip}{2ex}
    Return the hyperbolic arc tangent (measured in radians) of x.

\setlength{\parskip}{1ex}
    \end{boxedminipage}

    \label{math:ceil}
    \index{math.ceil \textit{(function)}}

    \vspace{0.5ex}

\hspace{.8\funcindent}\begin{boxedminipage}{\funcwidth}

    \raggedright \textbf{ceil}(\textit{x})

    \vspace{-1.5ex}

    \rule{\textwidth}{0.5\fboxrule}
\setlength{\parskip}{2ex}
    Return the ceiling of x as a float. This is the smallest integral value
    {\textgreater}= x.

\setlength{\parskip}{1ex}
    \end{boxedminipage}

    \label{math:copysign}
    \index{math.copysign \textit{(function)}}

    \vspace{0.5ex}

\hspace{.8\funcindent}\begin{boxedminipage}{\funcwidth}

    \raggedright \textbf{copysign}(\textit{x}, \textit{y})

    \vspace{-1.5ex}

    \rule{\textwidth}{0.5\fboxrule}
\setlength{\parskip}{2ex}
    Return x with the sign of y.

\setlength{\parskip}{1ex}
    \end{boxedminipage}

    \label{math:cos}
    \index{math.cos \textit{(function)}}

    \vspace{0.5ex}

\hspace{.8\funcindent}\begin{boxedminipage}{\funcwidth}

    \raggedright \textbf{cos}(\textit{x})

    \vspace{-1.5ex}

    \rule{\textwidth}{0.5\fboxrule}
\setlength{\parskip}{2ex}
    Return the cosine of x (measured in radians).

\setlength{\parskip}{1ex}
    \end{boxedminipage}

    \label{math:cosh}
    \index{math.cosh \textit{(function)}}

    \vspace{0.5ex}

\hspace{.8\funcindent}\begin{boxedminipage}{\funcwidth}

    \raggedright \textbf{cosh}(\textit{x})

    \vspace{-1.5ex}

    \rule{\textwidth}{0.5\fboxrule}
\setlength{\parskip}{2ex}
    Return the hyperbolic cosine of x.

\setlength{\parskip}{1ex}
    \end{boxedminipage}

    \label{math:degrees}
    \index{math.degrees \textit{(function)}}

    \vspace{0.5ex}

\hspace{.8\funcindent}\begin{boxedminipage}{\funcwidth}

    \raggedright \textbf{degrees}(\textit{x})

    \vspace{-1.5ex}

    \rule{\textwidth}{0.5\fboxrule}
\setlength{\parskip}{2ex}
    Convert angle x from radians to degrees.

\setlength{\parskip}{1ex}
    \end{boxedminipage}

    \label{math:erf}
    \index{math.erf \textit{(function)}}

    \vspace{0.5ex}

\hspace{.8\funcindent}\begin{boxedminipage}{\funcwidth}

    \raggedright \textbf{erf}(\textit{x})

    \vspace{-1.5ex}

    \rule{\textwidth}{0.5\fboxrule}
\setlength{\parskip}{2ex}
    Error function at x.

\setlength{\parskip}{1ex}
    \end{boxedminipage}

    \label{math:erfc}
    \index{math.erfc \textit{(function)}}

    \vspace{0.5ex}

\hspace{.8\funcindent}\begin{boxedminipage}{\funcwidth}

    \raggedright \textbf{erfc}(\textit{x})

    \vspace{-1.5ex}

    \rule{\textwidth}{0.5\fboxrule}
\setlength{\parskip}{2ex}
    Complementary error function at x.

\setlength{\parskip}{1ex}
    \end{boxedminipage}

    \label{math:exp}
    \index{math.exp \textit{(function)}}

    \vspace{0.5ex}

\hspace{.8\funcindent}\begin{boxedminipage}{\funcwidth}

    \raggedright \textbf{exp}(\textit{x})

    \vspace{-1.5ex}

    \rule{\textwidth}{0.5\fboxrule}
\setlength{\parskip}{2ex}
    Return e raised to the power of x.

\setlength{\parskip}{1ex}
    \end{boxedminipage}

    \label{math:expm1}
    \index{math.expm1 \textit{(function)}}

    \vspace{0.5ex}

\hspace{.8\funcindent}\begin{boxedminipage}{\funcwidth}

    \raggedright \textbf{expm1}(\textit{x})

    \vspace{-1.5ex}

    \rule{\textwidth}{0.5\fboxrule}
\setlength{\parskip}{2ex}
    Return exp(x)-1. This function avoids the loss of precision involved in
    the direct evaluation of exp(x)-1 for small x.

\setlength{\parskip}{1ex}
    \end{boxedminipage}

    \label{math:fabs}
    \index{math.fabs \textit{(function)}}

    \vspace{0.5ex}

\hspace{.8\funcindent}\begin{boxedminipage}{\funcwidth}

    \raggedright \textbf{fabs}(\textit{x})

    \vspace{-1.5ex}

    \rule{\textwidth}{0.5\fboxrule}
\setlength{\parskip}{2ex}
    Return the absolute value of the float x.

\setlength{\parskip}{1ex}
    \end{boxedminipage}

    \label{math:factorial}
    \index{math.factorial \textit{(function)}}

    \vspace{0.5ex}

\hspace{.8\funcindent}\begin{boxedminipage}{\funcwidth}

    \raggedright \textbf{factorial}(\textit{x})

    \vspace{-1.5ex}

    \rule{\textwidth}{0.5\fboxrule}
\setlength{\parskip}{2ex}
    Find x!. Raise a ValueError if x is negative or non-integral.

\setlength{\parskip}{1ex}
      \textbf{Return Value}
    \vspace{-1ex}

      \begin{quote}
      Integral

      \end{quote}

    \end{boxedminipage}

    \label{math:floor}
    \index{math.floor \textit{(function)}}

    \vspace{0.5ex}

\hspace{.8\funcindent}\begin{boxedminipage}{\funcwidth}

    \raggedright \textbf{floor}(\textit{x})

    \vspace{-1.5ex}

    \rule{\textwidth}{0.5\fboxrule}
\setlength{\parskip}{2ex}
    Return the floor of x as a float. This is the largest integral value 
    {\textless}= x.

\setlength{\parskip}{1ex}
    \end{boxedminipage}

    \label{math:fmod}
    \index{math.fmod \textit{(function)}}

    \vspace{0.5ex}

\hspace{.8\funcindent}\begin{boxedminipage}{\funcwidth}

    \raggedright \textbf{fmod}(\textit{x}, \textit{y})

    \vspace{-1.5ex}

    \rule{\textwidth}{0.5\fboxrule}
\setlength{\parskip}{2ex}
    Return fmod(x, y), according to platform C.  x \% y may differ.

\setlength{\parskip}{1ex}
    \end{boxedminipage}

    \label{math:frexp}
    \index{math.frexp \textit{(function)}}

    \vspace{0.5ex}

\hspace{.8\funcindent}\begin{boxedminipage}{\funcwidth}

    \raggedright \textbf{frexp}(\textit{x})

    \vspace{-1.5ex}

    \rule{\textwidth}{0.5\fboxrule}
\setlength{\parskip}{2ex}
    Return the mantissa and exponent of x, as pair (m, e). m is a float and
    e is an int, such that x = m * 2.**e. If x is 0, m and e are both 0.  
    Else 0.5 {\textless}= abs(m) {\textless} 1.0.

\setlength{\parskip}{1ex}
    \end{boxedminipage}

    \label{math:fsum}
    \index{math.fsum \textit{(function)}}

    \vspace{0.5ex}

\hspace{.8\funcindent}\begin{boxedminipage}{\funcwidth}

    \raggedright \textbf{fsum}(\textit{iterable})

    \vspace{-1.5ex}

    \rule{\textwidth}{0.5\fboxrule}
\setlength{\parskip}{2ex}
    Return an accurate floating point sum of values in the iterable. 
    Assumes IEEE-754 floating point arithmetic.

\setlength{\parskip}{1ex}
    \end{boxedminipage}

    \label{math:gamma}
    \index{math.gamma \textit{(function)}}

    \vspace{0.5ex}

\hspace{.8\funcindent}\begin{boxedminipage}{\funcwidth}

    \raggedright \textbf{gamma}(\textit{x})

    \vspace{-1.5ex}

    \rule{\textwidth}{0.5\fboxrule}
\setlength{\parskip}{2ex}
    Gamma function at x.

\setlength{\parskip}{1ex}
    \end{boxedminipage}

    \label{math:hypot}
    \index{math.hypot \textit{(function)}}

    \vspace{0.5ex}

\hspace{.8\funcindent}\begin{boxedminipage}{\funcwidth}

    \raggedright \textbf{hypot}(\textit{x}, \textit{y})

    \vspace{-1.5ex}

    \rule{\textwidth}{0.5\fboxrule}
\setlength{\parskip}{2ex}
    Return the Euclidean distance, sqrt(x*x + y*y).

\setlength{\parskip}{1ex}
    \end{boxedminipage}

    \label{math:isinf}
    \index{math.isinf \textit{(function)}}

    \vspace{0.5ex}

\hspace{.8\funcindent}\begin{boxedminipage}{\funcwidth}

    \raggedright \textbf{isinf}(\textit{x})

    \vspace{-1.5ex}

    \rule{\textwidth}{0.5\fboxrule}
\setlength{\parskip}{2ex}
    Check if float x is infinite (positive or negative).

\setlength{\parskip}{1ex}
      \textbf{Return Value}
    \vspace{-1ex}

      \begin{quote}
      bool

      \end{quote}

    \end{boxedminipage}

    \label{math:isnan}
    \index{math.isnan \textit{(function)}}

    \vspace{0.5ex}

\hspace{.8\funcindent}\begin{boxedminipage}{\funcwidth}

    \raggedright \textbf{isnan}(\textit{x})

    \vspace{-1.5ex}

    \rule{\textwidth}{0.5\fboxrule}
\setlength{\parskip}{2ex}
    Check if float x is not a number (NaN).

\setlength{\parskip}{1ex}
      \textbf{Return Value}
    \vspace{-1ex}

      \begin{quote}
      bool

      \end{quote}

    \end{boxedminipage}

    \label{math:ldexp}
    \index{math.ldexp \textit{(function)}}

    \vspace{0.5ex}

\hspace{.8\funcindent}\begin{boxedminipage}{\funcwidth}

    \raggedright \textbf{ldexp}(\textit{x}, \textit{i})

    \vspace{-1.5ex}

    \rule{\textwidth}{0.5\fboxrule}
\setlength{\parskip}{2ex}
    Return x * (2**i).

\setlength{\parskip}{1ex}
    \end{boxedminipage}

    \label{math:lgamma}
    \index{math.lgamma \textit{(function)}}

    \vspace{0.5ex}

\hspace{.8\funcindent}\begin{boxedminipage}{\funcwidth}

    \raggedright \textbf{lgamma}(\textit{x})

    \vspace{-1.5ex}

    \rule{\textwidth}{0.5\fboxrule}
\setlength{\parskip}{2ex}
    Natural logarithm of absolute value of Gamma function at x.

\setlength{\parskip}{1ex}
    \end{boxedminipage}

    \label{math:log}
    \index{math.log \textit{(function)}}

    \vspace{0.5ex}

\hspace{.8\funcindent}\begin{boxedminipage}{\funcwidth}

    \raggedright \textbf{log}(\textit{x}, \textit{base}={\tt ...})

    \vspace{-1.5ex}

    \rule{\textwidth}{0.5\fboxrule}
\setlength{\parskip}{2ex}
    Return the logarithm of x to the given base. If the base not specified,
    returns the natural logarithm (base e) of x.

\setlength{\parskip}{1ex}
    \end{boxedminipage}

    \label{math:log10}
    \index{math.log10 \textit{(function)}}

    \vspace{0.5ex}

\hspace{.8\funcindent}\begin{boxedminipage}{\funcwidth}

    \raggedright \textbf{log10}(\textit{x})

    \vspace{-1.5ex}

    \rule{\textwidth}{0.5\fboxrule}
\setlength{\parskip}{2ex}
    Return the base 10 logarithm of x.

\setlength{\parskip}{1ex}
    \end{boxedminipage}

    \label{math:log1p}
    \index{math.log1p \textit{(function)}}

    \vspace{0.5ex}

\hspace{.8\funcindent}\begin{boxedminipage}{\funcwidth}

    \raggedright \textbf{log1p}(\textit{x})

    \vspace{-1.5ex}

    \rule{\textwidth}{0.5\fboxrule}
\setlength{\parskip}{2ex}
    Return the natural logarithm of 1+x (base e). The result is computed in
    a way which is accurate for x near zero.

\setlength{\parskip}{1ex}
    \end{boxedminipage}

    \label{math:modf}
    \index{math.modf \textit{(function)}}

    \vspace{0.5ex}

\hspace{.8\funcindent}\begin{boxedminipage}{\funcwidth}

    \raggedright \textbf{modf}(\textit{x})

    \vspace{-1.5ex}

    \rule{\textwidth}{0.5\fboxrule}
\setlength{\parskip}{2ex}
    Return the fractional and integer parts of x.  Both results carry the 
    sign of x and are floats.

\setlength{\parskip}{1ex}
    \end{boxedminipage}

    \label{math:pow}
    \index{math.pow \textit{(function)}}

    \vspace{0.5ex}

\hspace{.8\funcindent}\begin{boxedminipage}{\funcwidth}

    \raggedright \textbf{pow}(\textit{x}, \textit{y})

    \vspace{-1.5ex}

    \rule{\textwidth}{0.5\fboxrule}
\setlength{\parskip}{2ex}
    Return x**y (x to the power of y).

\setlength{\parskip}{1ex}
    \end{boxedminipage}

    \label{math:radians}
    \index{math.radians \textit{(function)}}

    \vspace{0.5ex}

\hspace{.8\funcindent}\begin{boxedminipage}{\funcwidth}

    \raggedright \textbf{radians}(\textit{x})

    \vspace{-1.5ex}

    \rule{\textwidth}{0.5\fboxrule}
\setlength{\parskip}{2ex}
    Convert angle x from degrees to radians.

\setlength{\parskip}{1ex}
    \end{boxedminipage}

    \label{math:sin}
    \index{math.sin \textit{(function)}}

    \vspace{0.5ex}

\hspace{.8\funcindent}\begin{boxedminipage}{\funcwidth}

    \raggedright \textbf{sin}(\textit{x})

    \vspace{-1.5ex}

    \rule{\textwidth}{0.5\fboxrule}
\setlength{\parskip}{2ex}
    Return the sine of x (measured in radians).

\setlength{\parskip}{1ex}
    \end{boxedminipage}

    \label{math:sinh}
    \index{math.sinh \textit{(function)}}

    \vspace{0.5ex}

\hspace{.8\funcindent}\begin{boxedminipage}{\funcwidth}

    \raggedright \textbf{sinh}(\textit{x})

    \vspace{-1.5ex}

    \rule{\textwidth}{0.5\fboxrule}
\setlength{\parskip}{2ex}
    Return the hyperbolic sine of x.

\setlength{\parskip}{1ex}
    \end{boxedminipage}

    \label{math:sqrt}
    \index{math.sqrt \textit{(function)}}

    \vspace{0.5ex}

\hspace{.8\funcindent}\begin{boxedminipage}{\funcwidth}

    \raggedright \textbf{sqrt}(\textit{x})

    \vspace{-1.5ex}

    \rule{\textwidth}{0.5\fboxrule}
\setlength{\parskip}{2ex}
    Return the square root of x.

\setlength{\parskip}{1ex}
    \end{boxedminipage}

    \label{math:tan}
    \index{math.tan \textit{(function)}}

    \vspace{0.5ex}

\hspace{.8\funcindent}\begin{boxedminipage}{\funcwidth}

    \raggedright \textbf{tan}(\textit{x})

    \vspace{-1.5ex}

    \rule{\textwidth}{0.5\fboxrule}
\setlength{\parskip}{2ex}
    Return the tangent of x (measured in radians).

\setlength{\parskip}{1ex}
    \end{boxedminipage}

    \label{math:tanh}
    \index{math.tanh \textit{(function)}}

    \vspace{0.5ex}

\hspace{.8\funcindent}\begin{boxedminipage}{\funcwidth}

    \raggedright \textbf{tanh}(\textit{x})

    \vspace{-1.5ex}

    \rule{\textwidth}{0.5\fboxrule}
\setlength{\parskip}{2ex}
    Return the hyperbolic tangent of x.

\setlength{\parskip}{1ex}
    \end{boxedminipage}

    \label{math:trunc}
    \index{math.trunc \textit{(function)}}

    \vspace{0.5ex}

\hspace{.8\funcindent}\begin{boxedminipage}{\funcwidth}

    \raggedright \textbf{trunc}(\textit{...})

    \vspace{-1.5ex}

    \rule{\textwidth}{0.5\fboxrule}
\setlength{\parskip}{2ex}
    trunc(x:Real) -{\textgreater} Integral

    Truncates x to the nearest Integral toward 0. Uses the \_\_trunc\_\_ 
    magic method.

\setlength{\parskip}{1ex}
    \end{boxedminipage}


%%%%%%%%%%%%%%%%%%%%%%%%%%%%%%%%%%%%%%%%%%%%%%%%%%%%%%%%%%%%%%%%%%%%%%%%%%%
%%                               Variables                               %%
%%%%%%%%%%%%%%%%%%%%%%%%%%%%%%%%%%%%%%%%%%%%%%%%%%%%%%%%%%%%%%%%%%%%%%%%%%%

  \subsection{Variables}

    \vspace{-1cm}
\hspace{\varindent}\begin{longtable}{|p{\varnamewidth}|p{\vardescrwidth}|l}
\cline{1-2}
\cline{1-2} \centering \textbf{Name} & \centering \textbf{Description}& \\
\cline{1-2}
\endhead\cline{1-2}\multicolumn{3}{r}{\small\textit{continued on next page}}\\\endfoot\cline{1-2}
\endlastfoot\raggedright \_\-\_\-p\-a\-c\-k\-a\-g\-e\-\_\-\_\- & \raggedright \textbf{Value:} 
{\tt None}&\\
\cline{1-2}
\raggedright e\- & \raggedright \textbf{Value:} 
{\tt 2.71828182846}&\\
\cline{1-2}
\raggedright p\-i\- & \raggedright \textbf{Value:} 
{\tt 3.14159265359}&\\
\cline{1-2}
\end{longtable}

    \index{math \textit{(module)}|)}
