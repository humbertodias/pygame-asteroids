%
% API Documentation for API Documentation
% Module sys
%
% Generated by epydoc 3.0.1
% [Thu Dec 10 09:57:58 2015]
%

%%%%%%%%%%%%%%%%%%%%%%%%%%%%%%%%%%%%%%%%%%%%%%%%%%%%%%%%%%%%%%%%%%%%%%%%%%%
%%                          Module Description                           %%
%%%%%%%%%%%%%%%%%%%%%%%%%%%%%%%%%%%%%%%%%%%%%%%%%%%%%%%%%%%%%%%%%%%%%%%%%%%

    \index{sys \textit{(module)}|(}
\section{Module sys}

    \label{sys}
\begin{alltt}
This module provides access to some objects used or maintained by the
interpreter and to functions that interact strongly with the interpreter.

Dynamic objects:

argv -- command line arguments; argv[0] is the script pathname if known
path -- module search path; path[0] is the script directory, else ''
modules -- dictionary of loaded modules

displayhook -- called to show results in an interactive session
excepthook -- called to handle any uncaught exception other than SystemExit
  To customize printing in an interactive session or to install a custom
  top-level exception handler, assign other functions to replace these.

exitfunc -- if sys.exitfunc exists, this routine is called when Python exits
  Assigning to sys.exitfunc is deprecated; use the atexit module instead.

stdin -- standard input file object; used by raw\_input() and input()
stdout -- standard output file object; used by the print statement
stderr -- standard error object; used for error messages
  By assigning other file objects (or objects that behave like files)
  to these, it is possible to redirect all of the interpreter's I/O.

last\_type -- type of last uncaught exception
last\_value -- value of last uncaught exception
last\_traceback -- traceback of last uncaught exception
  These three are only available in an interactive session after a
  traceback has been printed.

exc\_type -- type of exception currently being handled
exc\_value -- value of exception currently being handled
exc\_traceback -- traceback of exception currently being handled
  The function exc\_info() should be used instead of these three,
  because it is thread-safe.

Static objects:

float\_info -- a dict with information about the float inplementation.
long\_info -- a struct sequence with information about the long implementation.
maxint -- the largest supported integer (the smallest is -maxint-1)
maxsize -- the largest supported length of containers.
maxunicode -- the largest supported character
builtin\_module\_names -- tuple of module names built into this interpreter
version -- the version of this interpreter as a string
version\_info -- version information as a named tuple
hexversion -- version information encoded as a single integer
copyright -- copyright notice pertaining to this interpreter
platform -- platform identifier
executable -- absolute path of the executable binary of the Python interpreter
prefix -- prefix used to find the Python library
exec\_prefix -- prefix used to find the machine-specific Python library
float\_repr\_style -- string indicating the style of repr() output for floats
\_\_stdin\_\_ -- the original stdin; don't touch!
\_\_stdout\_\_ -- the original stdout; don't touch!
\_\_stderr\_\_ -- the original stderr; don't touch!
\_\_displayhook\_\_ -- the original displayhook; don't touch!
\_\_excepthook\_\_ -- the original excepthook; don't touch!

Functions:

displayhook() -- print an object to the screen, and save it in \_\_builtin\_\_.\_
excepthook() -- print an exception and its traceback to sys.stderr
exc\_info() -- return thread-safe information about the current exception
exc\_clear() -- clear the exception state for the current thread
exit() -- exit the interpreter by raising SystemExit
getdlopenflags() -- returns flags to be used for dlopen() calls
getprofile() -- get the global profiling function
getrefcount() -- return the reference count for an object (plus one :-)
getrecursionlimit() -- return the max recursion depth for the interpreter
getsizeof() -- return the size of an object in bytes
gettrace() -- get the global debug tracing function
setcheckinterval() -- control how often the interpreter checks for events
setdlopenflags() -- set the flags to be used for dlopen() calls
setprofile() -- set the global profiling function
setrecursionlimit() -- set the max recursion depth for the interpreter
settrace() -- set the global debug tracing function
\end{alltt}


%%%%%%%%%%%%%%%%%%%%%%%%%%%%%%%%%%%%%%%%%%%%%%%%%%%%%%%%%%%%%%%%%%%%%%%%%%%
%%                               Functions                               %%
%%%%%%%%%%%%%%%%%%%%%%%%%%%%%%%%%%%%%%%%%%%%%%%%%%%%%%%%%%%%%%%%%%%%%%%%%%%

  \subsection{Functions}

    \label{sys:displayhook}
    \index{sys.displayhook \textit{(function)}}

    \vspace{0.5ex}

\hspace{.8\funcindent}\begin{boxedminipage}{\funcwidth}

    \raggedright \textbf{\_\_displayhook\_\_}(\textit{object})

    \vspace{-1.5ex}

    \rule{\textwidth}{0.5\fboxrule}
\setlength{\parskip}{2ex}
    Print an object to sys.stdout and also save it in \_\_builtin\_\_.\_

\setlength{\parskip}{1ex}
      \textbf{Return Value}
    \vspace{-1ex}

      \begin{quote}
      None

      \end{quote}

    \end{boxedminipage}

    \label{sys:excepthook}
    \index{sys.excepthook \textit{(function)}}

    \vspace{0.5ex}

\hspace{.8\funcindent}\begin{boxedminipage}{\funcwidth}

    \raggedright \textbf{\_\_excepthook\_\_}(\textit{exctype}, \textit{value}, \textit{traceback})

    \vspace{-1.5ex}

    \rule{\textwidth}{0.5\fboxrule}
\setlength{\parskip}{2ex}
    Handle an exception by displaying it with a traceback on sys.stderr.

\setlength{\parskip}{1ex}
      \textbf{Return Value}
    \vspace{-1ex}

      \begin{quote}
      None

      \end{quote}

    \end{boxedminipage}

    \label{sys:call_tracing}
    \index{sys.call\_tracing \textit{(function)}}

    \vspace{0.5ex}

\hspace{.8\funcindent}\begin{boxedminipage}{\funcwidth}

    \raggedright \textbf{call\_tracing}(\textit{func}, \textit{args})

    \vspace{-1.5ex}

    \rule{\textwidth}{0.5\fboxrule}
\setlength{\parskip}{2ex}
    Call func(*args), while tracing is enabled.  The tracing state is 
    saved, and restored afterwards.  This is intended to be called from a 
    debugger from a checkpoint, to recursively debug some other code.

\setlength{\parskip}{1ex}
      \textbf{Return Value}
    \vspace{-1ex}

      \begin{quote}
      object

      \end{quote}

    \end{boxedminipage}

    \label{sys:callstats}
    \index{sys.callstats \textit{(function)}}

    \vspace{0.5ex}

\hspace{.8\funcindent}\begin{boxedminipage}{\funcwidth}

    \raggedright \textbf{callstats}()

    \vspace{-1.5ex}

    \rule{\textwidth}{0.5\fboxrule}
\setlength{\parskip}{2ex}
\begin{alltt}
Return a tuple of function call statistics, if CALL\_PROFILE was defined
when Python was built.  Otherwise, return None.

When enabled, this function returns detailed, implementation-specific
details about the number of function calls executed. The return value is
a 11-tuple where the entries in the tuple are counts of:
0. all function calls
1. calls to PyFunction\_Type objects
2. PyFunction calls that do not create an argument tuple
3. PyFunction calls that do not create an argument tuple
   and bypass PyEval\_EvalCodeEx()
4. PyMethod calls
5. PyMethod calls on bound methods
6. PyType calls
7. PyCFunction calls
8. generator calls
9. All other calls
10. Number of stack pops performed by call\_function()
\end{alltt}

\setlength{\parskip}{1ex}
      \textbf{Return Value}
    \vspace{-1ex}

      \begin{quote}
      tuple of integers

      \end{quote}

    \end{boxedminipage}

    \label{sys:displayhook}
    \index{sys.displayhook \textit{(function)}}

    \vspace{0.5ex}

\hspace{.8\funcindent}\begin{boxedminipage}{\funcwidth}

    \raggedright \textbf{displayhook}(\textit{object})

    \vspace{-1.5ex}

    \rule{\textwidth}{0.5\fboxrule}
\setlength{\parskip}{2ex}
    Print an object to sys.stdout and also save it in \_\_builtin\_\_.\_

\setlength{\parskip}{1ex}
      \textbf{Return Value}
    \vspace{-1ex}

      \begin{quote}
      None

      \end{quote}

    \end{boxedminipage}

    \label{sys:exc_clear}
    \index{sys.exc\_clear \textit{(function)}}

    \vspace{0.5ex}

\hspace{.8\funcindent}\begin{boxedminipage}{\funcwidth}

    \raggedright \textbf{exc\_clear}()

    \vspace{-1.5ex}

    \rule{\textwidth}{0.5\fboxrule}
\setlength{\parskip}{2ex}
    Clear global information on the current exception.  Subsequent calls to
    exc\_info() will return (None,None,None) until another exception is 
    raised in the current thread or the execution stack returns to a frame 
    where another exception is being handled.

\setlength{\parskip}{1ex}
      \textbf{Return Value}
    \vspace{-1ex}

      \begin{quote}
      None

      \end{quote}

    \end{boxedminipage}

    \label{sys:exc_info}
    \index{sys.exc\_info \textit{(function)}}

    \vspace{0.5ex}

\hspace{.8\funcindent}\begin{boxedminipage}{\funcwidth}

    \raggedright \textbf{exc\_info}()

    \vspace{-1.5ex}

    \rule{\textwidth}{0.5\fboxrule}
\setlength{\parskip}{2ex}
    Return information about the most recent exception caught by an except 
    clause in the current stack frame or in an older stack frame.

\setlength{\parskip}{1ex}
      \textbf{Return Value}
    \vspace{-1ex}

      \begin{quote}
      (type, value, traceback)

      \end{quote}

    \end{boxedminipage}

    \label{sys:excepthook}
    \index{sys.excepthook \textit{(function)}}

    \vspace{0.5ex}

\hspace{.8\funcindent}\begin{boxedminipage}{\funcwidth}

    \raggedright \textbf{excepthook}(\textit{exctype}, \textit{value}, \textit{traceback})

    \vspace{-1.5ex}

    \rule{\textwidth}{0.5\fboxrule}
\setlength{\parskip}{2ex}
    Handle an exception by displaying it with a traceback on sys.stderr.

\setlength{\parskip}{1ex}
      \textbf{Return Value}
    \vspace{-1ex}

      \begin{quote}
      None

      \end{quote}

    \end{boxedminipage}

    \label{sys:exit}
    \index{sys.exit \textit{(function)}}

    \vspace{0.5ex}

\hspace{.8\funcindent}\begin{boxedminipage}{\funcwidth}

    \raggedright \textbf{exit}(\textit{status}={\tt ...})

    \vspace{-1.5ex}

    \rule{\textwidth}{0.5\fboxrule}
\setlength{\parskip}{2ex}
    Exit the interpreter by raising SystemExit(status). If the status is 
    omitted or None, it defaults to zero (i.e., success). If the status is 
    an integer, it will be used as the system exit status. If it is another
    kind of object, it will be printed and the system exit status will be 
    one (i.e., failure).

\setlength{\parskip}{1ex}
    \end{boxedminipage}

    \label{sys:getcheckinterval}
    \index{sys.getcheckinterval \textit{(function)}}

    \vspace{0.5ex}

\hspace{.8\funcindent}\begin{boxedminipage}{\funcwidth}

    \raggedright \textbf{getcheckinterval}()

\setlength{\parskip}{2ex}
\setlength{\parskip}{1ex}
      \textbf{Return Value}
    \vspace{-1ex}

      \begin{quote}
      current check interval; see setcheckinterval().

      \end{quote}

    \end{boxedminipage}

    \label{sys:getdefaultencoding}
    \index{sys.getdefaultencoding \textit{(function)}}

    \vspace{0.5ex}

\hspace{.8\funcindent}\begin{boxedminipage}{\funcwidth}

    \raggedright \textbf{getdefaultencoding}()

    \vspace{-1.5ex}

    \rule{\textwidth}{0.5\fboxrule}
\setlength{\parskip}{2ex}
    Return the current default string encoding used by the Unicode 
    implementation.

\setlength{\parskip}{1ex}
      \textbf{Return Value}
    \vspace{-1ex}

      \begin{quote}
      string

      \end{quote}

    \end{boxedminipage}

    \label{sys:getdlopenflags}
    \index{sys.getdlopenflags \textit{(function)}}

    \vspace{0.5ex}

\hspace{.8\funcindent}\begin{boxedminipage}{\funcwidth}

    \raggedright \textbf{getdlopenflags}()

    \vspace{-1.5ex}

    \rule{\textwidth}{0.5\fboxrule}
\setlength{\parskip}{2ex}
    Return the current value of the flags that are used for dlopen calls. 
    The flag constants are defined in the ctypes and DLFCN modules.

\setlength{\parskip}{1ex}
      \textbf{Return Value}
    \vspace{-1ex}

      \begin{quote}
      int

      \end{quote}

    \end{boxedminipage}

    \label{sys:getfilesystemencoding}
    \index{sys.getfilesystemencoding \textit{(function)}}

    \vspace{0.5ex}

\hspace{.8\funcindent}\begin{boxedminipage}{\funcwidth}

    \raggedright \textbf{getfilesystemencoding}()

    \vspace{-1.5ex}

    \rule{\textwidth}{0.5\fboxrule}
\setlength{\parskip}{2ex}
    Return the encoding used to convert Unicode filenames in operating 
    system filenames.

\setlength{\parskip}{1ex}
      \textbf{Return Value}
    \vspace{-1ex}

      \begin{quote}
      string

      \end{quote}

    \end{boxedminipage}

    \label{sys:getprofile}
    \index{sys.getprofile \textit{(function)}}

    \vspace{0.5ex}

\hspace{.8\funcindent}\begin{boxedminipage}{\funcwidth}

    \raggedright \textbf{getprofile}()

    \vspace{-1.5ex}

    \rule{\textwidth}{0.5\fboxrule}
\setlength{\parskip}{2ex}
    Return the profiling function set with sys.setprofile. See the profiler
    chapter in the library manual.

\setlength{\parskip}{1ex}
    \end{boxedminipage}

    \label{sys:getrecursionlimit}
    \index{sys.getrecursionlimit \textit{(function)}}

    \vspace{0.5ex}

\hspace{.8\funcindent}\begin{boxedminipage}{\funcwidth}

    \raggedright \textbf{getrecursionlimit}()

    \vspace{-1.5ex}

    \rule{\textwidth}{0.5\fboxrule}
\setlength{\parskip}{2ex}
    Return the current value of the recursion limit, the maximum depth of 
    the Python interpreter stack.  This limit prevents infinite recursion 
    from causing an overflow of the C stack and crashing Python.

\setlength{\parskip}{1ex}
    \end{boxedminipage}

    \label{sys:getrefcount}
    \index{sys.getrefcount \textit{(function)}}

    \vspace{0.5ex}

\hspace{.8\funcindent}\begin{boxedminipage}{\funcwidth}

    \raggedright \textbf{getrefcount}(\textit{object})

    \vspace{-1.5ex}

    \rule{\textwidth}{0.5\fboxrule}
\setlength{\parskip}{2ex}
    Return the reference count of object.  The count returned is generally 
    one higher than you might expect, because it includes the (temporary) 
    reference as an argument to getrefcount().

\setlength{\parskip}{1ex}
      \textbf{Return Value}
    \vspace{-1ex}

      \begin{quote}
      integer

      \end{quote}

    \end{boxedminipage}

    \label{sys:getsizeof}
    \index{sys.getsizeof \textit{(function)}}

    \vspace{0.5ex}

\hspace{.8\funcindent}\begin{boxedminipage}{\funcwidth}

    \raggedright \textbf{getsizeof}(\textit{object}, \textit{default})

    \vspace{-1.5ex}

    \rule{\textwidth}{0.5\fboxrule}
\setlength{\parskip}{2ex}
    Return the size of object in bytes.

\setlength{\parskip}{1ex}
      \textbf{Return Value}
    \vspace{-1ex}

      \begin{quote}
      int

      \end{quote}

    \end{boxedminipage}

    \label{sys:gettrace}
    \index{sys.gettrace \textit{(function)}}

    \vspace{0.5ex}

\hspace{.8\funcindent}\begin{boxedminipage}{\funcwidth}

    \raggedright \textbf{gettrace}()

    \vspace{-1.5ex}

    \rule{\textwidth}{0.5\fboxrule}
\setlength{\parskip}{2ex}
    Return the global debug tracing function set with sys.settrace. See the
    debugger chapter in the library manual.

\setlength{\parskip}{1ex}
    \end{boxedminipage}

    \label{sys:setcheckinterval}
    \index{sys.setcheckinterval \textit{(function)}}

    \vspace{0.5ex}

\hspace{.8\funcindent}\begin{boxedminipage}{\funcwidth}

    \raggedright \textbf{setcheckinterval}(\textit{n})

    \vspace{-1.5ex}

    \rule{\textwidth}{0.5\fboxrule}
\setlength{\parskip}{2ex}
    Tell the Python interpreter to check for asynchronous events every n 
    instructions.  This also affects how often thread switches occur.

\setlength{\parskip}{1ex}
    \end{boxedminipage}

    \label{sys:setdlopenflags}
    \index{sys.setdlopenflags \textit{(function)}}

    \vspace{0.5ex}

\hspace{.8\funcindent}\begin{boxedminipage}{\funcwidth}

    \raggedright \textbf{setdlopenflags}(\textit{n})

    \vspace{-1.5ex}

    \rule{\textwidth}{0.5\fboxrule}
\setlength{\parskip}{2ex}
    Set the flags used by the interpreter for dlopen calls, such as when 
    the interpreter loads extension modules.  Among other things, this will
    enable a lazy resolving of symbols when importing a module, if called 
    as sys.setdlopenflags(0).  To share symbols across extension modules, 
    call as sys.setdlopenflags(ctypes.RTLD\_GLOBAL).  Symbolic names for 
    the flag modules can be either found in the ctypes module, or in the 
    DLFCN module. If DLFCN is not available, it can be generated from 
    /usr/include/dlfcn.h using the h2py script.

\setlength{\parskip}{1ex}
      \textbf{Return Value}
    \vspace{-1ex}

      \begin{quote}
      None

      \end{quote}

    \end{boxedminipage}

    \label{sys:setprofile}
    \index{sys.setprofile \textit{(function)}}

    \vspace{0.5ex}

\hspace{.8\funcindent}\begin{boxedminipage}{\funcwidth}

    \raggedright \textbf{setprofile}(\textit{function})

    \vspace{-1.5ex}

    \rule{\textwidth}{0.5\fboxrule}
\setlength{\parskip}{2ex}
    Set the profiling function.  It will be called on each function call 
    and return.  See the profiler chapter in the library manual.

\setlength{\parskip}{1ex}
    \end{boxedminipage}

    \label{sys:setrecursionlimit}
    \index{sys.setrecursionlimit \textit{(function)}}

    \vspace{0.5ex}

\hspace{.8\funcindent}\begin{boxedminipage}{\funcwidth}

    \raggedright \textbf{setrecursionlimit}(\textit{n})

    \vspace{-1.5ex}

    \rule{\textwidth}{0.5\fboxrule}
\setlength{\parskip}{2ex}
    Set the maximum depth of the Python interpreter stack to n.  This limit
    prevents infinite recursion from causing an overflow of the C stack and
    crashing Python.  The highest possible limit is platform- dependent.

\setlength{\parskip}{1ex}
    \end{boxedminipage}

    \label{sys:settrace}
    \index{sys.settrace \textit{(function)}}

    \vspace{0.5ex}

\hspace{.8\funcindent}\begin{boxedminipage}{\funcwidth}

    \raggedright \textbf{settrace}(\textit{function})

    \vspace{-1.5ex}

    \rule{\textwidth}{0.5\fboxrule}
\setlength{\parskip}{2ex}
    Set the global debug tracing function.  It will be called on each 
    function call.  See the debugger chapter in the library manual.

\setlength{\parskip}{1ex}
    \end{boxedminipage}


%%%%%%%%%%%%%%%%%%%%%%%%%%%%%%%%%%%%%%%%%%%%%%%%%%%%%%%%%%%%%%%%%%%%%%%%%%%
%%                               Variables                               %%
%%%%%%%%%%%%%%%%%%%%%%%%%%%%%%%%%%%%%%%%%%%%%%%%%%%%%%%%%%%%%%%%%%%%%%%%%%%

  \subsection{Variables}

    \vspace{-1cm}
\hspace{\varindent}\begin{longtable}{|p{\varnamewidth}|p{\vardescrwidth}|l}
\cline{1-2}
\cline{1-2} \centering \textbf{Name} & \centering \textbf{Description}& \\
\cline{1-2}
\endhead\cline{1-2}\multicolumn{3}{r}{\small\textit{continued on next page}}\\\endfoot\cline{1-2}
\endlastfoot\raggedright \_\-\_\-p\-a\-c\-k\-a\-g\-e\-\_\-\_\- & \raggedright \textbf{Value:} 
{\tt None}&\\
\cline{1-2}
\raggedright \_\-\_\-s\-t\-d\-e\-r\-r\-\_\-\_\- & \raggedright \textbf{Value:} 
{\tt {\textless}open file '{\textless}stderr{\textgreater}', mode 'w' at 0x1002931e0{\textgreater}}&\\
\cline{1-2}
\raggedright \_\-\_\-s\-t\-d\-i\-n\-\_\-\_\- & \raggedright \textbf{Value:} 
{\tt {\textless}open file '{\textless}stdin{\textgreater}', mode 'r' at 0x1002930c0{\textgreater}}&\\
\cline{1-2}
\raggedright \_\-\_\-s\-t\-d\-o\-u\-t\-\_\-\_\- & \raggedright \textbf{Value:} 
{\tt {\textless}open file '{\textless}stdout{\textgreater}', mode 'w' at 0x100293150{\textgreater}}&\\
\cline{1-2}
\raggedright a\-p\-i\-\_\-v\-e\-r\-s\-i\-o\-n\- & \raggedright \textbf{Value:} 
{\tt 1013}&\\
\cline{1-2}
\raggedright a\-r\-g\-v\- & \raggedright \textbf{Value:} 
{\tt \texttt{[}\texttt{'}\texttt{/Library/Frameworks/Python.framework/Versions/2.7/bin/e}\texttt{...}}&\\
\cline{1-2}
\raggedright b\-u\-i\-l\-t\-i\-n\-\_\-m\-o\-d\-u\-l\-e\-\_\-n\-a\-m\-e\-s\- & \raggedright \textbf{Value:} 
{\tt \texttt{(}\texttt{'}\texttt{\_\_builtin\_\_}\texttt{'}\texttt{, }\texttt{'}\texttt{\_\_main\_\_}\texttt{'}\texttt{, }\texttt{'}\texttt{\_ast}\texttt{'}\texttt{, }\texttt{'}\texttt{\_codecs}\texttt{'}\texttt{, }\texttt{'}\texttt{\_sre}\texttt{'}\texttt{, }\texttt{'}\texttt{\_}\texttt{...}}&\\
\cline{1-2}
\raggedright b\-y\-t\-e\-o\-r\-d\-e\-r\- & \raggedright \textbf{Value:} 
{\tt \texttt{'}\texttt{little}\texttt{'}}&\\
\cline{1-2}
\raggedright c\-o\-p\-y\-r\-i\-g\-h\-t\- & \raggedright \textbf{Value:} 
{\tt \texttt{'}\texttt{Copyright (c) 2001-2014 Python Software Foundation.{\textbackslash}nAll}\texttt{...}}&\\
\cline{1-2}
\raggedright d\-o\-n\-t\-\_\-w\-r\-i\-t\-e\-\_\-b\-y\-t\-e\-c\-o\-d\-e\- & \raggedright \textbf{Value:} 
{\tt False}&\\
\cline{1-2}
\raggedright e\-x\-c\-\_\-t\-y\-p\-e\- & \raggedright \textbf{Value:} 
{\tt None}&\\
\cline{1-2}
\raggedright e\-x\-e\-c\-\_\-p\-r\-e\-f\-i\-x\- & \raggedright \textbf{Value:} 
{\tt \texttt{'}\texttt{/Library/Frameworks/Python.framework/Versions/2.7}\texttt{'}}&\\
\cline{1-2}
\raggedright e\-x\-e\-c\-u\-t\-a\-b\-l\-e\- & \raggedright \textbf{Value:} 
{\tt \texttt{'}\texttt{/Library/Frameworks/Python.framework/Versions/2.7/Resour}\texttt{...}}&\\
\cline{1-2}
\raggedright f\-l\-a\-g\-s\- & \raggedright \textbf{Value:} 
{\tt sys.flags(debug=0, py3k\_warning=0, division\_warning=0, di\texttt{...}}&\\
\cline{1-2}
\raggedright f\-l\-o\-a\-t\-\_\-i\-n\-f\-o\- & \raggedright \textbf{Value:} 
{\tt sys.float\_info(max=1.7976931348623157e+308, max\_exp=1024,\texttt{...}}&\\
\cline{1-2}
\raggedright f\-l\-o\-a\-t\-\_\-r\-e\-p\-r\-\_\-s\-t\-y\-l\-e\- & \raggedright \textbf{Value:} 
{\tt \texttt{'}\texttt{short}\texttt{'}}&\\
\cline{1-2}
\raggedright h\-e\-x\-v\-e\-r\-s\-i\-o\-n\- & \raggedright \textbf{Value:} 
{\tt 34015728}&\\
\cline{1-2}
\raggedright l\-o\-n\-g\-\_\-i\-n\-f\-o\- & \raggedright \textbf{Value:} 
{\tt sys.long\_info(bits\_per\_digit=30, sizeof\_digit=4)}&\\
\cline{1-2}
\raggedright m\-a\-x\-i\-n\-t\- & \raggedright \textbf{Value:} 
{\tt 9223372036854775807}&\\
\cline{1-2}
\raggedright m\-a\-x\-s\-i\-z\-e\- & \raggedright \textbf{Value:} 
{\tt 9223372036854775807}&\\
\cline{1-2}
\raggedright m\-a\-x\-u\-n\-i\-c\-o\-d\-e\- & \raggedright \textbf{Value:} 
{\tt 65535}&\\
\cline{1-2}
\raggedright m\-e\-t\-a\-\_\-p\-a\-t\-h\- & \raggedright \textbf{Value:} 
{\tt \texttt{[}\texttt{]}}&\\
\cline{1-2}
\raggedright m\-o\-d\-u\-l\-e\-s\- & \raggedright \textbf{Value:} 
{\tt \texttt{\{}\texttt{'}\texttt{ConfigParser}\texttt{'}\texttt{: }{\textless}module 'ConfigParser' from '/Library/Fr\texttt{...}}&\\
\cline{1-2}
\raggedright p\-a\-t\-h\- & \raggedright \textbf{Value:} 
{\tt \texttt{[}\texttt{'}\texttt{/Users/tux/git}\texttt{'}\texttt{, }\texttt{'}\texttt{/Library/Frameworks/Python.framework/}\texttt{...}}&\\
\cline{1-2}
\raggedright p\-a\-t\-h\-\_\-h\-o\-o\-k\-s\- & \raggedright \textbf{Value:} 
{\tt \texttt{[}{\textless}type 'zipimport.zipimporter'{\textgreater}\texttt{]}}&\\
\cline{1-2}
\raggedright p\-a\-t\-h\-\_\-i\-m\-p\-o\-r\-t\-e\-r\-\_\-c\-a\-c\-h\-e\- & \raggedright \textbf{Value:} 
{\tt \texttt{\{}\texttt{'}\texttt{}\texttt{'}\texttt{: }None\texttt{, }\texttt{'}\texttt{/Library/Frameworks/Python.framework/Versions}\texttt{...}}&\\
\cline{1-2}
\raggedright p\-l\-a\-t\-f\-o\-r\-m\- & \raggedright \textbf{Value:} 
{\tt \texttt{'}\texttt{darwin}\texttt{'}}&\\
\cline{1-2}
\raggedright p\-r\-e\-f\-i\-x\- & \raggedright \textbf{Value:} 
{\tt \texttt{'}\texttt{/Library/Frameworks/Python.framework/Versions/2.7}\texttt{'}}&\\
\cline{1-2}
\raggedright p\-y\-3\-k\-w\-a\-r\-n\-i\-n\-g\- & \raggedright \textbf{Value:} 
{\tt False}&\\
\cline{1-2}
\raggedright s\-t\-d\-e\-r\-r\- & \raggedright \textbf{Value:} 
{\tt {\textless}open file '{\textless}stderr{\textgreater}', mode 'w' at 0x1002931e0{\textgreater}}&\\
\cline{1-2}
\raggedright s\-t\-d\-i\-n\- & \raggedright \textbf{Value:} 
{\tt {\textless}open file '{\textless}stdin{\textgreater}', mode 'r' at 0x1002930c0{\textgreater}}&\\
\cline{1-2}
\raggedright s\-t\-d\-o\-u\-t\- & \raggedright \textbf{Value:} 
{\tt {\textless}open file '{\textless}stdout{\textgreater}', mode 'w' at 0x100293150{\textgreater}}&\\
\cline{1-2}
\raggedright s\-u\-b\-v\-e\-r\-s\-i\-o\-n\- & \raggedright \textbf{Value:} 
{\tt \texttt{(}\texttt{'}\texttt{CPython}\texttt{'}\texttt{, }\texttt{'}\texttt{}\texttt{'}\texttt{, }\texttt{'}\texttt{}\texttt{'}\texttt{)}}&\\
\cline{1-2}
\raggedright v\-e\-r\-s\-i\-o\-n\- & \raggedright \textbf{Value:} 
{\tt \texttt{'}\texttt{2.7.9 (v2.7.9:648dcafa7e5f, Dec 10 2014, 10:10:46) {\textbackslash}n[GC}\texttt{...}}&\\
\cline{1-2}
\raggedright v\-e\-r\-s\-i\-o\-n\-\_\-i\-n\-f\-o\- & \raggedright \textbf{Value:} 
{\tt sys.version\_info(major=2, minor=7, micro=9, releaselevel=\texttt{...}}&\\
\cline{1-2}
\raggedright w\-a\-r\-n\-o\-p\-t\-i\-o\-n\-s\- & \raggedright \textbf{Value:} 
{\tt \texttt{[}\texttt{]}}&\\
\cline{1-2}
\end{longtable}

    \index{sys \textit{(module)}|)}
