%
% API Documentation for API Documentation
% Module os
%
% Generated by epydoc 3.0.1
% [Thu Dec 10 09:57:57 2015]
%

%%%%%%%%%%%%%%%%%%%%%%%%%%%%%%%%%%%%%%%%%%%%%%%%%%%%%%%%%%%%%%%%%%%%%%%%%%%
%%                          Module Description                           %%
%%%%%%%%%%%%%%%%%%%%%%%%%%%%%%%%%%%%%%%%%%%%%%%%%%%%%%%%%%%%%%%%%%%%%%%%%%%

    \index{os \textit{(module)}|(}
\section{Module os}

    \label{os}
OS routines for NT or Posix depending on what system we're on.

This exports:

\begin{itemize}
\setlength{\parskip}{0.6ex}
  \item all functions from posix, nt, os2, or ce, e.g. unlink, stat, etc.

  \item os.path is one of the modules posixpath, or ntpath

  \item os.name is 'posix', 'nt', 'os2', 'ce' or 'riscos'

  \item os.curdir is a string representing the current directory ('.' or ':')

  \item os.pardir is a string representing the parent directory ('..' or '::')

  \item os.sep is the (or a most common) pathname separator ('/' or ':' or 
    '{\textbackslash}{\textbackslash}')

  \item os.extsep is the extension separator ('.' or '/')

  \item os.altsep is the alternate pathname separator (None or '/')

  \item os.pathsep is the component separator used in \$PATH etc

  \item os.linesep is the line separator in text files ('{\textbackslash}r' or 
    '{\textbackslash}n' or '{\textbackslash}r{\textbackslash}n')

  \item os.defpath is the default search path for executables

  \item os.devnull is the file path of the null device ('/dev/null', etc.)

\end{itemize}

Programs that import and use 'os' stand a better chance of being portable 
between different platforms.  Of course, they must then only use functions 
that are defined by all platforms (e.g., unlink and opendir), and leave all
pathname manipulation to os.path (e.g., split and join).


%%%%%%%%%%%%%%%%%%%%%%%%%%%%%%%%%%%%%%%%%%%%%%%%%%%%%%%%%%%%%%%%%%%%%%%%%%%
%%                               Functions                               %%
%%%%%%%%%%%%%%%%%%%%%%%%%%%%%%%%%%%%%%%%%%%%%%%%%%%%%%%%%%%%%%%%%%%%%%%%%%%

  \subsection{Functions}

    \label{os:WCOREDUMP}
    \index{os.WCOREDUMP \textit{(function)}}

    \vspace{0.5ex}

\hspace{.8\funcindent}\begin{boxedminipage}{\funcwidth}

    \raggedright \textbf{WCOREDUMP}(\textit{status})

    \vspace{-1.5ex}

    \rule{\textwidth}{0.5\fboxrule}
\setlength{\parskip}{2ex}
    Return True if the process returning 'status' was dumped to a core 
    file.

\setlength{\parskip}{1ex}
      \textbf{Return Value}
    \vspace{-1ex}

      \begin{quote}
      bool

      \end{quote}

    \end{boxedminipage}

    \label{os:WEXITSTATUS}
    \index{os.WEXITSTATUS \textit{(function)}}

    \vspace{0.5ex}

\hspace{.8\funcindent}\begin{boxedminipage}{\funcwidth}

    \raggedright \textbf{WEXITSTATUS}(\textit{status})

    \vspace{-1.5ex}

    \rule{\textwidth}{0.5\fboxrule}
\setlength{\parskip}{2ex}
    Return the process return code from 'status'.

\setlength{\parskip}{1ex}
      \textbf{Return Value}
    \vspace{-1ex}

      \begin{quote}
      integer

      \end{quote}

    \end{boxedminipage}

    \label{os:WIFCONTINUED}
    \index{os.WIFCONTINUED \textit{(function)}}

    \vspace{0.5ex}

\hspace{.8\funcindent}\begin{boxedminipage}{\funcwidth}

    \raggedright \textbf{WIFCONTINUED}(\textit{status})

    \vspace{-1.5ex}

    \rule{\textwidth}{0.5\fboxrule}
\setlength{\parskip}{2ex}
    Return True if the process returning 'status' was continued from a job 
    control stop.

\setlength{\parskip}{1ex}
      \textbf{Return Value}
    \vspace{-1ex}

      \begin{quote}
      bool

      \end{quote}

    \end{boxedminipage}

    \label{os:WIFEXITED}
    \index{os.WIFEXITED \textit{(function)}}

    \vspace{0.5ex}

\hspace{.8\funcindent}\begin{boxedminipage}{\funcwidth}

    \raggedright \textbf{WIFEXITED}(\textit{status})

    \vspace{-1.5ex}

    \rule{\textwidth}{0.5\fboxrule}
\setlength{\parskip}{2ex}
    Return true if the process returning 'status' exited using the exit() 
    system call.

\setlength{\parskip}{1ex}
      \textbf{Return Value}
    \vspace{-1ex}

      \begin{quote}
      bool

      \end{quote}

    \end{boxedminipage}

    \label{os:WIFSIGNALED}
    \index{os.WIFSIGNALED \textit{(function)}}

    \vspace{0.5ex}

\hspace{.8\funcindent}\begin{boxedminipage}{\funcwidth}

    \raggedright \textbf{WIFSIGNALED}(\textit{status})

    \vspace{-1.5ex}

    \rule{\textwidth}{0.5\fboxrule}
\setlength{\parskip}{2ex}
    Return True if the process returning 'status' was terminated by a 
    signal.

\setlength{\parskip}{1ex}
      \textbf{Return Value}
    \vspace{-1ex}

      \begin{quote}
      bool

      \end{quote}

    \end{boxedminipage}

    \label{os:WIFSTOPPED}
    \index{os.WIFSTOPPED \textit{(function)}}

    \vspace{0.5ex}

\hspace{.8\funcindent}\begin{boxedminipage}{\funcwidth}

    \raggedright \textbf{WIFSTOPPED}(\textit{status})

    \vspace{-1.5ex}

    \rule{\textwidth}{0.5\fboxrule}
\setlength{\parskip}{2ex}
    Return True if the process returning 'status' was stopped.

\setlength{\parskip}{1ex}
      \textbf{Return Value}
    \vspace{-1ex}

      \begin{quote}
      bool

      \end{quote}

    \end{boxedminipage}

    \label{os:WSTOPSIG}
    \index{os.WSTOPSIG \textit{(function)}}

    \vspace{0.5ex}

\hspace{.8\funcindent}\begin{boxedminipage}{\funcwidth}

    \raggedright \textbf{WSTOPSIG}(\textit{status})

    \vspace{-1.5ex}

    \rule{\textwidth}{0.5\fboxrule}
\setlength{\parskip}{2ex}
    Return the signal that stopped the process that provided the 'status' 
    value.

\setlength{\parskip}{1ex}
      \textbf{Return Value}
    \vspace{-1ex}

      \begin{quote}
      integer

      \end{quote}

    \end{boxedminipage}

    \label{os:WTERMSIG}
    \index{os.WTERMSIG \textit{(function)}}

    \vspace{0.5ex}

\hspace{.8\funcindent}\begin{boxedminipage}{\funcwidth}

    \raggedright \textbf{WTERMSIG}(\textit{status})

    \vspace{-1.5ex}

    \rule{\textwidth}{0.5\fboxrule}
\setlength{\parskip}{2ex}
    Return the signal that terminated the process that provided the 
    'status' value.

\setlength{\parskip}{1ex}
      \textbf{Return Value}
    \vspace{-1ex}

      \begin{quote}
      integer

      \end{quote}

    \end{boxedminipage}

    \label{os:abort}
    \index{os.abort \textit{(function)}}

    \vspace{0.5ex}

\hspace{.8\funcindent}\begin{boxedminipage}{\funcwidth}

    \raggedright \textbf{abort}()

    \vspace{-1.5ex}

    \rule{\textwidth}{0.5\fboxrule}
\setlength{\parskip}{2ex}
    Abort the interpreter immediately.  This 'dumps core' or otherwise 
    fails in the hardest way possible on the hosting operating system.

\setlength{\parskip}{1ex}
      \textbf{Return Value}
    \vspace{-1ex}

      \begin{quote}
      does not return!

      \end{quote}

    \end{boxedminipage}

    \label{os:access}
    \index{os.access \textit{(function)}}

    \vspace{0.5ex}

\hspace{.8\funcindent}\begin{boxedminipage}{\funcwidth}

    \raggedright \textbf{access}(\textit{path}, \textit{mode})

    \vspace{-1.5ex}

    \rule{\textwidth}{0.5\fboxrule}
\setlength{\parskip}{2ex}
    Use the real uid/gid to test for access to a path.  Note that most 
    operations will use the effective uid/gid, therefore this routine can 
    be used in a suid/sgid environment to test if the invoking user has the
    specified access to the path.  The mode argument can be F\_OK to test 
    existence, or the inclusive-OR of R\_OK, W\_OK, and X\_OK.

\setlength{\parskip}{1ex}
      \textbf{Return Value}
    \vspace{-1ex}

      \begin{quote}
      True if granted, False otherwise

      \end{quote}

    \end{boxedminipage}

    \label{os:chdir}
    \index{os.chdir \textit{(function)}}

    \vspace{0.5ex}

\hspace{.8\funcindent}\begin{boxedminipage}{\funcwidth}

    \raggedright \textbf{chdir}(\textit{path})

    \vspace{-1.5ex}

    \rule{\textwidth}{0.5\fboxrule}
\setlength{\parskip}{2ex}
    Change the current working directory to the specified path.

\setlength{\parskip}{1ex}
    \end{boxedminipage}

    \label{os:chflags}
    \index{os.chflags \textit{(function)}}

    \vspace{0.5ex}

\hspace{.8\funcindent}\begin{boxedminipage}{\funcwidth}

    \raggedright \textbf{chflags}(\textit{path}, \textit{flags})

    \vspace{-1.5ex}

    \rule{\textwidth}{0.5\fboxrule}
\setlength{\parskip}{2ex}
    Set file flags.

\setlength{\parskip}{1ex}
    \end{boxedminipage}

    \label{os:chmod}
    \index{os.chmod \textit{(function)}}

    \vspace{0.5ex}

\hspace{.8\funcindent}\begin{boxedminipage}{\funcwidth}

    \raggedright \textbf{chmod}(\textit{path}, \textit{mode})

    \vspace{-1.5ex}

    \rule{\textwidth}{0.5\fboxrule}
\setlength{\parskip}{2ex}
    Change the access permissions of a file.

\setlength{\parskip}{1ex}
    \end{boxedminipage}

    \label{os:chown}
    \index{os.chown \textit{(function)}}

    \vspace{0.5ex}

\hspace{.8\funcindent}\begin{boxedminipage}{\funcwidth}

    \raggedright \textbf{chown}(\textit{path}, \textit{uid}, \textit{gid})

    \vspace{-1.5ex}

    \rule{\textwidth}{0.5\fboxrule}
\setlength{\parskip}{2ex}
    Change the owner and group id of path to the numeric uid and gid.

\setlength{\parskip}{1ex}
    \end{boxedminipage}

    \label{os:chroot}
    \index{os.chroot \textit{(function)}}

    \vspace{0.5ex}

\hspace{.8\funcindent}\begin{boxedminipage}{\funcwidth}

    \raggedright \textbf{chroot}(\textit{path})

    \vspace{-1.5ex}

    \rule{\textwidth}{0.5\fboxrule}
\setlength{\parskip}{2ex}
    Change root directory to path.

\setlength{\parskip}{1ex}
    \end{boxedminipage}

    \label{os:close}
    \index{os.close \textit{(function)}}

    \vspace{0.5ex}

\hspace{.8\funcindent}\begin{boxedminipage}{\funcwidth}

    \raggedright \textbf{close}(\textit{fd})

    \vspace{-1.5ex}

    \rule{\textwidth}{0.5\fboxrule}
\setlength{\parskip}{2ex}
    Close a file descriptor (for low level IO).

\setlength{\parskip}{1ex}
    \end{boxedminipage}

    \label{os:closerange}
    \index{os.closerange \textit{(function)}}

    \vspace{0.5ex}

\hspace{.8\funcindent}\begin{boxedminipage}{\funcwidth}

    \raggedright \textbf{closerange}(\textit{fd\_low}, \textit{fd\_high})

    \vspace{-1.5ex}

    \rule{\textwidth}{0.5\fboxrule}
\setlength{\parskip}{2ex}
    Closes all file descriptors in [fd\_low, fd\_high), ignoring errors.

\setlength{\parskip}{1ex}
    \end{boxedminipage}

    \label{os:confstr}
    \index{os.confstr \textit{(function)}}

    \vspace{0.5ex}

\hspace{.8\funcindent}\begin{boxedminipage}{\funcwidth}

    \raggedright \textbf{confstr}(\textit{name})

    \vspace{-1.5ex}

    \rule{\textwidth}{0.5\fboxrule}
\setlength{\parskip}{2ex}
    Return a string-valued system configuration variable.

\setlength{\parskip}{1ex}
      \textbf{Return Value}
    \vspace{-1ex}

      \begin{quote}
      string

      \end{quote}

    \end{boxedminipage}

    \label{os:ctermid}
    \index{os.ctermid \textit{(function)}}

    \vspace{0.5ex}

\hspace{.8\funcindent}\begin{boxedminipage}{\funcwidth}

    \raggedright \textbf{ctermid}()

    \vspace{-1.5ex}

    \rule{\textwidth}{0.5\fboxrule}
\setlength{\parskip}{2ex}
    Return the name of the controlling terminal for this process.

\setlength{\parskip}{1ex}
      \textbf{Return Value}
    \vspace{-1ex}

      \begin{quote}
      string

      \end{quote}

    \end{boxedminipage}

    \label{os:dup}
    \index{os.dup \textit{(function)}}

    \vspace{0.5ex}

\hspace{.8\funcindent}\begin{boxedminipage}{\funcwidth}

    \raggedright \textbf{dup}(\textit{fd})

    \vspace{-1.5ex}

    \rule{\textwidth}{0.5\fboxrule}
\setlength{\parskip}{2ex}
    Return a duplicate of a file descriptor.

\setlength{\parskip}{1ex}
      \textbf{Return Value}
    \vspace{-1ex}

      \begin{quote}
      fd2

      \end{quote}

    \end{boxedminipage}

    \label{os:dup2}
    \index{os.dup2 \textit{(function)}}

    \vspace{0.5ex}

\hspace{.8\funcindent}\begin{boxedminipage}{\funcwidth}

    \raggedright \textbf{dup2}(\textit{old\_fd}, \textit{new\_fd})

    \vspace{-1.5ex}

    \rule{\textwidth}{0.5\fboxrule}
\setlength{\parskip}{2ex}
    Duplicate file descriptor.

\setlength{\parskip}{1ex}
    \end{boxedminipage}

    \label{os:execl}
    \index{os.execl \textit{(function)}}

    \vspace{0.5ex}

\hspace{.8\funcindent}\begin{boxedminipage}{\funcwidth}

    \raggedright \textbf{execl}(\textit{file}, *\textit{args})

    \vspace{-1.5ex}

    \rule{\textwidth}{0.5\fboxrule}
\setlength{\parskip}{2ex}
    Execute the executable file with argument list args, replacing the 
    current process.

\setlength{\parskip}{1ex}
    \end{boxedminipage}

    \label{os:execle}
    \index{os.execle \textit{(function)}}

    \vspace{0.5ex}

\hspace{.8\funcindent}\begin{boxedminipage}{\funcwidth}

    \raggedright \textbf{execle}(\textit{file}, \textit{env}, *\textit{args})

    \vspace{-1.5ex}

    \rule{\textwidth}{0.5\fboxrule}
\setlength{\parskip}{2ex}
    Execute the executable file with argument list args and environment 
    env, replacing the current process.

\setlength{\parskip}{1ex}
    \end{boxedminipage}

    \label{os:execlp}
    \index{os.execlp \textit{(function)}}

    \vspace{0.5ex}

\hspace{.8\funcindent}\begin{boxedminipage}{\funcwidth}

    \raggedright \textbf{execlp}(\textit{file}, *\textit{args})

    \vspace{-1.5ex}

    \rule{\textwidth}{0.5\fboxrule}
\setlength{\parskip}{2ex}
    Execute the executable file (which is searched for along \$PATH) with 
    argument list args, replacing the current process.

\setlength{\parskip}{1ex}
    \end{boxedminipage}

    \label{os:execlpe}
    \index{os.execlpe \textit{(function)}}

    \vspace{0.5ex}

\hspace{.8\funcindent}\begin{boxedminipage}{\funcwidth}

    \raggedright \textbf{execlpe}(\textit{file}, \textit{env}, *\textit{args})

    \vspace{-1.5ex}

    \rule{\textwidth}{0.5\fboxrule}
\setlength{\parskip}{2ex}
    Execute the executable file (which is searched for along \$PATH) with 
    argument list args and environment env, replacing the current process.

\setlength{\parskip}{1ex}
    \end{boxedminipage}

    \label{os:execv}
    \index{os.execv \textit{(function)}}

    \vspace{0.5ex}

\hspace{.8\funcindent}\begin{boxedminipage}{\funcwidth}

    \raggedright \textbf{execv}(\textit{path}, \textit{args})

    \vspace{-1.5ex}

    \rule{\textwidth}{0.5\fboxrule}
\setlength{\parskip}{2ex}
\begin{alltt}
Execute an executable path with arguments, replacing current process.

    path: path of executable file
    args: tuple or list of strings
\end{alltt}

\setlength{\parskip}{1ex}
    \end{boxedminipage}

    \label{os:execve}
    \index{os.execve \textit{(function)}}

    \vspace{0.5ex}

\hspace{.8\funcindent}\begin{boxedminipage}{\funcwidth}

    \raggedright \textbf{execve}(\textit{path}, \textit{args}, \textit{env})

    \vspace{-1.5ex}

    \rule{\textwidth}{0.5\fboxrule}
\setlength{\parskip}{2ex}
\begin{alltt}
Execute a path with arguments and environment, replacing current process.

    path: path of executable file
    args: tuple or list of arguments
    env: dictionary of strings mapping to strings
\end{alltt}

\setlength{\parskip}{1ex}
    \end{boxedminipage}

    \label{os:execvp}
    \index{os.execvp \textit{(function)}}

    \vspace{0.5ex}

\hspace{.8\funcindent}\begin{boxedminipage}{\funcwidth}

    \raggedright \textbf{execvp}(\textit{file}, \textit{args})

    \vspace{-1.5ex}

    \rule{\textwidth}{0.5\fboxrule}
\setlength{\parskip}{2ex}
    Execute the executable file (which is searched for along \$PATH) with 
    argument list args, replacing the current process. args may be a list 
    or tuple of strings.

\setlength{\parskip}{1ex}
    \end{boxedminipage}

    \label{os:execvpe}
    \index{os.execvpe \textit{(function)}}

    \vspace{0.5ex}

\hspace{.8\funcindent}\begin{boxedminipage}{\funcwidth}

    \raggedright \textbf{execvpe}(\textit{file}, \textit{args}, \textit{env})

    \vspace{-1.5ex}

    \rule{\textwidth}{0.5\fboxrule}
\setlength{\parskip}{2ex}
    Execute the executable file (which is searched for along \$PATH) with 
    argument list args and environment env , replacing the current process.
    args may be a list or tuple of strings.

\setlength{\parskip}{1ex}
    \end{boxedminipage}

    \label{os:fchdir}
    \index{os.fchdir \textit{(function)}}

    \vspace{0.5ex}

\hspace{.8\funcindent}\begin{boxedminipage}{\funcwidth}

    \raggedright \textbf{fchdir}(\textit{fildes})

    \vspace{-1.5ex}

    \rule{\textwidth}{0.5\fboxrule}
\setlength{\parskip}{2ex}
    Change to the directory of the given file descriptor.  fildes must be 
    opened on a directory, not a file.

\setlength{\parskip}{1ex}
    \end{boxedminipage}

    \label{os:fchmod}
    \index{os.fchmod \textit{(function)}}

    \vspace{0.5ex}

\hspace{.8\funcindent}\begin{boxedminipage}{\funcwidth}

    \raggedright \textbf{fchmod}(\textit{fd}, \textit{mode})

    \vspace{-1.5ex}

    \rule{\textwidth}{0.5\fboxrule}
\setlength{\parskip}{2ex}
    Change the access permissions of the file given by file descriptor fd.

\setlength{\parskip}{1ex}
    \end{boxedminipage}

    \label{os:fchown}
    \index{os.fchown \textit{(function)}}

    \vspace{0.5ex}

\hspace{.8\funcindent}\begin{boxedminipage}{\funcwidth}

    \raggedright \textbf{fchown}(\textit{fd}, \textit{uid}, \textit{gid})

    \vspace{-1.5ex}

    \rule{\textwidth}{0.5\fboxrule}
\setlength{\parskip}{2ex}
    Change the owner and group id of the file given by file descriptor fd 
    to the numeric uid and gid.

\setlength{\parskip}{1ex}
    \end{boxedminipage}

    \label{os:fdopen}
    \index{os.fdopen \textit{(function)}}

    \vspace{0.5ex}

\hspace{.8\funcindent}\begin{boxedminipage}{\funcwidth}

    \raggedright \textbf{fdopen}(\textit{fd}, \textit{mode}={\tt 'r' }, \textit{bufsize}={\tt ...})

    \vspace{-1.5ex}

    \rule{\textwidth}{0.5\fboxrule}
\setlength{\parskip}{2ex}
    Return an open file object connected to a file descriptor.

\setlength{\parskip}{1ex}
      \textbf{Return Value}
    \vspace{-1ex}

      \begin{quote}
      file\_object

      \end{quote}

    \end{boxedminipage}

    \label{os:fork}
    \index{os.fork \textit{(function)}}

    \vspace{0.5ex}

\hspace{.8\funcindent}\begin{boxedminipage}{\funcwidth}

    \raggedright \textbf{fork}()

    \vspace{-1.5ex}

    \rule{\textwidth}{0.5\fboxrule}
\setlength{\parskip}{2ex}
    Fork a child process. Return 0 to child process and PID of child to 
    parent process.

\setlength{\parskip}{1ex}
      \textbf{Return Value}
    \vspace{-1ex}

      \begin{quote}
      pid

      \end{quote}

    \end{boxedminipage}

    \label{os:forkpty}
    \index{os.forkpty \textit{(function)}}

    \vspace{0.5ex}

\hspace{.8\funcindent}\begin{boxedminipage}{\funcwidth}

    \raggedright \textbf{forkpty}()

    \vspace{-1.5ex}

    \rule{\textwidth}{0.5\fboxrule}
\setlength{\parskip}{2ex}
    Fork a new process with a new pseudo-terminal as controlling tty.

    Like fork(), return 0 as pid to child process, and PID of child to 
    parent. To both, return fd of newly opened pseudo-terminal.

\setlength{\parskip}{1ex}
      \textbf{Return Value}
    \vspace{-1ex}

      \begin{quote}
      (pid, master\_fd)

      \end{quote}

    \end{boxedminipage}

    \label{os:fpathconf}
    \index{os.fpathconf \textit{(function)}}

    \vspace{0.5ex}

\hspace{.8\funcindent}\begin{boxedminipage}{\funcwidth}

    \raggedright \textbf{fpathconf}(\textit{fd}, \textit{name})

    \vspace{-1.5ex}

    \rule{\textwidth}{0.5\fboxrule}
\setlength{\parskip}{2ex}
    Return the configuration limit name for the file descriptor fd. If 
    there is no limit, return -1.

\setlength{\parskip}{1ex}
      \textbf{Return Value}
    \vspace{-1ex}

      \begin{quote}
      integer

      \end{quote}

    \end{boxedminipage}

    \label{os:fstat}
    \index{os.fstat \textit{(function)}}

    \vspace{0.5ex}

\hspace{.8\funcindent}\begin{boxedminipage}{\funcwidth}

    \raggedright \textbf{fstat}(\textit{fd})

    \vspace{-1.5ex}

    \rule{\textwidth}{0.5\fboxrule}
\setlength{\parskip}{2ex}
    Like stat(), but for an open file descriptor.

\setlength{\parskip}{1ex}
      \textbf{Return Value}
    \vspace{-1ex}

      \begin{quote}
      stat result

      \end{quote}

    \end{boxedminipage}

    \label{os:fstatvfs}
    \index{os.fstatvfs \textit{(function)}}

    \vspace{0.5ex}

\hspace{.8\funcindent}\begin{boxedminipage}{\funcwidth}

    \raggedright \textbf{fstatvfs}(\textit{fd})

    \vspace{-1.5ex}

    \rule{\textwidth}{0.5\fboxrule}
\setlength{\parskip}{2ex}
    Perform an fstatvfs system call on the given fd.

\setlength{\parskip}{1ex}
      \textbf{Return Value}
    \vspace{-1ex}

      \begin{quote}
      statvfs result

      \end{quote}

    \end{boxedminipage}

    \label{os:fsync}
    \index{os.fsync \textit{(function)}}

    \vspace{0.5ex}

\hspace{.8\funcindent}\begin{boxedminipage}{\funcwidth}

    \raggedright \textbf{fsync}(\textit{fildes})

    \vspace{-1.5ex}

    \rule{\textwidth}{0.5\fboxrule}
\setlength{\parskip}{2ex}
    force write of file with filedescriptor to disk.

\setlength{\parskip}{1ex}
    \end{boxedminipage}

    \label{os:ftruncate}
    \index{os.ftruncate \textit{(function)}}

    \vspace{0.5ex}

\hspace{.8\funcindent}\begin{boxedminipage}{\funcwidth}

    \raggedright \textbf{ftruncate}(\textit{fd}, \textit{length})

    \vspace{-1.5ex}

    \rule{\textwidth}{0.5\fboxrule}
\setlength{\parskip}{2ex}
    Truncate a file to a specified length.

\setlength{\parskip}{1ex}
    \end{boxedminipage}

    \label{os:getcwd}
    \index{os.getcwd \textit{(function)}}

    \vspace{0.5ex}

\hspace{.8\funcindent}\begin{boxedminipage}{\funcwidth}

    \raggedright \textbf{getcwd}()

    \vspace{-1.5ex}

    \rule{\textwidth}{0.5\fboxrule}
\setlength{\parskip}{2ex}
    Return a string representing the current working directory.

\setlength{\parskip}{1ex}
      \textbf{Return Value}
    \vspace{-1ex}

      \begin{quote}
      path

      \end{quote}

    \end{boxedminipage}

    \label{os:getcwdu}
    \index{os.getcwdu \textit{(function)}}

    \vspace{0.5ex}

\hspace{.8\funcindent}\begin{boxedminipage}{\funcwidth}

    \raggedright \textbf{getcwdu}()

    \vspace{-1.5ex}

    \rule{\textwidth}{0.5\fboxrule}
\setlength{\parskip}{2ex}
    Return a unicode string representing the current working directory.

\setlength{\parskip}{1ex}
      \textbf{Return Value}
    \vspace{-1ex}

      \begin{quote}
      path

      \end{quote}

    \end{boxedminipage}

    \label{os:getegid}
    \index{os.getegid \textit{(function)}}

    \vspace{0.5ex}

\hspace{.8\funcindent}\begin{boxedminipage}{\funcwidth}

    \raggedright \textbf{getegid}()

    \vspace{-1.5ex}

    \rule{\textwidth}{0.5\fboxrule}
\setlength{\parskip}{2ex}
    Return the current process's effective group id.

\setlength{\parskip}{1ex}
      \textbf{Return Value}
    \vspace{-1ex}

      \begin{quote}
      egid

      \end{quote}

    \end{boxedminipage}

    \label{os:getenv}
    \index{os.getenv \textit{(function)}}

    \vspace{0.5ex}

\hspace{.8\funcindent}\begin{boxedminipage}{\funcwidth}

    \raggedright \textbf{getenv}(\textit{key}, \textit{default}={\tt None})

    \vspace{-1.5ex}

    \rule{\textwidth}{0.5\fboxrule}
\setlength{\parskip}{2ex}
    Get an environment variable, return None if it doesn't exist. The 
    optional second argument can specify an alternate default.

\setlength{\parskip}{1ex}
    \end{boxedminipage}

    \label{os:geteuid}
    \index{os.geteuid \textit{(function)}}

    \vspace{0.5ex}

\hspace{.8\funcindent}\begin{boxedminipage}{\funcwidth}

    \raggedright \textbf{geteuid}()

    \vspace{-1.5ex}

    \rule{\textwidth}{0.5\fboxrule}
\setlength{\parskip}{2ex}
    Return the current process's effective user id.

\setlength{\parskip}{1ex}
      \textbf{Return Value}
    \vspace{-1ex}

      \begin{quote}
      euid

      \end{quote}

    \end{boxedminipage}

    \label{os:getgid}
    \index{os.getgid \textit{(function)}}

    \vspace{0.5ex}

\hspace{.8\funcindent}\begin{boxedminipage}{\funcwidth}

    \raggedright \textbf{getgid}()

    \vspace{-1.5ex}

    \rule{\textwidth}{0.5\fboxrule}
\setlength{\parskip}{2ex}
    Return the current process's group id.

\setlength{\parskip}{1ex}
      \textbf{Return Value}
    \vspace{-1ex}

      \begin{quote}
      gid

      \end{quote}

    \end{boxedminipage}

    \label{os:getgroups}
    \index{os.getgroups \textit{(function)}}

    \vspace{0.5ex}

\hspace{.8\funcindent}\begin{boxedminipage}{\funcwidth}

    \raggedright \textbf{getgroups}()

    \vspace{-1.5ex}

    \rule{\textwidth}{0.5\fboxrule}
\setlength{\parskip}{2ex}
    Return list of supplemental group IDs for the process.

\setlength{\parskip}{1ex}
      \textbf{Return Value}
    \vspace{-1ex}

      \begin{quote}
      list of group IDs

      \end{quote}

    \end{boxedminipage}

    \label{os:getloadavg}
    \index{os.getloadavg \textit{(function)}}

    \vspace{0.5ex}

\hspace{.8\funcindent}\begin{boxedminipage}{\funcwidth}

    \raggedright \textbf{getloadavg}()

    \vspace{-1.5ex}

    \rule{\textwidth}{0.5\fboxrule}
\setlength{\parskip}{2ex}
    Return the number of processes in the system run queue averaged over 
    the last 1, 5, and 15 minutes or raises OSError if the load average was
    unobtainable

\setlength{\parskip}{1ex}
      \textbf{Return Value}
    \vspace{-1ex}

      \begin{quote}
      (float, float, float)

      \end{quote}

    \end{boxedminipage}

    \label{os:getlogin}
    \index{os.getlogin \textit{(function)}}

    \vspace{0.5ex}

\hspace{.8\funcindent}\begin{boxedminipage}{\funcwidth}

    \raggedright \textbf{getlogin}()

    \vspace{-1.5ex}

    \rule{\textwidth}{0.5\fboxrule}
\setlength{\parskip}{2ex}
    Return the actual login name.

\setlength{\parskip}{1ex}
      \textbf{Return Value}
    \vspace{-1ex}

      \begin{quote}
      string

      \end{quote}

    \end{boxedminipage}

    \label{os:getpgid}
    \index{os.getpgid \textit{(function)}}

    \vspace{0.5ex}

\hspace{.8\funcindent}\begin{boxedminipage}{\funcwidth}

    \raggedright \textbf{getpgid}(\textit{pid})

    \vspace{-1.5ex}

    \rule{\textwidth}{0.5\fboxrule}
\setlength{\parskip}{2ex}
    Call the system call getpgid().

\setlength{\parskip}{1ex}
      \textbf{Return Value}
    \vspace{-1ex}

      \begin{quote}
      pgid

      \end{quote}

    \end{boxedminipage}

    \label{os:getpgrp}
    \index{os.getpgrp \textit{(function)}}

    \vspace{0.5ex}

\hspace{.8\funcindent}\begin{boxedminipage}{\funcwidth}

    \raggedright \textbf{getpgrp}()

    \vspace{-1.5ex}

    \rule{\textwidth}{0.5\fboxrule}
\setlength{\parskip}{2ex}
    Return the current process group id.

\setlength{\parskip}{1ex}
      \textbf{Return Value}
    \vspace{-1ex}

      \begin{quote}
      pgrp

      \end{quote}

    \end{boxedminipage}

    \label{os:getpid}
    \index{os.getpid \textit{(function)}}

    \vspace{0.5ex}

\hspace{.8\funcindent}\begin{boxedminipage}{\funcwidth}

    \raggedright \textbf{getpid}()

    \vspace{-1.5ex}

    \rule{\textwidth}{0.5\fboxrule}
\setlength{\parskip}{2ex}
    Return the current process id

\setlength{\parskip}{1ex}
      \textbf{Return Value}
    \vspace{-1ex}

      \begin{quote}
      pid

      \end{quote}

    \end{boxedminipage}

    \label{os:getppid}
    \index{os.getppid \textit{(function)}}

    \vspace{0.5ex}

\hspace{.8\funcindent}\begin{boxedminipage}{\funcwidth}

    \raggedright \textbf{getppid}()

    \vspace{-1.5ex}

    \rule{\textwidth}{0.5\fboxrule}
\setlength{\parskip}{2ex}
    Return the parent's process id.

\setlength{\parskip}{1ex}
      \textbf{Return Value}
    \vspace{-1ex}

      \begin{quote}
      ppid

      \end{quote}

    \end{boxedminipage}

    \label{os:getsid}
    \index{os.getsid \textit{(function)}}

    \vspace{0.5ex}

\hspace{.8\funcindent}\begin{boxedminipage}{\funcwidth}

    \raggedright \textbf{getsid}(\textit{pid})

    \vspace{-1.5ex}

    \rule{\textwidth}{0.5\fboxrule}
\setlength{\parskip}{2ex}
    Call the system call getsid().

\setlength{\parskip}{1ex}
      \textbf{Return Value}
    \vspace{-1ex}

      \begin{quote}
      sid

      \end{quote}

    \end{boxedminipage}

    \label{os:getuid}
    \index{os.getuid \textit{(function)}}

    \vspace{0.5ex}

\hspace{.8\funcindent}\begin{boxedminipage}{\funcwidth}

    \raggedright \textbf{getuid}()

    \vspace{-1.5ex}

    \rule{\textwidth}{0.5\fboxrule}
\setlength{\parskip}{2ex}
    Return the current process's user id.

\setlength{\parskip}{1ex}
      \textbf{Return Value}
    \vspace{-1ex}

      \begin{quote}
      uid

      \end{quote}

    \end{boxedminipage}

    \label{os:initgroups}
    \index{os.initgroups \textit{(function)}}

    \vspace{0.5ex}

\hspace{.8\funcindent}\begin{boxedminipage}{\funcwidth}

    \raggedright \textbf{initgroups}(\textit{username}, \textit{gid})

    \vspace{-1.5ex}

    \rule{\textwidth}{0.5\fboxrule}
\setlength{\parskip}{2ex}
    Call the system initgroups() to initialize the group access list with 
    all of the groups of which the specified username is a member, plus the
    specified group id.

\setlength{\parskip}{1ex}
      \textbf{Return Value}
    \vspace{-1ex}

      \begin{quote}
      None

      \end{quote}

    \end{boxedminipage}

    \label{os:isatty}
    \index{os.isatty \textit{(function)}}

    \vspace{0.5ex}

\hspace{.8\funcindent}\begin{boxedminipage}{\funcwidth}

    \raggedright \textbf{isatty}(\textit{fd})

    \vspace{-1.5ex}

    \rule{\textwidth}{0.5\fboxrule}
\setlength{\parskip}{2ex}
    Return True if the file descriptor 'fd' is an open file descriptor 
    connected to the slave end of a terminal.

\setlength{\parskip}{1ex}
      \textbf{Return Value}
    \vspace{-1ex}

      \begin{quote}
      bool

      \end{quote}

    \end{boxedminipage}

    \label{os:kill}
    \index{os.kill \textit{(function)}}

    \vspace{0.5ex}

\hspace{.8\funcindent}\begin{boxedminipage}{\funcwidth}

    \raggedright \textbf{kill}(\textit{pid}, \textit{sig})

    \vspace{-1.5ex}

    \rule{\textwidth}{0.5\fboxrule}
\setlength{\parskip}{2ex}
    Kill a process with a signal.

\setlength{\parskip}{1ex}
    \end{boxedminipage}

    \label{os:killpg}
    \index{os.killpg \textit{(function)}}

    \vspace{0.5ex}

\hspace{.8\funcindent}\begin{boxedminipage}{\funcwidth}

    \raggedright \textbf{killpg}(\textit{pgid}, \textit{sig})

    \vspace{-1.5ex}

    \rule{\textwidth}{0.5\fboxrule}
\setlength{\parskip}{2ex}
    Kill a process group with a signal.

\setlength{\parskip}{1ex}
    \end{boxedminipage}

    \label{os:lchflags}
    \index{os.lchflags \textit{(function)}}

    \vspace{0.5ex}

\hspace{.8\funcindent}\begin{boxedminipage}{\funcwidth}

    \raggedright \textbf{lchflags}(\textit{path}, \textit{flags})

    \vspace{-1.5ex}

    \rule{\textwidth}{0.5\fboxrule}
\setlength{\parskip}{2ex}
    Set file flags. This function will not follow symbolic links.

\setlength{\parskip}{1ex}
    \end{boxedminipage}

    \label{os:lchmod}
    \index{os.lchmod \textit{(function)}}

    \vspace{0.5ex}

\hspace{.8\funcindent}\begin{boxedminipage}{\funcwidth}

    \raggedright \textbf{lchmod}(\textit{path}, \textit{mode})

    \vspace{-1.5ex}

    \rule{\textwidth}{0.5\fboxrule}
\setlength{\parskip}{2ex}
    Change the access permissions of a file. If path is a symlink, this 
    affects the link itself rather than the target.

\setlength{\parskip}{1ex}
    \end{boxedminipage}

    \label{os:lchown}
    \index{os.lchown \textit{(function)}}

    \vspace{0.5ex}

\hspace{.8\funcindent}\begin{boxedminipage}{\funcwidth}

    \raggedright \textbf{lchown}(\textit{path}, \textit{uid}, \textit{gid})

    \vspace{-1.5ex}

    \rule{\textwidth}{0.5\fboxrule}
\setlength{\parskip}{2ex}
    Change the owner and group id of path to the numeric uid and gid. This 
    function will not follow symbolic links.

\setlength{\parskip}{1ex}
    \end{boxedminipage}

    \label{os:link}
    \index{os.link \textit{(function)}}

    \vspace{0.5ex}

\hspace{.8\funcindent}\begin{boxedminipage}{\funcwidth}

    \raggedright \textbf{link}(\textit{src}, \textit{dst})

    \vspace{-1.5ex}

    \rule{\textwidth}{0.5\fboxrule}
\setlength{\parskip}{2ex}
    Create a hard link to a file.

\setlength{\parskip}{1ex}
    \end{boxedminipage}

    \label{os:listdir}
    \index{os.listdir \textit{(function)}}

    \vspace{0.5ex}

\hspace{.8\funcindent}\begin{boxedminipage}{\funcwidth}

    \raggedright \textbf{listdir}(\textit{path})

    \vspace{-1.5ex}

    \rule{\textwidth}{0.5\fboxrule}
\setlength{\parskip}{2ex}
\begin{alltt}
Return a list containing the names of the entries in the directory.

    path: path of directory to list

The list is in arbitrary order.  It does not include the special
entries '.' and '..' even if they are present in the directory.
\end{alltt}

\setlength{\parskip}{1ex}
      \textbf{Return Value}
    \vspace{-1ex}

      \begin{quote}
      list\_of\_strings

      \end{quote}

    \end{boxedminipage}

    \label{os:lseek}
    \index{os.lseek \textit{(function)}}

    \vspace{0.5ex}

\hspace{.8\funcindent}\begin{boxedminipage}{\funcwidth}

    \raggedright \textbf{lseek}(\textit{fd}, \textit{pos}, \textit{how})

    \vspace{-1.5ex}

    \rule{\textwidth}{0.5\fboxrule}
\setlength{\parskip}{2ex}
    Set the current position of a file descriptor. Return the new cursor 
    position in bytes, starting from the beginning.

\setlength{\parskip}{1ex}
      \textbf{Return Value}
    \vspace{-1ex}

      \begin{quote}
      newpos

      \end{quote}

    \end{boxedminipage}

    \label{os:lstat}
    \index{os.lstat \textit{(function)}}

    \vspace{0.5ex}

\hspace{.8\funcindent}\begin{boxedminipage}{\funcwidth}

    \raggedright \textbf{lstat}(\textit{path})

    \vspace{-1.5ex}

    \rule{\textwidth}{0.5\fboxrule}
\setlength{\parskip}{2ex}
    Like stat(path), but do not follow symbolic links.

\setlength{\parskip}{1ex}
      \textbf{Return Value}
    \vspace{-1ex}

      \begin{quote}
      stat result

      \end{quote}

    \end{boxedminipage}

    \label{os:major}
    \index{os.major \textit{(function)}}

    \vspace{0.5ex}

\hspace{.8\funcindent}\begin{boxedminipage}{\funcwidth}

    \raggedright \textbf{major}(\textit{device})

    \vspace{-1.5ex}

    \rule{\textwidth}{0.5\fboxrule}
\setlength{\parskip}{2ex}
    Extracts a device major number from a raw device number.

\setlength{\parskip}{1ex}
      \textbf{Return Value}
    \vspace{-1ex}

      \begin{quote}
      major number

      \end{quote}

    \end{boxedminipage}

    \label{os:makedev}
    \index{os.makedev \textit{(function)}}

    \vspace{0.5ex}

\hspace{.8\funcindent}\begin{boxedminipage}{\funcwidth}

    \raggedright \textbf{makedev}(\textit{major}, \textit{minor})

    \vspace{-1.5ex}

    \rule{\textwidth}{0.5\fboxrule}
\setlength{\parskip}{2ex}
    Composes a raw device number from the major and minor device numbers.

\setlength{\parskip}{1ex}
      \textbf{Return Value}
    \vspace{-1ex}

      \begin{quote}
      device number

      \end{quote}

    \end{boxedminipage}

    \label{os:makedirs}
    \index{os.makedirs \textit{(function)}}

    \vspace{0.5ex}

\hspace{.8\funcindent}\begin{boxedminipage}{\funcwidth}

    \raggedright \textbf{makedirs}(\textit{path}, \textit{mode}={\tt 0777})

    \vspace{-1.5ex}

    \rule{\textwidth}{0.5\fboxrule}
\setlength{\parskip}{2ex}
    Super-mkdir; create a leaf directory and all intermediate ones. Works 
    like mkdir, except that any intermediate path segment (not just the 
    rightmost) will be created if it does not exist.  This is recursive.

\setlength{\parskip}{1ex}
    \end{boxedminipage}

    \label{os:minor}
    \index{os.minor \textit{(function)}}

    \vspace{0.5ex}

\hspace{.8\funcindent}\begin{boxedminipage}{\funcwidth}

    \raggedright \textbf{minor}(\textit{device})

    \vspace{-1.5ex}

    \rule{\textwidth}{0.5\fboxrule}
\setlength{\parskip}{2ex}
    Extracts a device minor number from a raw device number.

\setlength{\parskip}{1ex}
      \textbf{Return Value}
    \vspace{-1ex}

      \begin{quote}
      minor number

      \end{quote}

    \end{boxedminipage}

    \label{os:mkdir}
    \index{os.mkdir \textit{(function)}}

    \vspace{0.5ex}

\hspace{.8\funcindent}\begin{boxedminipage}{\funcwidth}

    \raggedright \textbf{mkdir}(\textit{path}, \textit{mode}={\tt 0777})

    \vspace{-1.5ex}

    \rule{\textwidth}{0.5\fboxrule}
\setlength{\parskip}{2ex}
    Create a directory.

\setlength{\parskip}{1ex}
    \end{boxedminipage}

    \label{os:mkfifo}
    \index{os.mkfifo \textit{(function)}}

    \vspace{0.5ex}

\hspace{.8\funcindent}\begin{boxedminipage}{\funcwidth}

    \raggedright \textbf{mkfifo}(\textit{filename}, \textit{mode}={\tt 0666})

    \vspace{-1.5ex}

    \rule{\textwidth}{0.5\fboxrule}
\setlength{\parskip}{2ex}
    Create a FIFO (a POSIX named pipe).

\setlength{\parskip}{1ex}
    \end{boxedminipage}

    \label{os:mknod}
    \index{os.mknod \textit{(function)}}

    \vspace{0.5ex}

\hspace{.8\funcindent}\begin{boxedminipage}{\funcwidth}

    \raggedright \textbf{mknod}(\textit{filename}, \textit{mode}={\tt 0600}, \textit{device}={\tt ...})

    \vspace{-1.5ex}

    \rule{\textwidth}{0.5\fboxrule}
\setlength{\parskip}{2ex}
    Create a filesystem node (file, device special file or named pipe) 
    named filename. mode specifies both the permissions to use and the type
    of node to be created, being combined (bitwise OR) with one of 
    S\_IFREG, S\_IFCHR, S\_IFBLK, and S\_IFIFO. For S\_IFCHR and S\_IFBLK, 
    device defines the newly created device special file (probably using 
    os.makedev()), otherwise it is ignored.

\setlength{\parskip}{1ex}
    \end{boxedminipage}

    \label{os:nice}
    \index{os.nice \textit{(function)}}

    \vspace{0.5ex}

\hspace{.8\funcindent}\begin{boxedminipage}{\funcwidth}

    \raggedright \textbf{nice}(\textit{inc})

    \vspace{-1.5ex}

    \rule{\textwidth}{0.5\fboxrule}
\setlength{\parskip}{2ex}
    Decrease the priority of process by inc and return the new priority.

\setlength{\parskip}{1ex}
      \textbf{Return Value}
    \vspace{-1ex}

      \begin{quote}
      new\_priority

      \end{quote}

    \end{boxedminipage}

    \label{os:open}
    \index{os.open \textit{(function)}}

    \vspace{0.5ex}

\hspace{.8\funcindent}\begin{boxedminipage}{\funcwidth}

    \raggedright \textbf{open}(\textit{filename}, \textit{flag}, \textit{mode}={\tt 0777})

    \vspace{-1.5ex}

    \rule{\textwidth}{0.5\fboxrule}
\setlength{\parskip}{2ex}
    Open a file (for low level IO).

\setlength{\parskip}{1ex}
      \textbf{Return Value}
    \vspace{-1ex}

      \begin{quote}
      fd

      \end{quote}

    \end{boxedminipage}

    \label{os:openpty}
    \index{os.openpty \textit{(function)}}

    \vspace{0.5ex}

\hspace{.8\funcindent}\begin{boxedminipage}{\funcwidth}

    \raggedright \textbf{openpty}()

    \vspace{-1.5ex}

    \rule{\textwidth}{0.5\fboxrule}
\setlength{\parskip}{2ex}
    Open a pseudo-terminal, returning open fd's for both master and slave 
    end.

\setlength{\parskip}{1ex}
      \textbf{Return Value}
    \vspace{-1ex}

      \begin{quote}
      (master\_fd, slave\_fd)

      \end{quote}

    \end{boxedminipage}

    \label{os:pathconf}
    \index{os.pathconf \textit{(function)}}

    \vspace{0.5ex}

\hspace{.8\funcindent}\begin{boxedminipage}{\funcwidth}

    \raggedright \textbf{pathconf}(\textit{path}, \textit{name})

    \vspace{-1.5ex}

    \rule{\textwidth}{0.5\fboxrule}
\setlength{\parskip}{2ex}
    Return the configuration limit name for the file or directory path. If 
    there is no limit, return -1.

\setlength{\parskip}{1ex}
      \textbf{Return Value}
    \vspace{-1ex}

      \begin{quote}
      integer

      \end{quote}

    \end{boxedminipage}

    \label{os:pipe}
    \index{os.pipe \textit{(function)}}

    \vspace{0.5ex}

\hspace{.8\funcindent}\begin{boxedminipage}{\funcwidth}

    \raggedright \textbf{pipe}()

    \vspace{-1.5ex}

    \rule{\textwidth}{0.5\fboxrule}
\setlength{\parskip}{2ex}
    Create a pipe.

\setlength{\parskip}{1ex}
      \textbf{Return Value}
    \vspace{-1ex}

      \begin{quote}
      (read\_end, write\_end)

      \end{quote}

    \end{boxedminipage}

    \label{os:popen}
    \index{os.popen \textit{(function)}}

    \vspace{0.5ex}

\hspace{.8\funcindent}\begin{boxedminipage}{\funcwidth}

    \raggedright \textbf{popen}(\textit{command}, \textit{mode}={\tt 'r' }, \textit{bufsize}={\tt ...})

    \vspace{-1.5ex}

    \rule{\textwidth}{0.5\fboxrule}
\setlength{\parskip}{2ex}
    Open a pipe to/from a command returning a file object.

\setlength{\parskip}{1ex}
      \textbf{Return Value}
    \vspace{-1ex}

      \begin{quote}
      pipe

      \end{quote}

    \end{boxedminipage}

    \label{os:popen2}
    \index{os.popen2 \textit{(function)}}

    \vspace{0.5ex}

\hspace{.8\funcindent}\begin{boxedminipage}{\funcwidth}

    \raggedright \textbf{popen2}(\textit{cmd}, \textit{mode}={\tt \texttt{'}\texttt{t}\texttt{'}}, \textit{bufsize}={\tt -1})

    \vspace{-1.5ex}

    \rule{\textwidth}{0.5\fboxrule}
\setlength{\parskip}{2ex}
    Execute the shell command 'cmd' in a sub-process.  On UNIX, 'cmd' may 
    be a sequence, in which case arguments will be passed directly to the 
    program without shell intervention (as with os.spawnv()).  If 'cmd' is 
    a string it will be passed to the shell (as with os.system()). If 
    'bufsize' is specified, it sets the buffer size for the I/O pipes.  The
    file objects (child\_stdin, child\_stdout) are returned.

\setlength{\parskip}{1ex}
    \end{boxedminipage}

    \label{os:popen3}
    \index{os.popen3 \textit{(function)}}

    \vspace{0.5ex}

\hspace{.8\funcindent}\begin{boxedminipage}{\funcwidth}

    \raggedright \textbf{popen3}(\textit{cmd}, \textit{mode}={\tt \texttt{'}\texttt{t}\texttt{'}}, \textit{bufsize}={\tt -1})

    \vspace{-1.5ex}

    \rule{\textwidth}{0.5\fboxrule}
\setlength{\parskip}{2ex}
    Execute the shell command 'cmd' in a sub-process.  On UNIX, 'cmd' may 
    be a sequence, in which case arguments will be passed directly to the 
    program without shell intervention (as with os.spawnv()).  If 'cmd' is 
    a string it will be passed to the shell (as with os.system()). If 
    'bufsize' is specified, it sets the buffer size for the I/O pipes.  The
    file objects (child\_stdin, child\_stdout, child\_stderr) are returned.

\setlength{\parskip}{1ex}
    \end{boxedminipage}

    \label{os:popen4}
    \index{os.popen4 \textit{(function)}}

    \vspace{0.5ex}

\hspace{.8\funcindent}\begin{boxedminipage}{\funcwidth}

    \raggedright \textbf{popen4}(\textit{cmd}, \textit{mode}={\tt \texttt{'}\texttt{t}\texttt{'}}, \textit{bufsize}={\tt -1})

    \vspace{-1.5ex}

    \rule{\textwidth}{0.5\fboxrule}
\setlength{\parskip}{2ex}
    Execute the shell command 'cmd' in a sub-process.  On UNIX, 'cmd' may 
    be a sequence, in which case arguments will be passed directly to the 
    program without shell intervention (as with os.spawnv()).  If 'cmd' is 
    a string it will be passed to the shell (as with os.system()). If 
    'bufsize' is specified, it sets the buffer size for the I/O pipes.  The
    file objects (child\_stdin, child\_stdout\_stderr) are returned.

\setlength{\parskip}{1ex}
    \end{boxedminipage}

    \label{os:putenv}
    \index{os.putenv \textit{(function)}}

    \vspace{0.5ex}

\hspace{.8\funcindent}\begin{boxedminipage}{\funcwidth}

    \raggedright \textbf{putenv}(\textit{key}, \textit{value})

    \vspace{-1.5ex}

    \rule{\textwidth}{0.5\fboxrule}
\setlength{\parskip}{2ex}
    Change or add an environment variable.

\setlength{\parskip}{1ex}
    \end{boxedminipage}

    \label{os:read}
    \index{os.read \textit{(function)}}

    \vspace{0.5ex}

\hspace{.8\funcindent}\begin{boxedminipage}{\funcwidth}

    \raggedright \textbf{read}(\textit{fd}, \textit{buffersize})

    \vspace{-1.5ex}

    \rule{\textwidth}{0.5\fboxrule}
\setlength{\parskip}{2ex}
    Read a file descriptor.

\setlength{\parskip}{1ex}
      \textbf{Return Value}
    \vspace{-1ex}

      \begin{quote}
      string

      \end{quote}

    \end{boxedminipage}

    \label{os:readlink}
    \index{os.readlink \textit{(function)}}

    \vspace{0.5ex}

\hspace{.8\funcindent}\begin{boxedminipage}{\funcwidth}

    \raggedright \textbf{readlink}(\textit{path})

    \vspace{-1.5ex}

    \rule{\textwidth}{0.5\fboxrule}
\setlength{\parskip}{2ex}
    Return a string representing the path to which the symbolic link 
    points.

\setlength{\parskip}{1ex}
      \textbf{Return Value}
    \vspace{-1ex}

      \begin{quote}
      path

      \end{quote}

    \end{boxedminipage}

    \label{os:remove}
    \index{os.remove \textit{(function)}}

    \vspace{0.5ex}

\hspace{.8\funcindent}\begin{boxedminipage}{\funcwidth}

    \raggedright \textbf{remove}(\textit{path})

    \vspace{-1.5ex}

    \rule{\textwidth}{0.5\fboxrule}
\setlength{\parskip}{2ex}
    Remove a file (same as unlink(path)).

\setlength{\parskip}{1ex}
    \end{boxedminipage}

    \label{os:removedirs}
    \index{os.removedirs \textit{(function)}}

    \vspace{0.5ex}

\hspace{.8\funcindent}\begin{boxedminipage}{\funcwidth}

    \raggedright \textbf{removedirs}(\textit{path})

    \vspace{-1.5ex}

    \rule{\textwidth}{0.5\fboxrule}
\setlength{\parskip}{2ex}
    Super-rmdir; remove a leaf directory and all empty intermediate ones.  
    Works like rmdir except that, if the leaf directory is successfully 
    removed, directories corresponding to rightmost path segments will be 
    pruned away until either the whole path is consumed or an error occurs.
    Errors during this latter phase are ignored -- they generally mean that
    a directory was not empty.

\setlength{\parskip}{1ex}
    \end{boxedminipage}

    \label{os:rename}
    \index{os.rename \textit{(function)}}

    \vspace{0.5ex}

\hspace{.8\funcindent}\begin{boxedminipage}{\funcwidth}

    \raggedright \textbf{rename}(\textit{old}, \textit{new})

    \vspace{-1.5ex}

    \rule{\textwidth}{0.5\fboxrule}
\setlength{\parskip}{2ex}
    Rename a file or directory.

\setlength{\parskip}{1ex}
    \end{boxedminipage}

    \label{os:renames}
    \index{os.renames \textit{(function)}}

    \vspace{0.5ex}

\hspace{.8\funcindent}\begin{boxedminipage}{\funcwidth}

    \raggedright \textbf{renames}(\textit{old}, \textit{new})

    \vspace{-1.5ex}

    \rule{\textwidth}{0.5\fboxrule}
\setlength{\parskip}{2ex}
    Super-rename; create directories as necessary and delete any left 
    empty.  Works like rename, except creation of any intermediate 
    directories needed to make the new pathname good is attempted first.  
    After the rename, directories corresponding to rightmost path segments 
    of the old name will be pruned way until either the whole path is 
    consumed or a nonempty directory is found.

    Note: this function can fail with the new directory structure made if 
    you lack permissions needed to unlink the leaf directory or file.

\setlength{\parskip}{1ex}
    \end{boxedminipage}

    \label{os:rmdir}
    \index{os.rmdir \textit{(function)}}

    \vspace{0.5ex}

\hspace{.8\funcindent}\begin{boxedminipage}{\funcwidth}

    \raggedright \textbf{rmdir}(\textit{path})

    \vspace{-1.5ex}

    \rule{\textwidth}{0.5\fboxrule}
\setlength{\parskip}{2ex}
    Remove a directory.

\setlength{\parskip}{1ex}
    \end{boxedminipage}

    \label{os:setegid}
    \index{os.setegid \textit{(function)}}

    \vspace{0.5ex}

\hspace{.8\funcindent}\begin{boxedminipage}{\funcwidth}

    \raggedright \textbf{setegid}(\textit{gid})

    \vspace{-1.5ex}

    \rule{\textwidth}{0.5\fboxrule}
\setlength{\parskip}{2ex}
    Set the current process's effective group id.

\setlength{\parskip}{1ex}
    \end{boxedminipage}

    \label{os:seteuid}
    \index{os.seteuid \textit{(function)}}

    \vspace{0.5ex}

\hspace{.8\funcindent}\begin{boxedminipage}{\funcwidth}

    \raggedright \textbf{seteuid}(\textit{uid})

    \vspace{-1.5ex}

    \rule{\textwidth}{0.5\fboxrule}
\setlength{\parskip}{2ex}
    Set the current process's effective user id.

\setlength{\parskip}{1ex}
    \end{boxedminipage}

    \label{os:setgid}
    \index{os.setgid \textit{(function)}}

    \vspace{0.5ex}

\hspace{.8\funcindent}\begin{boxedminipage}{\funcwidth}

    \raggedright \textbf{setgid}(\textit{gid})

    \vspace{-1.5ex}

    \rule{\textwidth}{0.5\fboxrule}
\setlength{\parskip}{2ex}
    Set the current process's group id.

\setlength{\parskip}{1ex}
    \end{boxedminipage}

    \label{os:setgroups}
    \index{os.setgroups \textit{(function)}}

    \vspace{0.5ex}

\hspace{.8\funcindent}\begin{boxedminipage}{\funcwidth}

    \raggedright \textbf{setgroups}(\textit{list})

    \vspace{-1.5ex}

    \rule{\textwidth}{0.5\fboxrule}
\setlength{\parskip}{2ex}
    Set the groups of the current process to list.

\setlength{\parskip}{1ex}
    \end{boxedminipage}

    \label{os:setpgid}
    \index{os.setpgid \textit{(function)}}

    \vspace{0.5ex}

\hspace{.8\funcindent}\begin{boxedminipage}{\funcwidth}

    \raggedright \textbf{setpgid}(\textit{pid}, \textit{pgrp})

    \vspace{-1.5ex}

    \rule{\textwidth}{0.5\fboxrule}
\setlength{\parskip}{2ex}
    Call the system call setpgid().

\setlength{\parskip}{1ex}
    \end{boxedminipage}

    \label{os:setpgrp}
    \index{os.setpgrp \textit{(function)}}

    \vspace{0.5ex}

\hspace{.8\funcindent}\begin{boxedminipage}{\funcwidth}

    \raggedright \textbf{setpgrp}()

    \vspace{-1.5ex}

    \rule{\textwidth}{0.5\fboxrule}
\setlength{\parskip}{2ex}
    Make this process the process group leader.

\setlength{\parskip}{1ex}
    \end{boxedminipage}

    \label{os:setregid}
    \index{os.setregid \textit{(function)}}

    \vspace{0.5ex}

\hspace{.8\funcindent}\begin{boxedminipage}{\funcwidth}

    \raggedright \textbf{setregid}(\textit{rgid}, \textit{egid})

    \vspace{-1.5ex}

    \rule{\textwidth}{0.5\fboxrule}
\setlength{\parskip}{2ex}
    Set the current process's real and effective group ids.

\setlength{\parskip}{1ex}
    \end{boxedminipage}

    \label{os:setreuid}
    \index{os.setreuid \textit{(function)}}

    \vspace{0.5ex}

\hspace{.8\funcindent}\begin{boxedminipage}{\funcwidth}

    \raggedright \textbf{setreuid}(\textit{ruid}, \textit{euid})

    \vspace{-1.5ex}

    \rule{\textwidth}{0.5\fboxrule}
\setlength{\parskip}{2ex}
    Set the current process's real and effective user ids.

\setlength{\parskip}{1ex}
    \end{boxedminipage}

    \label{os:setsid}
    \index{os.setsid \textit{(function)}}

    \vspace{0.5ex}

\hspace{.8\funcindent}\begin{boxedminipage}{\funcwidth}

    \raggedright \textbf{setsid}()

    \vspace{-1.5ex}

    \rule{\textwidth}{0.5\fboxrule}
\setlength{\parskip}{2ex}
    Call the system call setsid().

\setlength{\parskip}{1ex}
    \end{boxedminipage}

    \label{os:setuid}
    \index{os.setuid \textit{(function)}}

    \vspace{0.5ex}

\hspace{.8\funcindent}\begin{boxedminipage}{\funcwidth}

    \raggedright \textbf{setuid}(\textit{uid})

    \vspace{-1.5ex}

    \rule{\textwidth}{0.5\fboxrule}
\setlength{\parskip}{2ex}
    Set the current process's user id.

\setlength{\parskip}{1ex}
    \end{boxedminipage}

    \label{os:spawnl}
    \index{os.spawnl \textit{(function)}}

    \vspace{0.5ex}

\hspace{.8\funcindent}\begin{boxedminipage}{\funcwidth}

    \raggedright \textbf{spawnl}(\textit{mode}, \textit{file}, *\textit{args})

    \vspace{-1.5ex}

    \rule{\textwidth}{0.5\fboxrule}
\setlength{\parskip}{2ex}
    Execute file with arguments from args in a subprocess. If mode == 
    P\_NOWAIT return the pid of the process. If mode == P\_WAIT return the 
    process's exit code if it exits normally; otherwise return -SIG, where 
    SIG is the signal that killed it.

\setlength{\parskip}{1ex}
      \textbf{Return Value}
    \vspace{-1ex}

      \begin{quote}
      integer

      \end{quote}

    \end{boxedminipage}

    \label{os:spawnle}
    \index{os.spawnle \textit{(function)}}

    \vspace{0.5ex}

\hspace{.8\funcindent}\begin{boxedminipage}{\funcwidth}

    \raggedright \textbf{spawnle}(\textit{mode}, \textit{file}, \textit{env}, *\textit{args})

    \vspace{-1.5ex}

    \rule{\textwidth}{0.5\fboxrule}
\setlength{\parskip}{2ex}
    Execute file with arguments from args in a subprocess with the supplied
    environment. If mode == P\_NOWAIT return the pid of the process. If 
    mode == P\_WAIT return the process's exit code if it exits normally; 
    otherwise return -SIG, where SIG is the signal that killed it.

\setlength{\parskip}{1ex}
      \textbf{Return Value}
    \vspace{-1ex}

      \begin{quote}
      integer

      \end{quote}

    \end{boxedminipage}

    \label{os:spawnlp}
    \index{os.spawnlp \textit{(function)}}

    \vspace{0.5ex}

\hspace{.8\funcindent}\begin{boxedminipage}{\funcwidth}

    \raggedright \textbf{spawnlp}(\textit{mode}, \textit{file}, *\textit{args})

    \vspace{-1.5ex}

    \rule{\textwidth}{0.5\fboxrule}
\setlength{\parskip}{2ex}
    Execute file (which is looked for along \$PATH) with arguments from 
    args in a subprocess with the supplied environment. If mode == 
    P\_NOWAIT return the pid of the process. If mode == P\_WAIT return the 
    process's exit code if it exits normally; otherwise return -SIG, where 
    SIG is the signal that killed it.

\setlength{\parskip}{1ex}
      \textbf{Return Value}
    \vspace{-1ex}

      \begin{quote}
      integer

      \end{quote}

    \end{boxedminipage}

    \label{os:spawnlpe}
    \index{os.spawnlpe \textit{(function)}}

    \vspace{0.5ex}

\hspace{.8\funcindent}\begin{boxedminipage}{\funcwidth}

    \raggedright \textbf{spawnlpe}(\textit{mode}, \textit{file}, \textit{env}, *\textit{args})

    \vspace{-1.5ex}

    \rule{\textwidth}{0.5\fboxrule}
\setlength{\parskip}{2ex}
    Execute file (which is looked for along \$PATH) with arguments from 
    args in a subprocess with the supplied environment. If mode == 
    P\_NOWAIT return the pid of the process. If mode == P\_WAIT return the 
    process's exit code if it exits normally; otherwise return -SIG, where 
    SIG is the signal that killed it.

\setlength{\parskip}{1ex}
      \textbf{Return Value}
    \vspace{-1ex}

      \begin{quote}
      integer

      \end{quote}

    \end{boxedminipage}

    \label{os:spawnv}
    \index{os.spawnv \textit{(function)}}

    \vspace{0.5ex}

\hspace{.8\funcindent}\begin{boxedminipage}{\funcwidth}

    \raggedright \textbf{spawnv}(\textit{mode}, \textit{file}, \textit{args})

    \vspace{-1.5ex}

    \rule{\textwidth}{0.5\fboxrule}
\setlength{\parskip}{2ex}
    Execute file with arguments from args in a subprocess. If mode == 
    P\_NOWAIT return the pid of the process. If mode == P\_WAIT return the 
    process's exit code if it exits normally; otherwise return -SIG, where 
    SIG is the signal that killed it.

\setlength{\parskip}{1ex}
      \textbf{Return Value}
    \vspace{-1ex}

      \begin{quote}
      integer

      \end{quote}

    \end{boxedminipage}

    \label{os:spawnve}
    \index{os.spawnve \textit{(function)}}

    \vspace{0.5ex}

\hspace{.8\funcindent}\begin{boxedminipage}{\funcwidth}

    \raggedright \textbf{spawnve}(\textit{mode}, \textit{file}, \textit{args}, \textit{env})

    \vspace{-1.5ex}

    \rule{\textwidth}{0.5\fboxrule}
\setlength{\parskip}{2ex}
    Execute file with arguments from args in a subprocess with the 
    specified environment. If mode == P\_NOWAIT return the pid of the 
    process. If mode == P\_WAIT return the process's exit code if it exits 
    normally; otherwise return -SIG, where SIG is the signal that killed 
    it.

\setlength{\parskip}{1ex}
      \textbf{Return Value}
    \vspace{-1ex}

      \begin{quote}
      integer

      \end{quote}

    \end{boxedminipage}

    \label{os:spawnvp}
    \index{os.spawnvp \textit{(function)}}

    \vspace{0.5ex}

\hspace{.8\funcindent}\begin{boxedminipage}{\funcwidth}

    \raggedright \textbf{spawnvp}(\textit{mode}, \textit{file}, \textit{args})

    \vspace{-1.5ex}

    \rule{\textwidth}{0.5\fboxrule}
\setlength{\parskip}{2ex}
    Execute file (which is looked for along \$PATH) with arguments from 
    args in a subprocess. If mode == P\_NOWAIT return the pid of the 
    process. If mode == P\_WAIT return the process's exit code if it exits 
    normally; otherwise return -SIG, where SIG is the signal that killed 
    it.

\setlength{\parskip}{1ex}
      \textbf{Return Value}
    \vspace{-1ex}

      \begin{quote}
      integer

      \end{quote}

    \end{boxedminipage}

    \label{os:spawnvpe}
    \index{os.spawnvpe \textit{(function)}}

    \vspace{0.5ex}

\hspace{.8\funcindent}\begin{boxedminipage}{\funcwidth}

    \raggedright \textbf{spawnvpe}(\textit{mode}, \textit{file}, \textit{args}, \textit{env})

    \vspace{-1.5ex}

    \rule{\textwidth}{0.5\fboxrule}
\setlength{\parskip}{2ex}
    Execute file (which is looked for along \$PATH) with arguments from 
    args in a subprocess with the supplied environment. If mode == 
    P\_NOWAIT return the pid of the process. If mode == P\_WAIT return the 
    process's exit code if it exits normally; otherwise return -SIG, where 
    SIG is the signal that killed it.

\setlength{\parskip}{1ex}
      \textbf{Return Value}
    \vspace{-1ex}

      \begin{quote}
      integer

      \end{quote}

    \end{boxedminipage}

    \label{os:stat}
    \index{os.stat \textit{(function)}}

    \vspace{0.5ex}

\hspace{.8\funcindent}\begin{boxedminipage}{\funcwidth}

    \raggedright \textbf{stat}(\textit{path})

    \vspace{-1.5ex}

    \rule{\textwidth}{0.5\fboxrule}
\setlength{\parskip}{2ex}
    Perform a stat system call on the given path.

\setlength{\parskip}{1ex}
      \textbf{Return Value}
    \vspace{-1ex}

      \begin{quote}
      stat result

      \end{quote}

    \end{boxedminipage}

    \label{os:stat_float_times}
    \index{os.stat\_float\_times \textit{(function)}}

    \vspace{0.5ex}

\hspace{.8\funcindent}\begin{boxedminipage}{\funcwidth}

    \raggedright \textbf{stat\_float\_times}(\textit{newval}={\tt ...})

    \vspace{-1.5ex}

    \rule{\textwidth}{0.5\fboxrule}
\setlength{\parskip}{2ex}
    Determine whether os.[lf]stat represents time stamps as float objects. 
    If newval is True, future calls to stat() return floats, if it is 
    False, future calls return ints. If newval is omitted, return the 
    current setting.

\setlength{\parskip}{1ex}
      \textbf{Return Value}
    \vspace{-1ex}

      \begin{quote}
      oldval

      \end{quote}

    \end{boxedminipage}

    \label{os:statvfs}
    \index{os.statvfs \textit{(function)}}

    \vspace{0.5ex}

\hspace{.8\funcindent}\begin{boxedminipage}{\funcwidth}

    \raggedright \textbf{statvfs}(\textit{path})

    \vspace{-1.5ex}

    \rule{\textwidth}{0.5\fboxrule}
\setlength{\parskip}{2ex}
    Perform a statvfs system call on the given path.

\setlength{\parskip}{1ex}
      \textbf{Return Value}
    \vspace{-1ex}

      \begin{quote}
      statvfs result

      \end{quote}

    \end{boxedminipage}

    \label{os:strerror}
    \index{os.strerror \textit{(function)}}

    \vspace{0.5ex}

\hspace{.8\funcindent}\begin{boxedminipage}{\funcwidth}

    \raggedright \textbf{strerror}(\textit{code})

    \vspace{-1.5ex}

    \rule{\textwidth}{0.5\fboxrule}
\setlength{\parskip}{2ex}
    Translate an error code to a message string.

\setlength{\parskip}{1ex}
      \textbf{Return Value}
    \vspace{-1ex}

      \begin{quote}
      string

      \end{quote}

    \end{boxedminipage}

    \label{os:symlink}
    \index{os.symlink \textit{(function)}}

    \vspace{0.5ex}

\hspace{.8\funcindent}\begin{boxedminipage}{\funcwidth}

    \raggedright \textbf{symlink}(\textit{src}, \textit{dst})

    \vspace{-1.5ex}

    \rule{\textwidth}{0.5\fboxrule}
\setlength{\parskip}{2ex}
    Create a symbolic link pointing to src named dst.

\setlength{\parskip}{1ex}
    \end{boxedminipage}

    \label{os:sysconf}
    \index{os.sysconf \textit{(function)}}

    \vspace{0.5ex}

\hspace{.8\funcindent}\begin{boxedminipage}{\funcwidth}

    \raggedright \textbf{sysconf}(\textit{name})

    \vspace{-1.5ex}

    \rule{\textwidth}{0.5\fboxrule}
\setlength{\parskip}{2ex}
    Return an integer-valued system configuration variable.

\setlength{\parskip}{1ex}
      \textbf{Return Value}
    \vspace{-1ex}

      \begin{quote}
      integer

      \end{quote}

    \end{boxedminipage}

    \label{os:system}
    \index{os.system \textit{(function)}}

    \vspace{0.5ex}

\hspace{.8\funcindent}\begin{boxedminipage}{\funcwidth}

    \raggedright \textbf{system}(\textit{command})

    \vspace{-1.5ex}

    \rule{\textwidth}{0.5\fboxrule}
\setlength{\parskip}{2ex}
    Execute the command (a string) in a subshell.

\setlength{\parskip}{1ex}
      \textbf{Return Value}
    \vspace{-1ex}

      \begin{quote}
      exit\_status

      \end{quote}

    \end{boxedminipage}

    \label{os:tcgetpgrp}
    \index{os.tcgetpgrp \textit{(function)}}

    \vspace{0.5ex}

\hspace{.8\funcindent}\begin{boxedminipage}{\funcwidth}

    \raggedright \textbf{tcgetpgrp}(\textit{fd})

    \vspace{-1.5ex}

    \rule{\textwidth}{0.5\fboxrule}
\setlength{\parskip}{2ex}
    Return the process group associated with the terminal given by a fd.

\setlength{\parskip}{1ex}
      \textbf{Return Value}
    \vspace{-1ex}

      \begin{quote}
      pgid

      \end{quote}

    \end{boxedminipage}

    \label{os:tcsetpgrp}
    \index{os.tcsetpgrp \textit{(function)}}

    \vspace{0.5ex}

\hspace{.8\funcindent}\begin{boxedminipage}{\funcwidth}

    \raggedright \textbf{tcsetpgrp}(\textit{fd}, \textit{pgid})

    \vspace{-1.5ex}

    \rule{\textwidth}{0.5\fboxrule}
\setlength{\parskip}{2ex}
    Set the process group associated with the terminal given by a fd.

\setlength{\parskip}{1ex}
    \end{boxedminipage}

    \label{os:tempnam}
    \index{os.tempnam \textit{(function)}}

    \vspace{0.5ex}

\hspace{.8\funcindent}\begin{boxedminipage}{\funcwidth}

    \raggedright \textbf{tempnam}(\textit{dir}={\tt ...}, \textit{prefix}={\tt ...})

    \vspace{-1.5ex}

    \rule{\textwidth}{0.5\fboxrule}
\setlength{\parskip}{2ex}
    Return a unique name for a temporary file. The directory and a prefix 
    may be specified as strings; they may be omitted or None if not needed.

\setlength{\parskip}{1ex}
      \textbf{Return Value}
    \vspace{-1ex}

      \begin{quote}
      string

      \end{quote}

    \end{boxedminipage}

    \label{os:times}
    \index{os.times \textit{(function)}}

    \vspace{0.5ex}

\hspace{.8\funcindent}\begin{boxedminipage}{\funcwidth}

    \raggedright \textbf{times}()

    \vspace{-1.5ex}

    \rule{\textwidth}{0.5\fboxrule}
\setlength{\parskip}{2ex}
    Return a tuple of floating point numbers indicating process times.

\setlength{\parskip}{1ex}
      \textbf{Return Value}
    \vspace{-1ex}

      \begin{quote}
      (utime, stime, cutime, cstime, elapsed\_time)

      \end{quote}

    \end{boxedminipage}

    \label{os:tmpfile}
    \index{os.tmpfile \textit{(function)}}

    \vspace{0.5ex}

\hspace{.8\funcindent}\begin{boxedminipage}{\funcwidth}

    \raggedright \textbf{tmpfile}()

    \vspace{-1.5ex}

    \rule{\textwidth}{0.5\fboxrule}
\setlength{\parskip}{2ex}
    Create a temporary file with no directory entries.

\setlength{\parskip}{1ex}
      \textbf{Return Value}
    \vspace{-1ex}

      \begin{quote}
      file object

      \end{quote}

    \end{boxedminipage}

    \label{os:tmpnam}
    \index{os.tmpnam \textit{(function)}}

    \vspace{0.5ex}

\hspace{.8\funcindent}\begin{boxedminipage}{\funcwidth}

    \raggedright \textbf{tmpnam}()

    \vspace{-1.5ex}

    \rule{\textwidth}{0.5\fboxrule}
\setlength{\parskip}{2ex}
    Return a unique name for a temporary file.

\setlength{\parskip}{1ex}
      \textbf{Return Value}
    \vspace{-1ex}

      \begin{quote}
      string

      \end{quote}

    \end{boxedminipage}

    \label{os:ttyname}
    \index{os.ttyname \textit{(function)}}

    \vspace{0.5ex}

\hspace{.8\funcindent}\begin{boxedminipage}{\funcwidth}

    \raggedright \textbf{ttyname}(\textit{fd})

    \vspace{-1.5ex}

    \rule{\textwidth}{0.5\fboxrule}
\setlength{\parskip}{2ex}
    Return the name of the terminal device connected to 'fd'.

\setlength{\parskip}{1ex}
      \textbf{Return Value}
    \vspace{-1ex}

      \begin{quote}
      string

      \end{quote}

    \end{boxedminipage}

    \label{os:umask}
    \index{os.umask \textit{(function)}}

    \vspace{0.5ex}

\hspace{.8\funcindent}\begin{boxedminipage}{\funcwidth}

    \raggedright \textbf{umask}(\textit{new\_mask})

    \vspace{-1.5ex}

    \rule{\textwidth}{0.5\fboxrule}
\setlength{\parskip}{2ex}
    Set the current numeric umask and return the previous umask.

\setlength{\parskip}{1ex}
      \textbf{Return Value}
    \vspace{-1ex}

      \begin{quote}
      old\_mask

      \end{quote}

    \end{boxedminipage}

    \label{os:uname}
    \index{os.uname \textit{(function)}}

    \vspace{0.5ex}

\hspace{.8\funcindent}\begin{boxedminipage}{\funcwidth}

    \raggedright \textbf{uname}()

    \vspace{-1.5ex}

    \rule{\textwidth}{0.5\fboxrule}
\setlength{\parskip}{2ex}
    Return a tuple identifying the current operating system.

\setlength{\parskip}{1ex}
      \textbf{Return Value}
    \vspace{-1ex}

      \begin{quote}
      (sysname, nodename, release, version, machine)

      \end{quote}

    \end{boxedminipage}

    \label{os:unlink}
    \index{os.unlink \textit{(function)}}

    \vspace{0.5ex}

\hspace{.8\funcindent}\begin{boxedminipage}{\funcwidth}

    \raggedright \textbf{unlink}(\textit{path})

    \vspace{-1.5ex}

    \rule{\textwidth}{0.5\fboxrule}
\setlength{\parskip}{2ex}
    Remove a file (same as remove(path)).

\setlength{\parskip}{1ex}
    \end{boxedminipage}

    \label{os:unsetenv}
    \index{os.unsetenv \textit{(function)}}

    \vspace{0.5ex}

\hspace{.8\funcindent}\begin{boxedminipage}{\funcwidth}

    \raggedright \textbf{unsetenv}(\textit{key})

    \vspace{-1.5ex}

    \rule{\textwidth}{0.5\fboxrule}
\setlength{\parskip}{2ex}
    Delete an environment variable.

\setlength{\parskip}{1ex}
    \end{boxedminipage}

    \label{os:urandom}
    \index{os.urandom \textit{(function)}}

    \vspace{0.5ex}

\hspace{.8\funcindent}\begin{boxedminipage}{\funcwidth}

    \raggedright \textbf{urandom}(\textit{n})

    \vspace{-1.5ex}

    \rule{\textwidth}{0.5\fboxrule}
\setlength{\parskip}{2ex}
    Return n random bytes suitable for cryptographic use.

\setlength{\parskip}{1ex}
      \textbf{Return Value}
    \vspace{-1ex}

      \begin{quote}
      str

      \end{quote}

    \end{boxedminipage}

    \label{os:utime}
    \index{os.utime \textit{(function)}}

    \vspace{0.5ex}

\hspace{.8\funcindent}\begin{boxedminipage}{\funcwidth}

    \raggedright \textbf{utime}(\textit{...})

    \vspace{-1.5ex}

    \rule{\textwidth}{0.5\fboxrule}
\setlength{\parskip}{2ex}
    utime(path, (atime, mtime)) utime(path, None)

    Set the access and modified time of the file to the given values.  If 
    the second form is used, set the access and modified times to the 
    current time.

\setlength{\parskip}{1ex}
    \end{boxedminipage}

    \label{os:wait}
    \index{os.wait \textit{(function)}}

    \vspace{0.5ex}

\hspace{.8\funcindent}\begin{boxedminipage}{\funcwidth}

    \raggedright \textbf{wait}()

    \vspace{-1.5ex}

    \rule{\textwidth}{0.5\fboxrule}
\setlength{\parskip}{2ex}
    Wait for completion of a child process.

\setlength{\parskip}{1ex}
      \textbf{Return Value}
    \vspace{-1ex}

      \begin{quote}
      (pid, status)

      \end{quote}

    \end{boxedminipage}

    \label{os:wait3}
    \index{os.wait3 \textit{(function)}}

    \vspace{0.5ex}

\hspace{.8\funcindent}\begin{boxedminipage}{\funcwidth}

    \raggedright \textbf{wait3}(\textit{options})

    \vspace{-1.5ex}

    \rule{\textwidth}{0.5\fboxrule}
\setlength{\parskip}{2ex}
    Wait for completion of a child process.

\setlength{\parskip}{1ex}
      \textbf{Return Value}
    \vspace{-1ex}

      \begin{quote}
      (pid, status, rusage)

      \end{quote}

    \end{boxedminipage}

    \label{os:wait4}
    \index{os.wait4 \textit{(function)}}

    \vspace{0.5ex}

\hspace{.8\funcindent}\begin{boxedminipage}{\funcwidth}

    \raggedright \textbf{wait4}(\textit{pid}, \textit{options})

    \vspace{-1.5ex}

    \rule{\textwidth}{0.5\fboxrule}
\setlength{\parskip}{2ex}
    Wait for completion of a given child process.

\setlength{\parskip}{1ex}
      \textbf{Return Value}
    \vspace{-1ex}

      \begin{quote}
      (pid, status, rusage)

      \end{quote}

    \end{boxedminipage}

    \label{os:waitpid}
    \index{os.waitpid \textit{(function)}}

    \vspace{0.5ex}

\hspace{.8\funcindent}\begin{boxedminipage}{\funcwidth}

    \raggedright \textbf{waitpid}(\textit{pid}, \textit{options})

    \vspace{-1.5ex}

    \rule{\textwidth}{0.5\fboxrule}
\setlength{\parskip}{2ex}
    Wait for completion of a given child process.

\setlength{\parskip}{1ex}
      \textbf{Return Value}
    \vspace{-1ex}

      \begin{quote}
      (pid, status)

      \end{quote}

    \end{boxedminipage}

    \label{os:walk}
    \index{os.walk \textit{(function)}}

    \vspace{0.5ex}

\hspace{.8\funcindent}\begin{boxedminipage}{\funcwidth}

    \raggedright \textbf{walk}(\textit{top}, \textit{topdown}={\tt True}, \textit{onerror}={\tt None}, \textit{followlinks}={\tt False})

    \vspace{-1.5ex}

    \rule{\textwidth}{0.5\fboxrule}
\setlength{\parskip}{2ex}
\begin{alltt}
Directory tree generator.

For each directory in the directory tree rooted at top (including top
itself, but excluding '.' and '..'), yields a 3-tuple

    dirpath, dirnames, filenames

dirpath is a string, the path to the directory.  dirnames is a list of
the names of the subdirectories in dirpath (excluding '.' and '..').
filenames is a list of the names of the non-directory files in dirpath.
Note that the names in the lists are just names, with no path components.
To get a full path (which begins with top) to a file or directory in
dirpath, do os.path.join(dirpath, name).

If optional arg 'topdown' is true or not specified, the triple for a
directory is generated before the triples for any of its subdirectories
(directories are generated top down).  If topdown is false, the triple
for a directory is generated after the triples for all of its
subdirectories (directories are generated bottom up).

When topdown is true, the caller can modify the dirnames list in-place
(e.g., via del or slice assignment), and walk will only recurse into the
subdirectories whose names remain in dirnames; this can be used to prune the
search, or to impose a specific order of visiting.  Modifying dirnames when
topdown is false is ineffective, since the directories in dirnames have
already been generated by the time dirnames itself is generated. No matter
the value of topdown, the list of subdirectories is retrieved before the
tuples for the directory and its subdirectories are generated.

By default errors from the os.listdir() call are ignored.  If
optional arg 'onerror' is specified, it should be a function; it
will be called with one argument, an os.error instance.  It can
report the error to continue with the walk, or raise the exception
to abort the walk.  Note that the filename is available as the
filename attribute of the exception object.

By default, os.walk does not follow symbolic links to subdirectories on
systems that support them.  In order to get this functionality, set the
optional argument 'followlinks' to true.

Caution:  if you pass a relative pathname for top, don't change the
current working directory between resumptions of walk.  walk never
changes the current directory, and assumes that the client doesn't
either.

Example:

import os
from os.path import join, getsize
for root, dirs, files in os.walk('python/Lib/email'):
    print root, "consumes",
    print sum([getsize(join(root, name)) for name in files]),
    print "bytes in", len(files), "non-directory files"
    if 'CVS' in dirs:
        dirs.remove('CVS')  \# don't visit CVS directories
\end{alltt}

\setlength{\parskip}{1ex}
    \end{boxedminipage}

    \label{os:write}
    \index{os.write \textit{(function)}}

    \vspace{0.5ex}

\hspace{.8\funcindent}\begin{boxedminipage}{\funcwidth}

    \raggedright \textbf{write}(\textit{fd}, \textit{string})

    \vspace{-1.5ex}

    \rule{\textwidth}{0.5\fboxrule}
\setlength{\parskip}{2ex}
    Write a string to a file descriptor.

\setlength{\parskip}{1ex}
      \textbf{Return Value}
    \vspace{-1ex}

      \begin{quote}
      byteswritten

      \end{quote}

    \end{boxedminipage}


%%%%%%%%%%%%%%%%%%%%%%%%%%%%%%%%%%%%%%%%%%%%%%%%%%%%%%%%%%%%%%%%%%%%%%%%%%%
%%                               Variables                               %%
%%%%%%%%%%%%%%%%%%%%%%%%%%%%%%%%%%%%%%%%%%%%%%%%%%%%%%%%%%%%%%%%%%%%%%%%%%%

  \subsection{Variables}

    \vspace{-1cm}
\hspace{\varindent}\begin{longtable}{|p{\varnamewidth}|p{\vardescrwidth}|l}
\cline{1-2}
\cline{1-2} \centering \textbf{Name} & \centering \textbf{Description}& \\
\cline{1-2}
\endhead\cline{1-2}\multicolumn{3}{r}{\small\textit{continued on next page}}\\\endfoot\cline{1-2}
\endlastfoot\raggedright E\-X\-\_\-C\-A\-N\-T\-C\-R\-E\-A\-T\- & \raggedright \textbf{Value:} 
{\tt 73}&\\
\cline{1-2}
\raggedright E\-X\-\_\-C\-O\-N\-F\-I\-G\- & \raggedright \textbf{Value:} 
{\tt 78}&\\
\cline{1-2}
\raggedright E\-X\-\_\-D\-A\-T\-A\-E\-R\-R\- & \raggedright \textbf{Value:} 
{\tt 65}&\\
\cline{1-2}
\raggedright E\-X\-\_\-I\-O\-E\-R\-R\- & \raggedright \textbf{Value:} 
{\tt 74}&\\
\cline{1-2}
\raggedright E\-X\-\_\-N\-O\-H\-O\-S\-T\- & \raggedright \textbf{Value:} 
{\tt 68}&\\
\cline{1-2}
\raggedright E\-X\-\_\-N\-O\-I\-N\-P\-U\-T\- & \raggedright \textbf{Value:} 
{\tt 66}&\\
\cline{1-2}
\raggedright E\-X\-\_\-N\-O\-P\-E\-R\-M\- & \raggedright \textbf{Value:} 
{\tt 77}&\\
\cline{1-2}
\raggedright E\-X\-\_\-N\-O\-U\-S\-E\-R\- & \raggedright \textbf{Value:} 
{\tt 67}&\\
\cline{1-2}
\raggedright E\-X\-\_\-O\-K\- & \raggedright \textbf{Value:} 
{\tt 0}&\\
\cline{1-2}
\raggedright E\-X\-\_\-O\-S\-E\-R\-R\- & \raggedright \textbf{Value:} 
{\tt 71}&\\
\cline{1-2}
\raggedright E\-X\-\_\-O\-S\-F\-I\-L\-E\- & \raggedright \textbf{Value:} 
{\tt 72}&\\
\cline{1-2}
\raggedright E\-X\-\_\-P\-R\-O\-T\-O\-C\-O\-L\- & \raggedright \textbf{Value:} 
{\tt 76}&\\
\cline{1-2}
\raggedright E\-X\-\_\-S\-O\-F\-T\-W\-A\-R\-E\- & \raggedright \textbf{Value:} 
{\tt 70}&\\
\cline{1-2}
\raggedright E\-X\-\_\-T\-E\-M\-P\-F\-A\-I\-L\- & \raggedright \textbf{Value:} 
{\tt 75}&\\
\cline{1-2}
\raggedright E\-X\-\_\-U\-N\-A\-V\-A\-I\-L\-A\-B\-L\-E\- & \raggedright \textbf{Value:} 
{\tt 69}&\\
\cline{1-2}
\raggedright E\-X\-\_\-U\-S\-A\-G\-E\- & \raggedright \textbf{Value:} 
{\tt 64}&\\
\cline{1-2}
\raggedright F\-\_\-O\-K\- & \raggedright \textbf{Value:} 
{\tt 0}&\\
\cline{1-2}
\raggedright N\-G\-R\-O\-U\-P\-S\-\_\-M\-A\-X\- & \raggedright \textbf{Value:} 
{\tt 16}&\\
\cline{1-2}
\raggedright O\-\_\-A\-P\-P\-E\-N\-D\- & \raggedright \textbf{Value:} 
{\tt 8}&\\
\cline{1-2}
\raggedright O\-\_\-A\-S\-Y\-N\-C\- & \raggedright \textbf{Value:} 
{\tt 64}&\\
\cline{1-2}
\raggedright O\-\_\-C\-R\-E\-A\-T\- & \raggedright \textbf{Value:} 
{\tt 512}&\\
\cline{1-2}
\raggedright O\-\_\-D\-I\-R\-E\-C\-T\-O\-R\-Y\- & \raggedright \textbf{Value:} 
{\tt 1048576}&\\
\cline{1-2}
\raggedright O\-\_\-D\-S\-Y\-N\-C\- & \raggedright \textbf{Value:} 
{\tt 4194304}&\\
\cline{1-2}
\raggedright O\-\_\-E\-X\-C\-L\- & \raggedright \textbf{Value:} 
{\tt 2048}&\\
\cline{1-2}
\raggedright O\-\_\-E\-X\-L\-O\-C\-K\- & \raggedright \textbf{Value:} 
{\tt 32}&\\
\cline{1-2}
\raggedright O\-\_\-N\-D\-E\-L\-A\-Y\- & \raggedright \textbf{Value:} 
{\tt 4}&\\
\cline{1-2}
\raggedright O\-\_\-N\-O\-C\-T\-T\-Y\- & \raggedright \textbf{Value:} 
{\tt 131072}&\\
\cline{1-2}
\raggedright O\-\_\-N\-O\-F\-O\-L\-L\-O\-W\- & \raggedright \textbf{Value:} 
{\tt 256}&\\
\cline{1-2}
\raggedright O\-\_\-N\-O\-N\-B\-L\-O\-C\-K\- & \raggedright \textbf{Value:} 
{\tt 4}&\\
\cline{1-2}
\raggedright O\-\_\-R\-D\-O\-N\-L\-Y\- & \raggedright \textbf{Value:} 
{\tt 0}&\\
\cline{1-2}
\raggedright O\-\_\-R\-D\-W\-R\- & \raggedright \textbf{Value:} 
{\tt 2}&\\
\cline{1-2}
\raggedright O\-\_\-S\-H\-L\-O\-C\-K\- & \raggedright \textbf{Value:} 
{\tt 16}&\\
\cline{1-2}
\raggedright O\-\_\-S\-Y\-N\-C\- & \raggedright \textbf{Value:} 
{\tt 128}&\\
\cline{1-2}
\raggedright O\-\_\-T\-R\-U\-N\-C\- & \raggedright \textbf{Value:} 
{\tt 1024}&\\
\cline{1-2}
\raggedright O\-\_\-W\-R\-O\-N\-L\-Y\- & \raggedright \textbf{Value:} 
{\tt 1}&\\
\cline{1-2}
\raggedright R\-\_\-O\-K\- & \raggedright \textbf{Value:} 
{\tt 4}&\\
\cline{1-2}
\raggedright S\-E\-E\-K\-\_\-C\-U\-R\- & \raggedright \textbf{Value:} 
{\tt 1}&\\
\cline{1-2}
\raggedright S\-E\-E\-K\-\_\-E\-N\-D\- & \raggedright \textbf{Value:} 
{\tt 2}&\\
\cline{1-2}
\raggedright S\-E\-E\-K\-\_\-S\-E\-T\- & \raggedright \textbf{Value:} 
{\tt 0}&\\
\cline{1-2}
\raggedright T\-M\-P\-\_\-M\-A\-X\- & \raggedright \textbf{Value:} 
{\tt 308915776}&\\
\cline{1-2}
\raggedright W\-C\-O\-N\-T\-I\-N\-U\-E\-D\- & \raggedright \textbf{Value:} 
{\tt 16}&\\
\cline{1-2}
\raggedright W\-N\-O\-H\-A\-N\-G\- & \raggedright \textbf{Value:} 
{\tt 1}&\\
\cline{1-2}
\raggedright W\-U\-N\-T\-R\-A\-C\-E\-D\- & \raggedright \textbf{Value:} 
{\tt 2}&\\
\cline{1-2}
\raggedright W\-\_\-O\-K\- & \raggedright \textbf{Value:} 
{\tt 2}&\\
\cline{1-2}
\raggedright X\-\_\-O\-K\- & \raggedright \textbf{Value:} 
{\tt 1}&\\
\cline{1-2}
\raggedright a\-l\-t\-s\-e\-p\- & \raggedright \textbf{Value:} 
{\tt None}&\\
\cline{1-2}
\raggedright c\-o\-n\-f\-s\-t\-r\-\_\-n\-a\-m\-e\-s\- & \raggedright \textbf{Value:} 
{\tt \texttt{\{}\texttt{'}\texttt{CS\_PATH}\texttt{'}\texttt{: }1\texttt{, }\texttt{'}\texttt{CS\_XBS5\_ILP32\_OFF32\_CFLAGS}\texttt{'}\texttt{: }20\texttt{, }\texttt{'}\texttt{CS\_XBS5}\texttt{...}}&\\
\cline{1-2}
\raggedright c\-u\-r\-d\-i\-r\- & \raggedright \textbf{Value:} 
{\tt \texttt{'}\texttt{.}\texttt{'}}&\\
\cline{1-2}
\raggedright d\-e\-f\-p\-a\-t\-h\- & \raggedright \textbf{Value:} 
{\tt \texttt{'}\texttt{:/bin:/usr/bin}\texttt{'}}&\\
\cline{1-2}
\raggedright d\-e\-v\-n\-u\-l\-l\- & \raggedright \textbf{Value:} 
{\tt \texttt{'}\texttt{/dev/null}\texttt{'}}&\\
\cline{1-2}
\raggedright e\-n\-v\-i\-r\-o\-n\- & \raggedright \textbf{Value:} 
{\tt \{'GOPATH': '/usr/local/opt/go/libexec/bin', 'HOME': '/Use\texttt{...}}&\\
\cline{1-2}
\raggedright e\-x\-t\-s\-e\-p\- & \raggedright \textbf{Value:} 
{\tt \texttt{'}\texttt{.}\texttt{'}}&\\
\cline{1-2}
\raggedright l\-i\-n\-e\-s\-e\-p\- & \raggedright \textbf{Value:} 
{\tt \texttt{'}\texttt{{\textbackslash}n}\texttt{'}}&\\
\cline{1-2}
\raggedright n\-a\-m\-e\- & \raggedright \textbf{Value:} 
{\tt \texttt{'}\texttt{posix}\texttt{'}}&\\
\cline{1-2}
\raggedright p\-a\-r\-d\-i\-r\- & \raggedright \textbf{Value:} 
{\tt \texttt{'}\texttt{..}\texttt{'}}&\\
\cline{1-2}
\raggedright p\-a\-t\-h\-c\-o\-n\-f\-\_\-n\-a\-m\-e\-s\- & \raggedright \textbf{Value:} 
{\tt \texttt{\{}\texttt{'}\texttt{PC\_ASYNC\_IO}\texttt{'}\texttt{: }17\texttt{, }\texttt{'}\texttt{PC\_CHOWN\_RESTRICTED}\texttt{'}\texttt{: }7\texttt{, }\texttt{'}\texttt{PC\_FILESIZ}\texttt{...}}&\\
\cline{1-2}
\raggedright p\-a\-t\-h\-s\-e\-p\- & \raggedright \textbf{Value:} 
{\tt \texttt{'}\texttt{:}\texttt{'}}&\\
\cline{1-2}
\raggedright s\-e\-p\- & \raggedright \textbf{Value:} 
{\tt \texttt{'}\texttt{/}\texttt{'}}&\\
\cline{1-2}
\raggedright s\-y\-s\-c\-o\-n\-f\-\_\-n\-a\-m\-e\-s\- & \raggedright \textbf{Value:} 
{\tt \texttt{\{}\texttt{'}\texttt{SC\_2\_CHAR\_TERM}\texttt{'}\texttt{: }20\texttt{, }\texttt{'}\texttt{SC\_2\_C\_BIND}\texttt{'}\texttt{: }18\texttt{, }\texttt{'}\texttt{SC\_2\_C\_DEV}\texttt{'}\texttt{: }1\texttt{...}}&\\
\cline{1-2}
\end{longtable}


%%%%%%%%%%%%%%%%%%%%%%%%%%%%%%%%%%%%%%%%%%%%%%%%%%%%%%%%%%%%%%%%%%%%%%%%%%%
%%                           Class Description                           %%
%%%%%%%%%%%%%%%%%%%%%%%%%%%%%%%%%%%%%%%%%%%%%%%%%%%%%%%%%%%%%%%%%%%%%%%%%%%

    \index{exceptions.OSError \textit{(class)}|(}
\subsection{Class OSError}

    \label{exceptions:OSError}
\begin{tabular}{cccccccccccccc}
% Line for object, linespec=[False, False, False, False, False]
\multicolumn{2}{r}{\settowidth{\BCL}{object}\multirow{2}{\BCL}{object}}
&&
&&
&&
&&
&&
  \\\cline{3-3}
  &&\multicolumn{1}{c|}{}
&&
&&
&&
&&
&&
  \\
% Line for exceptions.BaseException, linespec=[False, False, False, False]
\multicolumn{4}{r}{\settowidth{\BCL}{exceptions.BaseException}\multirow{2}{\BCL}{exceptions.BaseException}}
&&
&&
&&
&&
  \\\cline{5-5}
  &&&&\multicolumn{1}{c|}{}
&&
&&
&&
&&
  \\
% Line for exceptions.Exception, linespec=[False, False, False]
\multicolumn{6}{r}{\settowidth{\BCL}{exceptions.Exception}\multirow{2}{\BCL}{exceptions.Exception}}
&&
&&
&&
  \\\cline{7-7}
  &&&&&&\multicolumn{1}{c|}{}
&&
&&
&&
  \\
% Line for exceptions.StandardError, linespec=[False, False]
\multicolumn{8}{r}{\settowidth{\BCL}{exceptions.StandardError}\multirow{2}{\BCL}{exceptions.StandardError}}
&&
&&
  \\\cline{9-9}
  &&&&&&&&\multicolumn{1}{c|}{}
&&
&&
  \\
% Line for exceptions.EnvironmentError, linespec=[False]
\multicolumn{10}{r}{\settowidth{\BCL}{exceptions.EnvironmentError}\multirow{2}{\BCL}{exceptions.EnvironmentError}}
&&
  \\\cline{11-11}
  &&&&&&&&&&\multicolumn{1}{c|}{}
&&
  \\
&&&&&&&&&&\multicolumn{2}{l}{\textbf{exceptions.OSError}}
\end{tabular}

OS system call failed.


%%%%%%%%%%%%%%%%%%%%%%%%%%%%%%%%%%%%%%%%%%%%%%%%%%%%%%%%%%%%%%%%%%%%%%%%%%%
%%                                Methods                                %%
%%%%%%%%%%%%%%%%%%%%%%%%%%%%%%%%%%%%%%%%%%%%%%%%%%%%%%%%%%%%%%%%%%%%%%%%%%%

  \subsubsection{Methods}

    \vspace{0.5ex}

\hspace{.8\funcindent}\begin{boxedminipage}{\funcwidth}

    \raggedright \textbf{\_\_init\_\_}(\textit{...})

    \vspace{-1.5ex}

    \rule{\textwidth}{0.5\fboxrule}
\setlength{\parskip}{2ex}
    x.\_\_init\_\_(...) initializes x; see help(type(x)) for signature

\setlength{\parskip}{1ex}
      Overrides: object.\_\_init\_\_

    \end{boxedminipage}

    \vspace{0.5ex}

\hspace{.8\funcindent}\begin{boxedminipage}{\funcwidth}

    \raggedright \textbf{\_\_new\_\_}(\textit{T}, \textit{S}, \textit{...})

\setlength{\parskip}{2ex}
\setlength{\parskip}{1ex}
      \textbf{Return Value}
    \vspace{-1ex}

      \begin{quote}
      a new object with type S, a subtype of T

      \end{quote}

      Overrides: object.\_\_new\_\_

    \end{boxedminipage}


\large{\textbf{\textit{Inherited from exceptions.EnvironmentError}}}

\begin{quote}
\_\_reduce\_\_(), \_\_str\_\_()
\end{quote}

\large{\textbf{\textit{Inherited from exceptions.BaseException}}}

\begin{quote}
\_\_delattr\_\_(), \_\_getattribute\_\_(), \_\_getitem\_\_(), \_\_getslice\_\_(), \_\_repr\_\_(), \_\_setattr\_\_(), \_\_setstate\_\_(), \_\_unicode\_\_()
\end{quote}

\large{\textbf{\textit{Inherited from object}}}

\begin{quote}
\_\_format\_\_(), \_\_hash\_\_(), \_\_reduce\_ex\_\_(), \_\_sizeof\_\_(), \_\_subclasshook\_\_()
\end{quote}

%%%%%%%%%%%%%%%%%%%%%%%%%%%%%%%%%%%%%%%%%%%%%%%%%%%%%%%%%%%%%%%%%%%%%%%%%%%
%%                              Properties                               %%
%%%%%%%%%%%%%%%%%%%%%%%%%%%%%%%%%%%%%%%%%%%%%%%%%%%%%%%%%%%%%%%%%%%%%%%%%%%

  \subsubsection{Properties}

    \vspace{-1cm}
\hspace{\varindent}\begin{longtable}{|p{\varnamewidth}|p{\vardescrwidth}|l}
\cline{1-2}
\cline{1-2} \centering \textbf{Name} & \centering \textbf{Description}& \\
\cline{1-2}
\endhead\cline{1-2}\multicolumn{3}{r}{\small\textit{continued on next page}}\\\endfoot\cline{1-2}
\endlastfoot\multicolumn{2}{|l|}{\textit{Inherited from exceptions.EnvironmentError}}\\
\multicolumn{2}{|p{\varwidth}|}{\raggedright errno, filename, strerror}\\
\cline{1-2}
\multicolumn{2}{|l|}{\textit{Inherited from exceptions.BaseException}}\\
\multicolumn{2}{|p{\varwidth}|}{\raggedright args, message}\\
\cline{1-2}
\multicolumn{2}{|l|}{\textit{Inherited from object}}\\
\multicolumn{2}{|p{\varwidth}|}{\raggedright \_\_class\_\_}\\
\cline{1-2}
\end{longtable}

    \index{exceptions.OSError \textit{(class)}|)}

%%%%%%%%%%%%%%%%%%%%%%%%%%%%%%%%%%%%%%%%%%%%%%%%%%%%%%%%%%%%%%%%%%%%%%%%%%%
%%                           Class Description                           %%
%%%%%%%%%%%%%%%%%%%%%%%%%%%%%%%%%%%%%%%%%%%%%%%%%%%%%%%%%%%%%%%%%%%%%%%%%%%

    \index{posix.stat\_result \textit{(class)}|(}
\subsection{Class stat\_result}

    \label{posix:stat_result}
\begin{tabular}{cccccc}
% Line for object, linespec=[False]
\multicolumn{2}{r}{\settowidth{\BCL}{object}\multirow{2}{\BCL}{object}}
&&
  \\\cline{3-3}
  &&\multicolumn{1}{c|}{}
&&
  \\
&&\multicolumn{2}{l}{\textbf{posix.stat\_result}}
\end{tabular}

\begin{alltt}
stat\_result: Result from stat or lstat.

This object may be accessed either as a tuple of
  (mode, ino, dev, nlink, uid, gid, size, atime, mtime, ctime)
or via the attributes st\_mode, st\_ino, st\_dev, st\_nlink, st\_uid, and so on.

Posix/windows: If your platform supports st\_blksize, st\_blocks, st\_rdev,
or st\_flags, they are available as attributes only.

See os.stat for more information.
\end{alltt}


%%%%%%%%%%%%%%%%%%%%%%%%%%%%%%%%%%%%%%%%%%%%%%%%%%%%%%%%%%%%%%%%%%%%%%%%%%%
%%                                Methods                                %%
%%%%%%%%%%%%%%%%%%%%%%%%%%%%%%%%%%%%%%%%%%%%%%%%%%%%%%%%%%%%%%%%%%%%%%%%%%%

  \subsubsection{Methods}

    \label{posix:stat_result:__add__}
    \index{posix.stat\_result.\_\_add\_\_ \textit{(function)}}

    \vspace{0.5ex}

\hspace{.8\funcindent}\begin{boxedminipage}{\funcwidth}

    \raggedright \textbf{\_\_add\_\_}(\textit{x}, \textit{y})

    \vspace{-1.5ex}

    \rule{\textwidth}{0.5\fboxrule}
\setlength{\parskip}{2ex}
    x+y

\setlength{\parskip}{1ex}
    \end{boxedminipage}

    \label{posix:stat_result:__contains__}
    \index{posix.stat\_result.\_\_contains\_\_ \textit{(function)}}

    \vspace{0.5ex}

\hspace{.8\funcindent}\begin{boxedminipage}{\funcwidth}

    \raggedright \textbf{\_\_contains\_\_}(\textit{x}, \textit{y})

    \vspace{-1.5ex}

    \rule{\textwidth}{0.5\fboxrule}
\setlength{\parskip}{2ex}
    y in x

\setlength{\parskip}{1ex}
    \end{boxedminipage}

    \label{posix:stat_result:__eq__}
    \index{posix.stat\_result.\_\_eq\_\_ \textit{(function)}}

    \vspace{0.5ex}

\hspace{.8\funcindent}\begin{boxedminipage}{\funcwidth}

    \raggedright \textbf{\_\_eq\_\_}(\textit{x}, \textit{y})

    \vspace{-1.5ex}

    \rule{\textwidth}{0.5\fboxrule}
\setlength{\parskip}{2ex}
    x==y

\setlength{\parskip}{1ex}
    \end{boxedminipage}

    \label{posix:stat_result:__ge__}
    \index{posix.stat\_result.\_\_ge\_\_ \textit{(function)}}

    \vspace{0.5ex}

\hspace{.8\funcindent}\begin{boxedminipage}{\funcwidth}

    \raggedright \textbf{\_\_ge\_\_}(\textit{x}, \textit{y})

    \vspace{-1.5ex}

    \rule{\textwidth}{0.5\fboxrule}
\setlength{\parskip}{2ex}
    x{\textgreater}=y

\setlength{\parskip}{1ex}
    \end{boxedminipage}

    \label{posix:stat_result:__getitem__}
    \index{posix.stat\_result.\_\_getitem\_\_ \textit{(function)}}

    \vspace{0.5ex}

\hspace{.8\funcindent}\begin{boxedminipage}{\funcwidth}

    \raggedright \textbf{\_\_getitem\_\_}(\textit{x}, \textit{y})

    \vspace{-1.5ex}

    \rule{\textwidth}{0.5\fboxrule}
\setlength{\parskip}{2ex}
    x[y]

\setlength{\parskip}{1ex}
    \end{boxedminipage}

    \label{posix:stat_result:__getslice__}
    \index{posix.stat\_result.\_\_getslice\_\_ \textit{(function)}}

    \vspace{0.5ex}

\hspace{.8\funcindent}\begin{boxedminipage}{\funcwidth}

    \raggedright \textbf{\_\_getslice\_\_}(\textit{x}, \textit{i}, \textit{j})

    \vspace{-1.5ex}

    \rule{\textwidth}{0.5\fboxrule}
\setlength{\parskip}{2ex}
    x[i:j]

    Use of negative indices is not supported.

\setlength{\parskip}{1ex}
    \end{boxedminipage}

    \label{posix:stat_result:__gt__}
    \index{posix.stat\_result.\_\_gt\_\_ \textit{(function)}}

    \vspace{0.5ex}

\hspace{.8\funcindent}\begin{boxedminipage}{\funcwidth}

    \raggedright \textbf{\_\_gt\_\_}(\textit{x}, \textit{y})

    \vspace{-1.5ex}

    \rule{\textwidth}{0.5\fboxrule}
\setlength{\parskip}{2ex}
    x{\textgreater}y

\setlength{\parskip}{1ex}
    \end{boxedminipage}

    \vspace{0.5ex}

\hspace{.8\funcindent}\begin{boxedminipage}{\funcwidth}

    \raggedright \textbf{\_\_hash\_\_}(\textit{x})

    \vspace{-1.5ex}

    \rule{\textwidth}{0.5\fboxrule}
\setlength{\parskip}{2ex}
    hash(x)

\setlength{\parskip}{1ex}
      Overrides: object.\_\_hash\_\_

    \end{boxedminipage}

    \label{posix:stat_result:__le__}
    \index{posix.stat\_result.\_\_le\_\_ \textit{(function)}}

    \vspace{0.5ex}

\hspace{.8\funcindent}\begin{boxedminipage}{\funcwidth}

    \raggedright \textbf{\_\_le\_\_}(\textit{x}, \textit{y})

    \vspace{-1.5ex}

    \rule{\textwidth}{0.5\fboxrule}
\setlength{\parskip}{2ex}
    x{\textless}=y

\setlength{\parskip}{1ex}
    \end{boxedminipage}

    \label{posix:stat_result:__len__}
    \index{posix.stat\_result.\_\_len\_\_ \textit{(function)}}

    \vspace{0.5ex}

\hspace{.8\funcindent}\begin{boxedminipage}{\funcwidth}

    \raggedright \textbf{\_\_len\_\_}(\textit{x})

    \vspace{-1.5ex}

    \rule{\textwidth}{0.5\fboxrule}
\setlength{\parskip}{2ex}
    len(x)

\setlength{\parskip}{1ex}
    \end{boxedminipage}

    \label{posix:stat_result:__lt__}
    \index{posix.stat\_result.\_\_lt\_\_ \textit{(function)}}

    \vspace{0.5ex}

\hspace{.8\funcindent}\begin{boxedminipage}{\funcwidth}

    \raggedright \textbf{\_\_lt\_\_}(\textit{x}, \textit{y})

    \vspace{-1.5ex}

    \rule{\textwidth}{0.5\fboxrule}
\setlength{\parskip}{2ex}
    x{\textless}y

\setlength{\parskip}{1ex}
    \end{boxedminipage}

    \label{posix:stat_result:__mul__}
    \index{posix.stat\_result.\_\_mul\_\_ \textit{(function)}}

    \vspace{0.5ex}

\hspace{.8\funcindent}\begin{boxedminipage}{\funcwidth}

    \raggedright \textbf{\_\_mul\_\_}(\textit{x}, \textit{n})

    \vspace{-1.5ex}

    \rule{\textwidth}{0.5\fboxrule}
\setlength{\parskip}{2ex}
    x*n

\setlength{\parskip}{1ex}
    \end{boxedminipage}

    \label{posix:stat_result:__ne__}
    \index{posix.stat\_result.\_\_ne\_\_ \textit{(function)}}

    \vspace{0.5ex}

\hspace{.8\funcindent}\begin{boxedminipage}{\funcwidth}

    \raggedright \textbf{\_\_ne\_\_}(\textit{x}, \textit{y})

    \vspace{-1.5ex}

    \rule{\textwidth}{0.5\fboxrule}
\setlength{\parskip}{2ex}
    x!=y

\setlength{\parskip}{1ex}
    \end{boxedminipage}

    \vspace{0.5ex}

\hspace{.8\funcindent}\begin{boxedminipage}{\funcwidth}

    \raggedright \textbf{\_\_new\_\_}(\textit{T}, \textit{S}, \textit{...})

\setlength{\parskip}{2ex}
\setlength{\parskip}{1ex}
      \textbf{Return Value}
    \vspace{-1ex}

      \begin{quote}
      a new object with type S, a subtype of T

      \end{quote}

      Overrides: object.\_\_new\_\_

    \end{boxedminipage}

    \vspace{0.5ex}

\hspace{.8\funcindent}\begin{boxedminipage}{\funcwidth}

    \raggedright \textbf{\_\_reduce\_\_}(\textit{...})

\setlength{\parskip}{2ex}
    helper for pickle

\setlength{\parskip}{1ex}
      Overrides: object.\_\_reduce\_\_ 	extit{(inherited documentation)}

    \end{boxedminipage}

    \vspace{0.5ex}

\hspace{.8\funcindent}\begin{boxedminipage}{\funcwidth}

    \raggedright \textbf{\_\_repr\_\_}(\textit{x})

    \vspace{-1.5ex}

    \rule{\textwidth}{0.5\fboxrule}
\setlength{\parskip}{2ex}
    repr(x)

\setlength{\parskip}{1ex}
      Overrides: object.\_\_repr\_\_

    \end{boxedminipage}

    \label{posix:stat_result:__rmul__}
    \index{posix.stat\_result.\_\_rmul\_\_ \textit{(function)}}

    \vspace{0.5ex}

\hspace{.8\funcindent}\begin{boxedminipage}{\funcwidth}

    \raggedright \textbf{\_\_rmul\_\_}(\textit{x}, \textit{n})

    \vspace{-1.5ex}

    \rule{\textwidth}{0.5\fboxrule}
\setlength{\parskip}{2ex}
    n*x

\setlength{\parskip}{1ex}
    \end{boxedminipage}


\large{\textbf{\textit{Inherited from object}}}

\begin{quote}
\_\_delattr\_\_(), \_\_format\_\_(), \_\_getattribute\_\_(), \_\_init\_\_(), \_\_reduce\_ex\_\_(), \_\_setattr\_\_(), \_\_sizeof\_\_(), \_\_str\_\_(), \_\_subclasshook\_\_()
\end{quote}

%%%%%%%%%%%%%%%%%%%%%%%%%%%%%%%%%%%%%%%%%%%%%%%%%%%%%%%%%%%%%%%%%%%%%%%%%%%
%%                              Properties                               %%
%%%%%%%%%%%%%%%%%%%%%%%%%%%%%%%%%%%%%%%%%%%%%%%%%%%%%%%%%%%%%%%%%%%%%%%%%%%

  \subsubsection{Properties}

    \vspace{-1cm}
\hspace{\varindent}\begin{longtable}{|p{\varnamewidth}|p{\vardescrwidth}|l}
\cline{1-2}
\cline{1-2} \centering \textbf{Name} & \centering \textbf{Description}& \\
\cline{1-2}
\endhead\cline{1-2}\multicolumn{3}{r}{\small\textit{continued on next page}}\\\endfoot\cline{1-2}
\endlastfoot\raggedright s\-t\-\_\-a\-t\-i\-m\-e\- & \raggedright time of last access&\\
\cline{1-2}
\raggedright s\-t\-\_\-b\-i\-r\-t\-h\-t\-i\-m\-e\- & \raggedright time of creation&\\
\cline{1-2}
\raggedright s\-t\-\_\-b\-l\-k\-s\-i\-z\-e\- & \raggedright blocksize for filesystem I/O&\\
\cline{1-2}
\raggedright s\-t\-\_\-b\-l\-o\-c\-k\-s\- & \raggedright number of blocks allocated&\\
\cline{1-2}
\raggedright s\-t\-\_\-c\-t\-i\-m\-e\- & \raggedright time of last change&\\
\cline{1-2}
\raggedright s\-t\-\_\-d\-e\-v\- & \raggedright device&\\
\cline{1-2}
\raggedright s\-t\-\_\-f\-l\-a\-g\-s\- & \raggedright user defined flags for file&\\
\cline{1-2}
\raggedright s\-t\-\_\-g\-e\-n\- & \raggedright generation number&\\
\cline{1-2}
\raggedright s\-t\-\_\-g\-i\-d\- & \raggedright group ID of owner&\\
\cline{1-2}
\raggedright s\-t\-\_\-i\-n\-o\- & \raggedright inode&\\
\cline{1-2}
\raggedright s\-t\-\_\-m\-o\-d\-e\- & \raggedright protection bits&\\
\cline{1-2}
\raggedright s\-t\-\_\-m\-t\-i\-m\-e\- & \raggedright time of last modification&\\
\cline{1-2}
\raggedright s\-t\-\_\-n\-l\-i\-n\-k\- & \raggedright number of hard links&\\
\cline{1-2}
\raggedright s\-t\-\_\-r\-d\-e\-v\- & \raggedright device type (if inode device)&\\
\cline{1-2}
\raggedright s\-t\-\_\-s\-i\-z\-e\- & \raggedright total size, in bytes&\\
\cline{1-2}
\raggedright s\-t\-\_\-u\-i\-d\- & \raggedright user ID of owner&\\
\cline{1-2}
\multicolumn{2}{|l|}{\textit{Inherited from object}}\\
\multicolumn{2}{|p{\varwidth}|}{\raggedright \_\_class\_\_}\\
\cline{1-2}
\end{longtable}


%%%%%%%%%%%%%%%%%%%%%%%%%%%%%%%%%%%%%%%%%%%%%%%%%%%%%%%%%%%%%%%%%%%%%%%%%%%
%%                            Class Variables                            %%
%%%%%%%%%%%%%%%%%%%%%%%%%%%%%%%%%%%%%%%%%%%%%%%%%%%%%%%%%%%%%%%%%%%%%%%%%%%

  \subsubsection{Class Variables}

    \vspace{-1cm}
\hspace{\varindent}\begin{longtable}{|p{\varnamewidth}|p{\vardescrwidth}|l}
\cline{1-2}
\cline{1-2} \centering \textbf{Name} & \centering \textbf{Description}& \\
\cline{1-2}
\endhead\cline{1-2}\multicolumn{3}{r}{\small\textit{continued on next page}}\\\endfoot\cline{1-2}
\endlastfoot\raggedright n\-\_\-f\-i\-e\-l\-d\-s\- & \raggedright \textbf{Value:} 
{\tt 19}&\\
\cline{1-2}
\raggedright n\-\_\-s\-e\-q\-u\-e\-n\-c\-e\-\_\-f\-i\-e\-l\-d\-s\- & \raggedright \textbf{Value:} 
{\tt 10}&\\
\cline{1-2}
\raggedright n\-\_\-u\-n\-n\-a\-m\-e\-d\-\_\-f\-i\-e\-l\-d\-s\- & \raggedright \textbf{Value:} 
{\tt 3}&\\
\cline{1-2}
\end{longtable}

    \index{posix.stat\_result \textit{(class)}|)}

%%%%%%%%%%%%%%%%%%%%%%%%%%%%%%%%%%%%%%%%%%%%%%%%%%%%%%%%%%%%%%%%%%%%%%%%%%%
%%                           Class Description                           %%
%%%%%%%%%%%%%%%%%%%%%%%%%%%%%%%%%%%%%%%%%%%%%%%%%%%%%%%%%%%%%%%%%%%%%%%%%%%

    \index{posix.statvfs\_result \textit{(class)}|(}
\subsection{Class statvfs\_result}

    \label{posix:statvfs_result}
\begin{tabular}{cccccc}
% Line for object, linespec=[False]
\multicolumn{2}{r}{\settowidth{\BCL}{object}\multirow{2}{\BCL}{object}}
&&
  \\\cline{3-3}
  &&\multicolumn{1}{c|}{}
&&
  \\
&&\multicolumn{2}{l}{\textbf{posix.statvfs\_result}}
\end{tabular}

\begin{alltt}
statvfs\_result: Result from statvfs or fstatvfs.

This object may be accessed either as a tuple of
  (bsize, frsize, blocks, bfree, bavail, files, ffree, favail, flag, namemax),
or via the attributes f\_bsize, f\_frsize, f\_blocks, f\_bfree, and so on.

See os.statvfs for more information.
\end{alltt}


%%%%%%%%%%%%%%%%%%%%%%%%%%%%%%%%%%%%%%%%%%%%%%%%%%%%%%%%%%%%%%%%%%%%%%%%%%%
%%                                Methods                                %%
%%%%%%%%%%%%%%%%%%%%%%%%%%%%%%%%%%%%%%%%%%%%%%%%%%%%%%%%%%%%%%%%%%%%%%%%%%%

  \subsubsection{Methods}

    \label{posix:statvfs_result:__add__}
    \index{posix.statvfs\_result.\_\_add\_\_ \textit{(function)}}

    \vspace{0.5ex}

\hspace{.8\funcindent}\begin{boxedminipage}{\funcwidth}

    \raggedright \textbf{\_\_add\_\_}(\textit{x}, \textit{y})

    \vspace{-1.5ex}

    \rule{\textwidth}{0.5\fboxrule}
\setlength{\parskip}{2ex}
    x+y

\setlength{\parskip}{1ex}
    \end{boxedminipage}

    \label{posix:statvfs_result:__contains__}
    \index{posix.statvfs\_result.\_\_contains\_\_ \textit{(function)}}

    \vspace{0.5ex}

\hspace{.8\funcindent}\begin{boxedminipage}{\funcwidth}

    \raggedright \textbf{\_\_contains\_\_}(\textit{x}, \textit{y})

    \vspace{-1.5ex}

    \rule{\textwidth}{0.5\fboxrule}
\setlength{\parskip}{2ex}
    y in x

\setlength{\parskip}{1ex}
    \end{boxedminipage}

    \label{posix:statvfs_result:__eq__}
    \index{posix.statvfs\_result.\_\_eq\_\_ \textit{(function)}}

    \vspace{0.5ex}

\hspace{.8\funcindent}\begin{boxedminipage}{\funcwidth}

    \raggedright \textbf{\_\_eq\_\_}(\textit{x}, \textit{y})

    \vspace{-1.5ex}

    \rule{\textwidth}{0.5\fboxrule}
\setlength{\parskip}{2ex}
    x==y

\setlength{\parskip}{1ex}
    \end{boxedminipage}

    \label{posix:statvfs_result:__ge__}
    \index{posix.statvfs\_result.\_\_ge\_\_ \textit{(function)}}

    \vspace{0.5ex}

\hspace{.8\funcindent}\begin{boxedminipage}{\funcwidth}

    \raggedright \textbf{\_\_ge\_\_}(\textit{x}, \textit{y})

    \vspace{-1.5ex}

    \rule{\textwidth}{0.5\fboxrule}
\setlength{\parskip}{2ex}
    x{\textgreater}=y

\setlength{\parskip}{1ex}
    \end{boxedminipage}

    \label{posix:statvfs_result:__getitem__}
    \index{posix.statvfs\_result.\_\_getitem\_\_ \textit{(function)}}

    \vspace{0.5ex}

\hspace{.8\funcindent}\begin{boxedminipage}{\funcwidth}

    \raggedright \textbf{\_\_getitem\_\_}(\textit{x}, \textit{y})

    \vspace{-1.5ex}

    \rule{\textwidth}{0.5\fboxrule}
\setlength{\parskip}{2ex}
    x[y]

\setlength{\parskip}{1ex}
    \end{boxedminipage}

    \label{posix:statvfs_result:__getslice__}
    \index{posix.statvfs\_result.\_\_getslice\_\_ \textit{(function)}}

    \vspace{0.5ex}

\hspace{.8\funcindent}\begin{boxedminipage}{\funcwidth}

    \raggedright \textbf{\_\_getslice\_\_}(\textit{x}, \textit{i}, \textit{j})

    \vspace{-1.5ex}

    \rule{\textwidth}{0.5\fboxrule}
\setlength{\parskip}{2ex}
    x[i:j]

    Use of negative indices is not supported.

\setlength{\parskip}{1ex}
    \end{boxedminipage}

    \label{posix:statvfs_result:__gt__}
    \index{posix.statvfs\_result.\_\_gt\_\_ \textit{(function)}}

    \vspace{0.5ex}

\hspace{.8\funcindent}\begin{boxedminipage}{\funcwidth}

    \raggedright \textbf{\_\_gt\_\_}(\textit{x}, \textit{y})

    \vspace{-1.5ex}

    \rule{\textwidth}{0.5\fboxrule}
\setlength{\parskip}{2ex}
    x{\textgreater}y

\setlength{\parskip}{1ex}
    \end{boxedminipage}

    \vspace{0.5ex}

\hspace{.8\funcindent}\begin{boxedminipage}{\funcwidth}

    \raggedright \textbf{\_\_hash\_\_}(\textit{x})

    \vspace{-1.5ex}

    \rule{\textwidth}{0.5\fboxrule}
\setlength{\parskip}{2ex}
    hash(x)

\setlength{\parskip}{1ex}
      Overrides: object.\_\_hash\_\_

    \end{boxedminipage}

    \label{posix:statvfs_result:__le__}
    \index{posix.statvfs\_result.\_\_le\_\_ \textit{(function)}}

    \vspace{0.5ex}

\hspace{.8\funcindent}\begin{boxedminipage}{\funcwidth}

    \raggedright \textbf{\_\_le\_\_}(\textit{x}, \textit{y})

    \vspace{-1.5ex}

    \rule{\textwidth}{0.5\fboxrule}
\setlength{\parskip}{2ex}
    x{\textless}=y

\setlength{\parskip}{1ex}
    \end{boxedminipage}

    \label{posix:statvfs_result:__len__}
    \index{posix.statvfs\_result.\_\_len\_\_ \textit{(function)}}

    \vspace{0.5ex}

\hspace{.8\funcindent}\begin{boxedminipage}{\funcwidth}

    \raggedright \textbf{\_\_len\_\_}(\textit{x})

    \vspace{-1.5ex}

    \rule{\textwidth}{0.5\fboxrule}
\setlength{\parskip}{2ex}
    len(x)

\setlength{\parskip}{1ex}
    \end{boxedminipage}

    \label{posix:statvfs_result:__lt__}
    \index{posix.statvfs\_result.\_\_lt\_\_ \textit{(function)}}

    \vspace{0.5ex}

\hspace{.8\funcindent}\begin{boxedminipage}{\funcwidth}

    \raggedright \textbf{\_\_lt\_\_}(\textit{x}, \textit{y})

    \vspace{-1.5ex}

    \rule{\textwidth}{0.5\fboxrule}
\setlength{\parskip}{2ex}
    x{\textless}y

\setlength{\parskip}{1ex}
    \end{boxedminipage}

    \label{posix:statvfs_result:__mul__}
    \index{posix.statvfs\_result.\_\_mul\_\_ \textit{(function)}}

    \vspace{0.5ex}

\hspace{.8\funcindent}\begin{boxedminipage}{\funcwidth}

    \raggedright \textbf{\_\_mul\_\_}(\textit{x}, \textit{n})

    \vspace{-1.5ex}

    \rule{\textwidth}{0.5\fboxrule}
\setlength{\parskip}{2ex}
    x*n

\setlength{\parskip}{1ex}
    \end{boxedminipage}

    \label{posix:statvfs_result:__ne__}
    \index{posix.statvfs\_result.\_\_ne\_\_ \textit{(function)}}

    \vspace{0.5ex}

\hspace{.8\funcindent}\begin{boxedminipage}{\funcwidth}

    \raggedright \textbf{\_\_ne\_\_}(\textit{x}, \textit{y})

    \vspace{-1.5ex}

    \rule{\textwidth}{0.5\fboxrule}
\setlength{\parskip}{2ex}
    x!=y

\setlength{\parskip}{1ex}
    \end{boxedminipage}

    \vspace{0.5ex}

\hspace{.8\funcindent}\begin{boxedminipage}{\funcwidth}

    \raggedright \textbf{\_\_new\_\_}(\textit{T}, \textit{S}, \textit{...})

\setlength{\parskip}{2ex}
\setlength{\parskip}{1ex}
      \textbf{Return Value}
    \vspace{-1ex}

      \begin{quote}
      a new object with type S, a subtype of T

      \end{quote}

      Overrides: object.\_\_new\_\_

    \end{boxedminipage}

    \vspace{0.5ex}

\hspace{.8\funcindent}\begin{boxedminipage}{\funcwidth}

    \raggedright \textbf{\_\_reduce\_\_}(\textit{...})

\setlength{\parskip}{2ex}
    helper for pickle

\setlength{\parskip}{1ex}
      Overrides: object.\_\_reduce\_\_ 	extit{(inherited documentation)}

    \end{boxedminipage}

    \vspace{0.5ex}

\hspace{.8\funcindent}\begin{boxedminipage}{\funcwidth}

    \raggedright \textbf{\_\_repr\_\_}(\textit{x})

    \vspace{-1.5ex}

    \rule{\textwidth}{0.5\fboxrule}
\setlength{\parskip}{2ex}
    repr(x)

\setlength{\parskip}{1ex}
      Overrides: object.\_\_repr\_\_

    \end{boxedminipage}

    \label{posix:statvfs_result:__rmul__}
    \index{posix.statvfs\_result.\_\_rmul\_\_ \textit{(function)}}

    \vspace{0.5ex}

\hspace{.8\funcindent}\begin{boxedminipage}{\funcwidth}

    \raggedright \textbf{\_\_rmul\_\_}(\textit{x}, \textit{n})

    \vspace{-1.5ex}

    \rule{\textwidth}{0.5\fboxrule}
\setlength{\parskip}{2ex}
    n*x

\setlength{\parskip}{1ex}
    \end{boxedminipage}


\large{\textbf{\textit{Inherited from object}}}

\begin{quote}
\_\_delattr\_\_(), \_\_format\_\_(), \_\_getattribute\_\_(), \_\_init\_\_(), \_\_reduce\_ex\_\_(), \_\_setattr\_\_(), \_\_sizeof\_\_(), \_\_str\_\_(), \_\_subclasshook\_\_()
\end{quote}

%%%%%%%%%%%%%%%%%%%%%%%%%%%%%%%%%%%%%%%%%%%%%%%%%%%%%%%%%%%%%%%%%%%%%%%%%%%
%%                              Properties                               %%
%%%%%%%%%%%%%%%%%%%%%%%%%%%%%%%%%%%%%%%%%%%%%%%%%%%%%%%%%%%%%%%%%%%%%%%%%%%

  \subsubsection{Properties}

    \vspace{-1cm}
\hspace{\varindent}\begin{longtable}{|p{\varnamewidth}|p{\vardescrwidth}|l}
\cline{1-2}
\cline{1-2} \centering \textbf{Name} & \centering \textbf{Description}& \\
\cline{1-2}
\endhead\cline{1-2}\multicolumn{3}{r}{\small\textit{continued on next page}}\\\endfoot\cline{1-2}
\endlastfoot\raggedright f\-\_\-b\-a\-v\-a\-i\-l\- & &\\
\cline{1-2}
\raggedright f\-\_\-b\-f\-r\-e\-e\- & &\\
\cline{1-2}
\raggedright f\-\_\-b\-l\-o\-c\-k\-s\- & &\\
\cline{1-2}
\raggedright f\-\_\-b\-s\-i\-z\-e\- & &\\
\cline{1-2}
\raggedright f\-\_\-f\-a\-v\-a\-i\-l\- & &\\
\cline{1-2}
\raggedright f\-\_\-f\-f\-r\-e\-e\- & &\\
\cline{1-2}
\raggedright f\-\_\-f\-i\-l\-e\-s\- & &\\
\cline{1-2}
\raggedright f\-\_\-f\-l\-a\-g\- & &\\
\cline{1-2}
\raggedright f\-\_\-f\-r\-s\-i\-z\-e\- & &\\
\cline{1-2}
\raggedright f\-\_\-n\-a\-m\-e\-m\-a\-x\- & &\\
\cline{1-2}
\multicolumn{2}{|l|}{\textit{Inherited from object}}\\
\multicolumn{2}{|p{\varwidth}|}{\raggedright \_\_class\_\_}\\
\cline{1-2}
\end{longtable}


%%%%%%%%%%%%%%%%%%%%%%%%%%%%%%%%%%%%%%%%%%%%%%%%%%%%%%%%%%%%%%%%%%%%%%%%%%%
%%                            Class Variables                            %%
%%%%%%%%%%%%%%%%%%%%%%%%%%%%%%%%%%%%%%%%%%%%%%%%%%%%%%%%%%%%%%%%%%%%%%%%%%%

  \subsubsection{Class Variables}

    \vspace{-1cm}
\hspace{\varindent}\begin{longtable}{|p{\varnamewidth}|p{\vardescrwidth}|l}
\cline{1-2}
\cline{1-2} \centering \textbf{Name} & \centering \textbf{Description}& \\
\cline{1-2}
\endhead\cline{1-2}\multicolumn{3}{r}{\small\textit{continued on next page}}\\\endfoot\cline{1-2}
\endlastfoot\raggedright n\-\_\-f\-i\-e\-l\-d\-s\- & \raggedright \textbf{Value:} 
{\tt 10}&\\
\cline{1-2}
\raggedright n\-\_\-s\-e\-q\-u\-e\-n\-c\-e\-\_\-f\-i\-e\-l\-d\-s\- & \raggedright \textbf{Value:} 
{\tt 10}&\\
\cline{1-2}
\raggedright n\-\_\-u\-n\-n\-a\-m\-e\-d\-\_\-f\-i\-e\-l\-d\-s\- & \raggedright \textbf{Value:} 
{\tt 0}&\\
\cline{1-2}
\end{longtable}

    \index{posix.statvfs\_result \textit{(class)}|)}
    \index{os \textit{(module)}|)}
