%
% API Documentation for API Documentation
% Module codecs
%
% Generated by epydoc 3.0.1
% [Thu Dec 10 09:57:57 2015]
%

%%%%%%%%%%%%%%%%%%%%%%%%%%%%%%%%%%%%%%%%%%%%%%%%%%%%%%%%%%%%%%%%%%%%%%%%%%%
%%                          Module Description                           %%
%%%%%%%%%%%%%%%%%%%%%%%%%%%%%%%%%%%%%%%%%%%%%%%%%%%%%%%%%%%%%%%%%%%%%%%%%%%

    \index{codecs \textit{(module)}|(}
\section{Module codecs}

    \label{codecs}
codecs -- Python Codec Registry, API and helpers.

Written by Marc-Andre Lemburg (mal@lemburg.com).

(c) Copyright CNRI, All Rights Reserved. NO WARRANTY.


%%%%%%%%%%%%%%%%%%%%%%%%%%%%%%%%%%%%%%%%%%%%%%%%%%%%%%%%%%%%%%%%%%%%%%%%%%%
%%                               Functions                               %%
%%%%%%%%%%%%%%%%%%%%%%%%%%%%%%%%%%%%%%%%%%%%%%%%%%%%%%%%%%%%%%%%%%%%%%%%%%%

  \subsection{Functions}

    \label{codecs:EncodedFile}
    \index{codecs.EncodedFile \textit{(function)}}

    \vspace{0.5ex}

\hspace{.8\funcindent}\begin{boxedminipage}{\funcwidth}

    \raggedright \textbf{EncodedFile}(\textit{file}, \textit{data\_encoding}, \textit{file\_encoding}={\tt None}, \textit{errors}={\tt \texttt{'}\texttt{strict}\texttt{'}})

    \vspace{-1.5ex}

    \rule{\textwidth}{0.5\fboxrule}
\setlength{\parskip}{2ex}
    Return a wrapped version of file which provides transparent encoding 
    translation.

    Strings written to the wrapped file are interpreted according to the 
    given data\_encoding and then written to the original file as string 
    using file\_encoding. The intermediate encoding will usually be Unicode
    but depends on the specified codecs.

    Strings are read from the file using file\_encoding and then passed 
    back to the caller as string using data\_encoding.

    If file\_encoding is not given, it defaults to data\_encoding.

    errors may be given to define the error handling. It defaults to 
    'strict' which causes ValueErrors to be raised in case an encoding 
    error occurs.

    The returned wrapped file object provides two extra attributes 
    .data\_encoding and .file\_encoding which reflect the given parameters 
    of the same name. The attributes can be used for introspection by 
    Python programs.

\setlength{\parskip}{1ex}
    \end{boxedminipage}

    \label{codecs:ignore_errors}
    \index{codecs.ignore\_errors \textit{(function)}}

    \vspace{0.5ex}

\hspace{.8\funcindent}\begin{boxedminipage}{\funcwidth}

    \raggedright \textbf{ignore\_errors}(\textit{...})

    \vspace{-1.5ex}

    \rule{\textwidth}{0.5\fboxrule}
\setlength{\parskip}{2ex}
    Implements the 'ignore' error handling, which ignores malformed data 
    and continues.

\setlength{\parskip}{1ex}
    \end{boxedminipage}

    \label{codecs:lookup}
    \index{codecs.lookup \textit{(function)}}

    \vspace{0.5ex}

\hspace{.8\funcindent}\begin{boxedminipage}{\funcwidth}

    \raggedright \textbf{lookup}(\textit{encoding})

    \vspace{-1.5ex}

    \rule{\textwidth}{0.5\fboxrule}
\setlength{\parskip}{2ex}
    Looks up a codec tuple in the Python codec registry and returns a 
    CodecInfo object.

\setlength{\parskip}{1ex}
      \textbf{Return Value}
    \vspace{-1ex}

      \begin{quote}
      CodecInfo

      \end{quote}

    \end{boxedminipage}

    \label{codecs:lookup_error}
    \index{codecs.lookup\_error \textit{(function)}}

    \vspace{0.5ex}

\hspace{.8\funcindent}\begin{boxedminipage}{\funcwidth}

    \raggedright \textbf{lookup\_error}(\textit{errors})

    \vspace{-1.5ex}

    \rule{\textwidth}{0.5\fboxrule}
\setlength{\parskip}{2ex}
    Return the error handler for the specified error handling name or raise
    a LookupError, if no handler exists under this name.

\setlength{\parskip}{1ex}
      \textbf{Return Value}
    \vspace{-1ex}

      \begin{quote}
      handler

      \end{quote}

    \end{boxedminipage}

    \label{codecs:open}
    \index{codecs.open \textit{(function)}}

    \vspace{0.5ex}

\hspace{.8\funcindent}\begin{boxedminipage}{\funcwidth}

    \raggedright \textbf{open}(\textit{filename}, \textit{mode}={\tt \texttt{'}\texttt{rb}\texttt{'}}, \textit{encoding}={\tt None}, \textit{errors}={\tt \texttt{'}\texttt{strict}\texttt{'}}, \textit{buffering}={\tt 1})

    \vspace{-1.5ex}

    \rule{\textwidth}{0.5\fboxrule}
\setlength{\parskip}{2ex}
    Open an encoded file using the given mode and return a wrapped version 
    providing transparent encoding/decoding.

    Note: The wrapped version will only accept the object format defined by
    the codecs, i.e. Unicode objects for most builtin codecs. Output is 
    also codec dependent and will usually be Unicode as well.

    Files are always opened in binary mode, even if no binary mode was 
    specified. This is done to avoid data loss due to encodings using 8-bit
    values. The default file mode is 'rb' meaning to open the file in 
    binary read mode.

    encoding specifies the encoding which is to be used for the file.

    errors may be given to define the error handling. It defaults to 
    'strict' which causes ValueErrors to be raised in case an encoding 
    error occurs.

    buffering has the same meaning as for the builtin open() API. It 
    defaults to line buffered.

    The returned wrapped file object provides an extra attribute .encoding 
    which allows querying the used encoding. This attribute is only 
    available if an encoding was specified as parameter.

\setlength{\parskip}{1ex}
    \end{boxedminipage}

    \label{codecs:register}
    \index{codecs.register \textit{(function)}}

    \vspace{0.5ex}

\hspace{.8\funcindent}\begin{boxedminipage}{\funcwidth}

    \raggedright \textbf{register}(\textit{search\_function})

    \vspace{-1.5ex}

    \rule{\textwidth}{0.5\fboxrule}
\setlength{\parskip}{2ex}
    Register a codec search function. Search functions are expected to take
    one argument, the encoding name in all lower case letters, and return a
    tuple of functions (encoder, decoder, stream\_reader, stream\_writer) 
    (or a CodecInfo object).

\setlength{\parskip}{1ex}
    \end{boxedminipage}

    \label{codecs:register_error}
    \index{codecs.register\_error \textit{(function)}}

    \vspace{0.5ex}

\hspace{.8\funcindent}\begin{boxedminipage}{\funcwidth}

    \raggedright \textbf{register\_error}(\textit{errors}, \textit{handler})

    \vspace{-1.5ex}

    \rule{\textwidth}{0.5\fboxrule}
\setlength{\parskip}{2ex}
    Register the specified error handler under the name errors. handler 
    must be a callable object, that will be called with an exception 
    instance containing information about the location of the 
    encoding/decoding error and must return a (replacement, new position) 
    tuple.

\setlength{\parskip}{1ex}
    \end{boxedminipage}

    \label{codecs:replace_errors}
    \index{codecs.replace\_errors \textit{(function)}}

    \vspace{0.5ex}

\hspace{.8\funcindent}\begin{boxedminipage}{\funcwidth}

    \raggedright \textbf{replace\_errors}(\textit{...})

    \vspace{-1.5ex}

    \rule{\textwidth}{0.5\fboxrule}
\setlength{\parskip}{2ex}
    Implements the 'replace' error handling, which replaces malformed data 
    with a replacement marker.

\setlength{\parskip}{1ex}
    \end{boxedminipage}

    \label{codecs:strict_errors}
    \index{codecs.strict\_errors \textit{(function)}}

    \vspace{0.5ex}

\hspace{.8\funcindent}\begin{boxedminipage}{\funcwidth}

    \raggedright \textbf{strict\_errors}(\textit{...})

    \vspace{-1.5ex}

    \rule{\textwidth}{0.5\fboxrule}
\setlength{\parskip}{2ex}
    Implements the 'strict' error handling, which raises a UnicodeError on 
    coding errors.

\setlength{\parskip}{1ex}
    \end{boxedminipage}

    \label{codecs:xmlcharrefreplace_errors}
    \index{codecs.xmlcharrefreplace\_errors \textit{(function)}}

    \vspace{0.5ex}

\hspace{.8\funcindent}\begin{boxedminipage}{\funcwidth}

    \raggedright \textbf{xmlcharrefreplace\_errors}(\textit{...})

    \vspace{-1.5ex}

    \rule{\textwidth}{0.5\fboxrule}
\setlength{\parskip}{2ex}
    Implements the 'xmlcharrefreplace' error handling, which replaces an 
    unencodable character with the appropriate XML character reference.

\setlength{\parskip}{1ex}
    \end{boxedminipage}


%%%%%%%%%%%%%%%%%%%%%%%%%%%%%%%%%%%%%%%%%%%%%%%%%%%%%%%%%%%%%%%%%%%%%%%%%%%
%%                               Variables                               %%
%%%%%%%%%%%%%%%%%%%%%%%%%%%%%%%%%%%%%%%%%%%%%%%%%%%%%%%%%%%%%%%%%%%%%%%%%%%

  \subsection{Variables}

    \vspace{-1cm}
\hspace{\varindent}\begin{longtable}{|p{\varnamewidth}|p{\vardescrwidth}|l}
\cline{1-2}
\cline{1-2} \centering \textbf{Name} & \centering \textbf{Description}& \\
\cline{1-2}
\endhead\cline{1-2}\multicolumn{3}{r}{\small\textit{continued on next page}}\\\endfoot\cline{1-2}
\endlastfoot\raggedright B\-O\-M\- & \raggedright \textbf{Value:} 
{\tt \texttt{'}\texttt{{\textbackslash}xff{\textbackslash}xfe}\texttt{'}}&\\
\cline{1-2}
\raggedright B\-O\-M\-3\-2\-\_\-B\-E\- & \raggedright \textbf{Value:} 
{\tt \texttt{'}\texttt{{\textbackslash}xfe{\textbackslash}xff}\texttt{'}}&\\
\cline{1-2}
\raggedright B\-O\-M\-3\-2\-\_\-L\-E\- & \raggedright \textbf{Value:} 
{\tt \texttt{'}\texttt{{\textbackslash}xff{\textbackslash}xfe}\texttt{'}}&\\
\cline{1-2}
\raggedright B\-O\-M\-6\-4\-\_\-B\-E\- & \raggedright \textbf{Value:} 
{\tt \texttt{'}\texttt{{\textbackslash}x00{\textbackslash}x00{\textbackslash}xfe{\textbackslash}xff}\texttt{'}}&\\
\cline{1-2}
\raggedright B\-O\-M\-6\-4\-\_\-L\-E\- & \raggedright \textbf{Value:} 
{\tt \texttt{'}\texttt{{\textbackslash}xff{\textbackslash}xfe{\textbackslash}x00{\textbackslash}x00}\texttt{'}}&\\
\cline{1-2}
\raggedright B\-O\-M\-\_\-B\-E\- & \raggedright \textbf{Value:} 
{\tt \texttt{'}\texttt{{\textbackslash}xfe{\textbackslash}xff}\texttt{'}}&\\
\cline{1-2}
\raggedright B\-O\-M\-\_\-L\-E\- & \raggedright \textbf{Value:} 
{\tt \texttt{'}\texttt{{\textbackslash}xff{\textbackslash}xfe}\texttt{'}}&\\
\cline{1-2}
\raggedright B\-O\-M\-\_\-U\-T\-F\-1\-6\- & \raggedright \textbf{Value:} 
{\tt \texttt{'}\texttt{{\textbackslash}xff{\textbackslash}xfe}\texttt{'}}&\\
\cline{1-2}
\raggedright B\-O\-M\-\_\-U\-T\-F\-1\-6\-\_\-B\-E\- & \raggedright \textbf{Value:} 
{\tt \texttt{'}\texttt{{\textbackslash}xfe{\textbackslash}xff}\texttt{'}}&\\
\cline{1-2}
\raggedright B\-O\-M\-\_\-U\-T\-F\-1\-6\-\_\-L\-E\- & \raggedright \textbf{Value:} 
{\tt \texttt{'}\texttt{{\textbackslash}xff{\textbackslash}xfe}\texttt{'}}&\\
\cline{1-2}
\raggedright B\-O\-M\-\_\-U\-T\-F\-3\-2\- & \raggedright \textbf{Value:} 
{\tt \texttt{'}\texttt{{\textbackslash}xff{\textbackslash}xfe{\textbackslash}x00{\textbackslash}x00}\texttt{'}}&\\
\cline{1-2}
\raggedright B\-O\-M\-\_\-U\-T\-F\-3\-2\-\_\-B\-E\- & \raggedright \textbf{Value:} 
{\tt \texttt{'}\texttt{{\textbackslash}x00{\textbackslash}x00{\textbackslash}xfe{\textbackslash}xff}\texttt{'}}&\\
\cline{1-2}
\raggedright B\-O\-M\-\_\-U\-T\-F\-3\-2\-\_\-L\-E\- & \raggedright \textbf{Value:} 
{\tt \texttt{'}\texttt{{\textbackslash}xff{\textbackslash}xfe{\textbackslash}x00{\textbackslash}x00}\texttt{'}}&\\
\cline{1-2}
\raggedright B\-O\-M\-\_\-U\-T\-F\-8\- & \raggedright \textbf{Value:} 
{\tt \texttt{'}\texttt{{\textbackslash}xef{\textbackslash}xbb{\textbackslash}xbf}\texttt{'}}&\\
\cline{1-2}
\end{longtable}

    \index{codecs \textit{(module)}|)}
